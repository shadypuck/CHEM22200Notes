\documentclass[../notes.tex]{subfiles}

\pagestyle{main}
\renewcommand{\chaptermark}[1]{\markboth{\chaptername\ \thechapter\ (#1)}{}}
\setcounter{chapter}{4}

\begin{document}




\chapter{Alpha-Carbon Reactions}
\section[Reactions at the \texorpdfstring{$\alpha$}{TEXT}-Carbon of Carbonyl Compounds 2]{Reactions at the \texorpdfstring{$\bm{\alpha}$}{TEXT}-Carbon of Carbonyl Compounds 2}
\begin{itemize}
    \item \marginnote{4/26:}Announcements.
    \begin{itemize}
        \item Professor Tang starts next Tuesday.
        \item Midterm 2 will be written by Levin.
        \item PSet 4-6 and the final will be written by Tang (she will release practice exams).
    \end{itemize}
    \item Midterm 1 stats.
    \begin{itemize}
        \item Range: 0-91.
        \item Mean/st. dev: $34\pm 20$ recurved to $70\pm 10$.
        \item Median: 33.
        \item Nobody got 3a, the first mechanism.
        \item Such recurving will be done for all exams.
    \end{itemize}
    \item Midterm 1 comments.
    \begin{itemize}
        \item This is Levin's first time teaching undergrads. As an undergrad, he had a professor for whom it was their first time and it was brutal for him, so he said he wouldn't do that but accidentally did it regardless.
        \item Levin also says that for all the people who feel like they don't know what's going on, that's on him.
        \item If you wanna judge how good you're doing, see how you did on the cyanohydrin formation and the amine cyclization. If those felt ok, you're doing fine; you can consider the others to have been challenge problems.
    \end{itemize}
    \item Reversible formation of enols and enolates.
    \begin{itemize}
        \item As discussed in the previous lecture, a ketone in the presence of a hydroxide base will equilibriate with its enolate.
        \item Since $\pKa=25$ for the ketone and $\pKa=15$ for the enolate, $10^{10}$ times more of the ketone is present in solution.
        \item Note that the amount of enolate present is still sufficient to do some chemistry (like that which we discussed last time). It does beg the question, however, of how stoichiometric deprotonation can be accomplished.
        \item Stoichiometric deprotonation is useful (and necessary) for the reaction of enolates with relatively weaker electrophiles.
    \end{itemize}
    \item Stoichiometric deprotonation.
    \item In theory, we could just use a stronger base.
    \begin{itemize}
        \item We might assume that nBuLi\footnote{Also pronounced "BYOO-lee".} will deprotontate ketones to form butane and the enolate (with a lithium countercation).
        \item Since butane is so basic, this would work very well ($K\approx 10^{25}$). However, nBuLi has competitive reactivity as a nucleophile attacking the carbonyl, and this is what it will do (as we discussed last unit).
        \item Thus, we need an \textbf{innocent base}.
    \end{itemize}
    \item \textbf{Innocent base}: A base that does not have reactivity competing with its ability to do deprotonations.
    \item \textbf{LDA}: Lithium diisopropyl amide, a sterically hindered, very strong, innocent base. \emph{Structure}
    \begin{figure}[h!]
        \centering
        \footnotesize
        \chemfig{\charge{90:3pt=$\ominus$}{N}(-[,0.3,,,white]\charge{90:3pt=$\oplus$}{Li})(-[:120](-[::60])(-[::-60]))(-[:-120](-[::60])(-[::-60]))}
        \caption{Lithium diisopropyl amide (LDA).}
        \label{fig:LDA}
    \end{figure}
    \begin{itemize}
        \item One implication of the name of this compound is that the term "amide" refers to both the carboxylic acid derivatives of nitrogen (see Figure \ref{fig:carboxylicAcidDerivativese}) and deprotonated amines (such as LDA).
        \item Some chemists proclaim that there is a difference in pronunciation, i.e., that one is pronounced "AM-id" and the other "AE-mide."
    \end{itemize}
    \item Synthesis of LDA.
    \begin{figure}[h!]
        \centering
        \footnotesize
        \schemestart
            \chemfig{-[:30]-[:-30]-[:30]-[:-30]Li}
            \+{1em,1em}
            \chemfig{N(-H)(-[:120](-[::60])(-[::-60]))(-[:-120](-[::60])(-[::-60]))}
            \arrow
            \chemfig{-[:30]-[:-30]-[:30]}
            \+{1em,1em}
            \chemfig{\charge{90:3pt=$\ominus$}{N}(-[,0.3,,,white]\charge{90:3pt=$\oplus$}{Li})(-[:120](-[::60])(-[::-60]))(-[:-120](-[::60])(-[::-60]))}
        \schemestop
        \caption{Synthesizing LDA.}
        \label{fig:LDASynthesis}
    \end{figure}
    \begin{itemize}
        \item The reactants are n-butyl lithium and diisopropyl amine.
    \end{itemize}
    \item Consider what would happen if LDA tried to act as a nucleophile.
    \begin{itemize}
        \item The product would be sterically disfavored.
        \item Additionally, there is an easy reversible mechanism because while we an alkoxide can't kick out carbon, the amide is a good leaving group.
        \item Thus, this is a reversible reaction that favors the starting material.
    \end{itemize}
    \item Since LDA has no competitive reactivity, it will stoichiometrically deprotonate ketones.
    \begin{itemize}
        \item Consider the reaciton of methyl phenyl ketone and LDA.
        \item Since the ketone has $\pKa\approx 25$ and diisopropylamine has $\pKa\approx 36$, the equilibrium constant is approximately $10^{11}$.
    \end{itemize}
    \item Orbital effects for deprotonation.
    \begin{figure}[H]
        \centering
        \begin{tikzpicture}[
            every node/.style=black
        ]
            \footnotesize
            \begin{scope}[xshift=-3cm]
                \fill [ory] (0,0.85) circle (3.5mm)
                    node[left=3.5mm]{$s$}
                ;
                \filldraw [thick,draw=orx,fill=ory] (0,0)      to[out=120,in=60,looseness=250]   ++(0.01,0);
                \filldraw [thick,draw=orx,fill=ory] (0,0)      to[out=-120,in=-60,looseness=100] ++(0.01,0)
                    node[below=2.5mm]{$sp^3$}
                ;
                \filldraw [thick,draw=orx,fill=ory] (30:0.8)   to[out=110,in=70,looseness=300]   ++(0.01,0)
                    node[above=8mm]{$p$}
                ;
                \filldraw [thick,draw=orx,fill=ory] (30:0.8)   to[out=-110,in=-70,looseness=300] ++(0.01,0);
                \filldraw [thick,draw=orx,fill=ory] (1.7,0.52) to[out=110,in=70,looseness=250]   ++(0.01,0)
                    node[above=6.5mm]{$p$}
                ;
                \filldraw [thick,draw=orx,fill=ory] (1.7,0.15) to[out=-110,in=-70,looseness=250] ++(0.01,0);
    
                \draw
                    (90:0.6) node[above=-1mm,circle,thick,draw=orx,inner sep=4pt]{\ce{H}}
                        -- (0,0)
                        -- (30:0.8)
                        -- ++(-5:0.8) node[right]{\ce{O}}
                    (30:0.8) ++(30:0.1) -- ++(-5:0.75)
                ;
    
                \node at (-1.8,1.5) {\chemfig{\charge{0=\:,45:1pt=$\ominus$}{N}(-[:120](-[::60])(-[::-60]))(-[:-120](-[::60])(-[::-60]))}};
                \draw [curved arrow={0pt}{0pt},-CF] (-1,1.5) to[out=0,in=120] (-0.3,1.2);
                \draw [curved arrow={0pt}{2pt},-CF] (0.2,0.4) to[out=60,in=100,looseness=2.5] (30:0.6);
                \draw [curved arrow={2pt}{2pt},-CF] (30:0.9) ++(-5:0.3) to[bend left=60,looseness=1.8] ++(-5:0.55);
            \end{scope}
            \draw [-CF] (-0.5,0.4) -- ++(1,0);
            \begin{scope}[xshift=1cm]
                \filldraw [thick,draw=orx,fill=ory] (0,0)      to[out=110,in=70,looseness=300]   ++(0.01,0)
                    node[above=8mm]{$p$}
                ;
                \filldraw [thick,draw=orx,fill=ory] (0,0)      to[out=-110,in=-70,looseness=300] ++(0.01,0);
                \filldraw [thick,draw=orx,fill=ory] (30:0.8)   to[out=110,in=70,looseness=300]   ++(0.01,0)
                    node[above=8mm]{$p$}
                ;
                \filldraw [thick,draw=orx,fill=ory] (30:0.8)   to[out=-110,in=-70,looseness=300] ++(0.01,0);
                \filldraw [thick,draw=orx,fill=ory] (1.7,0.52) to[out=110,in=70,looseness=250]   ++(0.01,0)
                    node[above=6.5mm]{$p$}
                ;
                \filldraw [thick,draw=orx,fill=ory] (1.7,0.15) to[out=-110,in=-70,looseness=250] ++(0.01,0);
    
                \draw
                    (0,0)
                        -- (30:0.8)
                        -- ++(-5:0.8) node[right]{\ce{O-}}
                    (30:0.8) ++(170:0.1) -- ++(-150:0.62)
                ;
            \end{scope}
        \end{tikzpicture}
        \caption{Orbital effects for LDA deprotonation.}
        \label{fig:orbitalDeprotonation}
    \end{figure}
    \begin{itemize}
        \item In the fully formed enolate, conjugation of the oxygen anion into the $\pi$-system is stabilizing because of resonance.
        \item However, for the reaction to proceed, there must be resonance stabilization from the moment the anion begins forming.
        \item Thus, we need the $sp^3$ and the two $p$-orbitals to be aligned, as above. Notice how the \ce{C-H} bond is parallel to the $p$-orbitals of the \ce{C=O} $\pi$-system.
        \item As we deprotonate, we continuously transform the $sp^3$ orbital into a third $p$ orbital that will be in conjugation with the other two preexisting ones.
    \end{itemize}
    \item Consequences of orbital effects.
    \begin{figure}[h!]
        \centering
        \footnotesize
        \begin{subfigure}[b]{0.3\linewidth}
            \centering
            \chemfig{*6(-(-(-[:-30])(-[6])(-[:-150]))---(=O)--)}
            \caption{Axial vs. equatorial acidity.}
            \label{fig:LDAcyclohexanea}
        \end{subfigure}
        \begin{subfigure}[b]{0.3\linewidth}
            \centering
            \chemfig{*6(-(<\ce{Bu^{$t$}})--(<:Me)-(=O)-(<:Me)-)}
            \caption{Blocked $\alpha$-hydrogens.}
            \label{fig:LDAcyclohexaneb}
        \end{subfigure}
        \begin{subfigure}[b]{0.3\linewidth}
            \centering
            \chemfig{[:-30]*6(?---(-[:165,1.05]?)-(=O)-(-[:100])(-[:140])-)}
            \caption{Locked $\alpha$-hydrogen.}
            \label{fig:LDAcyclohexanec}
        \end{subfigure}
        \caption{Molecules with deprotonation reactivity affected by orbital effects.}
        \label{fig:LDAcyclohexane}
    \end{figure}
    \begin{itemize}
        \item Cyclohexane conformations affect the acidity of equatorial and axial $\alpha$-hydrogens.
        \item Consider the molecule in Figure \ref{fig:LDAcyclohexanea}.
        \begin{itemize}
            \item Recall that \emph{tert}-butyl groups are always equatorial.
            \item It follows that the carbonyl is equatorial, too, and therefore that its $\pi$-system is axial.
            \item Thus, the axial $\alpha$-protons are more acidic because of their alignment with the \ce{C=O} $\pi$-system.
            \item Consequently, LDA selectively deprotonates these.
            \item We can confirm this via selective deuteration of some cyclohexane hydrogens.
        \end{itemize}
        \item Now consider the molecule in Figure \ref{fig:LDAcyclohexaneb}.
        \begin{itemize}
            \item Once again, conformations force the \ce{Bu^{$t$}} group to be equatorial.
            \item Thus, this compound cannot be deprotonated by LDA because it has no acidic protons.
        \end{itemize}
        \item Lastly, consider the molecule in Figure \ref{fig:LDAcyclohexanec}.
        \begin{itemize}
            \item A bicyclic hydrocarbon can be locked in the unreactive conformation.
        \end{itemize}
        \item Drawing the relevant chair conformations here is an important skill.
    \end{itemize}
    \item Selectivity.
    \item Some compounds will not be selectively deprotonated.
    \begin{itemize}
        \item For example, treating 1-phenylheptan-4-one with LDA will yield products that have been deprotonated at every $\alpha$-hydrogen in equal amounts.
    \end{itemize}
    \item LDA prefers to deprotonate at less substituted positions due to its sterics.
    \item Comparing LDA- and hydroxide-based deprotonations.
    \begin{itemize}
        \item LDA is a lot more basic than hydroxide.
        \item Thus, hydroxide deprotonations are reversible while LDA deprotonations are irreversible.
        \item It follows that hydride deprotonations are under thermodynamic control (stability is important) while LDA deprotonations are under kinetic control (rate is important).
    \end{itemize}
    \item Rate is controlled by the transition state energy.
    \begin{itemize}
        \item Levin draws a 1D energy diagram for an exothermic reaction with a large $\Delta G^\ddagger$, noting that this large $\Delta G^\ddagger$ will make the reaction slower.
    \end{itemize}
    \item Selectivity in terms of kinetic and thermodynamic control.
    \begin{figure}[h!]
        \centering
        \begin{tikzpicture}[
            every node/.style=black
        ]
            \footnotesize
            \draw [very thin,dashed]
                (-5,0) -- ++(6,0)
                (5,-1) -- ++(-6,0)
                (-2.5,2) -- ++(5,0)
            ;
            \draw [very thin,<->,shorten <=1pt,shorten >=1pt] (0,-1) -- node[right]{Thermodynamics} ++(0,1);
            \draw [very thin,<->,shorten <=1pt,shorten >=1pt] (-2,2) -- node[right]{$\Delta G^\ddagger$} ++(0,1);
            \draw [very thin,<->,shorten <=1pt,shorten >=1pt] (2,2) -- node[right]{$\Delta G^\ddagger$} ++(0,1.5);
    
            \draw [grx,thick] (-6,0)
                -- node[pos=0.6,above,text width=1.1cm]{Left product} node[pos=0.6,below=1.5cm,align=center]{Kinetic\\product\\(less stable)} (-5,0)
                to[out=0,in=180] (-2,3) node[above=5mm,align=center]{lower energy TS\\faster}
                to[out=0,in=180] (0,2) node[above=1mm]{SM}
                to[out=0,in=180] (2,3.5) node[above,align=center]{higher energy TS\\slower}
                to[out=0,in=180] (5,-1)
                -- node[above,text width=1.1cm]{Right product} node[pos=0.6,below=0.5cm,align=center]{Thermodynamic\\product\\(more stable)} (6,-1)
            ;
        \end{tikzpicture}
        \caption{Thermodynamic vs. kinetic control.}
        \label{fig:thermodynamicKineticPotential}
    \end{figure}
    \begin{itemize}
        \item A reaction that is reversible will form the thermodynamic product.
        \item A reaction that is irreversible will form the kinetic product.
        \item Note that there are paradigms in which one product is both the kinetic and thermodynamic one.
        \item To determine the kinetic product, we compare transition states.
        \item To determine the thermodynamic product, we compare the products, themselves.
    \end{itemize}
    \item Application.
    \begin{figure}[H]
        \centering
        \footnotesize
        \begin{subfigure}[b]{\linewidth}
            \centering
            \schemestart
                \chemname{\chemfig{-[:30]-[:-30](-[6])-[:30](-[2]\charge{45:1pt=$\ominus$}{O})=^[:-30]-[:30]}}{Less stable}
                \arrow{0}
                \chemname{\chemfig{-[:30]-[:-30](-[6])=^[:30](-[2]\charge{45:1pt=$\ominus$}{O})-[:-30]-[:30]}}{More stable}
            \schemestop
            \chemnameinit{}
            \caption{Thermodynamic stability.}
            \label{fig:thermoKineticStabilitya}
        \end{subfigure}
    \end{figure}
    \begin{figure}[h!]
        \ContinuedFloat
        \footnotesize
        \begin{subfigure}[b]{\linewidth}
            \centering
            \schemestart
                \chemname{
                    \chemleft{[}
                        \chemfig{H-[:-5](-[2,,,,dash pattern=on 2pt off 2pt]H-[:130,,,,dash pattern=on 2pt off 2pt]\charge{45:1pt=$\ominus$}{N}(-[2](-[::60])(-[::-60]))(-[:-175](-[::60])(-[::-60])))(-[:-50]CH_3)-[:30,,,,rddbond](-[:-20,,,,lddbond]O)-[:70,0.7](-[:120,0.6])-[:15]-[:-30]}
                    \chemright{]^\ddagger}
                }{Favored}
                \arrow{0}
                \chemname{
                    \chemleft{[}
                        \chemfig{-[:-5](-[2,,,,dash pattern=on 2pt off 2pt]H-[:130,,,,dash pattern=on 2pt off 2pt]\charge{45:1pt=$\ominus$}{N}(-[2](-[::60])(-[::-60]))(-[:-175](-[::60]@{C})(-[::-60])))(-[:-50]-[:10])-[:30,,,,rddbond](-[:-20,,,,lddbond]O)-[:70,0.7]-[:15]CH_3}
                    \chemright{]^\ddagger}
                }{Disfavored}
            \schemestop
            \chemnameinit{}
            \chemmove{
                \draw [blx,thick,-] ([xshift=1mm,yshift=-4mm]C.center) to[bend right=70,looseness=1.5] ++(0.5,0.4);
                \draw [blx,thick,-] ([xshift=4mm,yshift=-7mm]C.center) to[bend left=70,looseness=1.5] ++(0.5,0.4);
            }
            \caption{Kinetic favorability.}
            \label{fig:thermoKineticStabilityb}
        \end{subfigure}
        \caption{Thermodynamic and kinetic stability in enolates.}
        \label{fig:thermoKineticStability}
    \end{figure}
    \begin{itemize}
        \item The tetrasubstituted enolate is the more stable product by Zaitsev's rule.
        \item The trisubstituted enolate has a more stable transition state.
    \end{itemize}
    \item "An analogy may assist in understanding kinetically and thermodynamically controlled reactions. Imagine a very inebriated gentleman stumbling randomly around a pasture. Near each other in the paster are a shallow watering hole and a deep well with a high fence around it. Our drunken friend is likely to fall in the hole several times, but because it is shallow, he can climb out of it and continue staggering around the pasture. After a very long while, however, he makes it over the fence and falls into the well; once in the well, he is there to stay. If we now imagine Avogadro's number of people staggering around a (very large) pasture, we get a reasonably good picture of kinetic and thermodynamic control. Initially, a large number of people fall into the shallow hole. If we wait long enough, however, most of the will end up in the deep well. The frequent occurrence --- falling in the shallow hole --- is reversible, but the rare occurrence --- climbing the fence and falling in the well --- is irreversible" \parencite{bib:DrunkGentleman}.
    \begin{itemize}
        \item The Avogadro's number correction is to bring an element of statistics and probability into the example.
        \item In the new edition, the drunken gentleman has been changed to "disoriented steers," maybe to be PC.
    \end{itemize}
    \item In general, we cannot guess how long it will take the thermodynamic enolate to accumulate (though it will likely be a long time), so we need an alternate method of generating them.
    \item Generating thermodynamic enolates.
    \begin{enumerate}
        \item Use \ce{OH-}, which catalyzes \emph{reversible} enolate formation.
        \item Use a sub-stoichiometric equivalent ($\approx 0.95$) of LDA.
    \end{enumerate}
    \item Sub-stoichiometric LDA addition.
    \begin{figure}[h!]
        \centering
        \footnotesize
        \schemestart
            \chemfig{-[:30]-[:-30](-[6])-[:30](-[@{sb1}2]@{O1}\charge{180=\:,90:3pt=$\ominus$}{O})=^[@{db1}:-30]-[:30]}
            \+
            \chemfig{-[:30]-[:-30](-[6])(-[@{sb2a}2]@{H2}H)-[@{sb2b}:30](=[@{db2}2]@{O2}O)-[:-30]-[:30]}
            \arrow
            \chemfig{-[:30]-[:-30](-[6])-[:30](=[2]O)-[:-30]-[:30]}
            \+
            \chemfig{-[:30]-[:-30](-[6])=^[:30](-[2]\charge{180=\:,90:3pt=$\ominus$}{O})-[:-30]-[:30]}
        \schemestop
        \chemmove{
            \draw [curved arrow={6pt}{2pt}] (O1) to[bend right=90,looseness=3] (sb1);
            \draw [curved arrow={4pt}{2pt}] (db1) to[out=60,in=180] (H2);
            \draw [curved arrow={2pt}{2pt}] (sb2a) to[bend left=70,looseness=3] (sb2b);
            \draw [curved arrow={3pt}{2pt}] (db2) to[bend right=90,looseness=3] (O2);
        }
        \caption{Sub-stoichiometric LDA addition.}
        \label{fig:LDASubStoichiometric}
    \end{figure}
    \begin{itemize}
        \item Using a sub-stoichiometric amount leaves some ketone behind to react with the kinetic enolate as in the above picture, generating the thermodynamic enolate and regenerating the ketone to react again.
        \item The wait time for this process to occur is usually a few hours at room temperature.
    \end{itemize}
    \item Note that if you want to form solely the kinetic product, you will need to do it quickly and with more than one equivalent (we can just say one equivalent for the purposes of this class; we use a little excess to account for any mismeasurement/human error in real life), and you will need to keep the mixture at $-\SI{78}{\celsius}$ (using a dry ice/acetone bath).
    \item Kinetic enolate formation.
    \begin{equation*}
        \ce{Ketone ->[LDA ($>1\text{ equiv}$)][\SI{-78}{\celsius}] Enolate}
    \end{equation*}
    \item Thermodynamic enolate formation.
    \begin{equation*}
        \ce{Ketone ->[LDA ($0.95\text{ equiv}$)][time] Enolate}
    \end{equation*}
    \item Uses for enolates.
    \begin{enumerate}
        \item Halogenation.
        \item \ce{C-C} bond formation.
        \item Selenium electrophile reactions.
    \end{enumerate}
    \item \textbf{N-bromosuccinimide}: A source of electrophilic bromine. \emph{Also known as} \textbf{NBS}. \emph{Structure}
    \begin{figure}[h!]
        \centering
        \footnotesize
        \chemfig{*5(-(=O)-N(-Br)-(=O)--)}
        \caption{N-bromosuccinimide.}
        \label{fig:NBS}
    \end{figure}
    \item Halogenation.
    \item General form.
    \begin{center}
        \footnotesize
        \setchemfig{atom sep=1.4em}
        \schemestart
            \chemfig{-[:30]-[:-30](-[6])-[:30](=[2]O)-[:-30]-[:30]}
            \arrow{->[1. LDA][2. NBS]}[,1.2]
            \chemfig{-[:30]-[:-30](-[6])-[:30](=[2]O)-[:-30](-[6]Br)-[:30]}
        \schemestop
    \end{center}
    \begin{itemize}
        \item We get bromination of the kinetic enolate assuming we perform keep this reaction cold and perform it fast.
    \end{itemize}
    \item Mechanism.
    \begin{itemize}
        \item The enolate attacks the bromine of NBS, and the \ce{N-Br} electrons retreat onto the nitrogen.
    \end{itemize}
    \item Reacting the thermodynamic enolate.
    \begin{itemize}
        \item Although we might think to use \ce{OH-} / \ce{Br2}, this would from an $\alpha,\beta$ unsaturated compound as per Figure \ref{fig:haloformBeta}.
        \item Thus, we turn to acidic conditions.
    \end{itemize}
    \item Acidic conditions form thermodynamic enols.
    \begin{center}
        \footnotesize
        \setchemfig{atom sep=1.4em}
        \schemestart
            \chemfig{-[:30]-[:-30](-[6])-[:30](=[2]O)-[:-30]-[:30]}
            \arrow{->[\ce{H+}][\ce{Br2}]}
            \chemfig{-[:30]-[:-30](-[:-110])(-[:-70]Br)-[:30](=[2]O)-[:-30]-[:30]}
        \schemestop
    \end{center}
    \begin{itemize}
        \item These enols are formed reversibly (see Figure \ref{fig:mechanismEnolHalogenationAcid}), so they have an opportunity to equilibriate and favor the thermodynamic product.
        \item Once the thermodynamic enol has been built up, it reacts selectively with \ce{Br2}.
    \end{itemize}
    \item \ce{C-C} bond formation with enolates.
    \item General form.
    \begin{center}
        \footnotesize
        \setchemfig{atom sep=1.4em}
        \schemestart
            \chemfig{*6(=-=(-(=[2]O)-[:-30]-)-=-)}
            \arrow{->[1. LDA][2. \ce{RX}\rule{5pt}{0pt}]}[,1.2]
            \chemfig{*6(=-=(-(=[2]O)-[:-30](-[6]R)-)-=-)}
        \schemestop
    \end{center}
    \begin{itemize}
        \item Note that phenyl alkyl ketones have no selectivity problems because they only have $\alpha$-hydrogens on one side.
        \item \ce{X} is bromine or iodine.
        \item Works great if \ce{R} is a methyl group.
        \item Works ok if \ce{R} is a primary alkyl.
        \item E2 of the alkyl halide starts to dominate if \ce{R} is secondary or tertiary.
    \end{itemize}
    \item Examples.
    \begin{figure}[h!]
        \centering
        \footnotesize
        \begin{subfigure}[b]{\linewidth}
            \centering
            \schemestart
                \chemfig{*6(=-=(-(=[2]O)-[:-30]-)-=-)}
                \arrow{->[1. LDA][2. \ce{MeI}\rule{3pt}{0pt}]}[,1.2]
                \chemfig{*6(=-=(-(=[2]O)-[:-30](-[6])-)-=-)}
            \schemestop
            \caption{Methyl \ce{R} group.}
            \label{fig:enolateCCexamplesa}
        \end{subfigure}\\[2em]
        \begin{subfigure}[b]{\linewidth}
            \centering
            \schemestart
                \chemfig{*6(=-=(-(=[2]O)-[:-30]-)-=-)}
                \arrow{->[LDA]}
                \chemfig{*6(=-=(-(-[@{sb2}2]@{O2}\charge{180=\:,45:1pt=$\ominus$}{O})=^[@{db2}:-30]-)-=-)}
                \arrow{->[\chemfig[atom sep=1.4em]{@{H3}H-[@{sb3a}:-30]-[@{sb3b}:30](-[@{sb3c}2]@{I3}I)-[:-30]}]}[,1.4]
                \chemfig{*6(=-=(-(=[2]O)-[:-30](-[6])-)-=-)}
                \arrow{0}[,0.1]\+
                \chemfig{=_[:30]-[:-30]}
            \schemestop
            \chemmove{
                \draw [curved arrow={6pt}{2pt}] (O2) to[bend right=90,looseness=3] (sb2);
                \draw [curved arrow={4pt}{2pt}] (db2) to[out=60,in=150,looseness=1.2] (H3);
                \draw [curved arrow={2pt}{2pt}] (sb3a) to[bend left=80,looseness=3] (sb3b);
                \draw [curved arrow={2pt}{2pt}] (sb3c) to[bend right=90,looseness=3] (I3);
            }
            \caption{Isopropyl \ce{R} group.}
            \label{fig:enolateCCexamplesb}
        \end{subfigure}
        \caption{Examples of \ce{C-C} bond-forming reactions with enolates.}
        \label{fig:enolateCCexamples}
    \end{figure}
    \begin{itemize}
        \item We'll fix the issue that arises in Figure \ref{fig:enolateCCexamplesb} next time.
    \end{itemize}
    \item A new electrophile (selenium).
    \item General form.
    \begin{center}
        \footnotesize
        \setchemfig{atom sep=1.4em}
        \schemestart
            \chemfig{*6(=-=(-(=[2]O)-[:-30]-)-=-)}
            \arrow{->[1. LDA\rule{1.1em}{0pt}][2. \ce{PhSeCl}]}[,1.4]
            \chemfig{*6(=-=(-(=[2]O)-[:-30](-[6]SePh)-)-=-)}
        \schemestop
    \end{center}
    \begin{itemize}
        \item We can use either the phenyl selenyl chloride or phenyl selenyl bromide.
    \end{itemize}
    \item Mechanism.
    \begin{itemize}
        \item The enolate attacks the selenium atom and kicks out chlorine in one concerted step.
    \end{itemize}
    \item The purpose of adding selenium to compounds.
    \begin{itemize}
        \item We put selenium in just to take it back out again.
        \item We typically don't want to build molecules with it because it's quite toxic and not commonly used in biochemistry.
    \end{itemize}
    \item Eliminating phenyl selenide.
    \item General form.
    \begin{center}
        \footnotesize
        \setchemfig{atom sep=1.4em}
        \schemestart
            \chemfig{*6(=-=(-(=[2]O)-[:-30](-[6])-SePh)-=-)}
            \arrow{->[reagents]}[,1.2]
            \chemfig{*6(=-=(-(=[2]O)-[:-30]=_[6])-=-)}
        \schemestop
    \end{center}
    \begin{itemize}
        \item The reagents are either mCPBA or \ce{H2O2}.
        \item We use this method over hydroxide and bromine because it is compatible with LDA, which means that we can get selectivity for elimination now in addition to bromination.
    \end{itemize}
    \item Mechanism.
    \begin{figure}[h!]
        \centering
        \footnotesize
        \schemestart
            \chemfig{*6(=-=(-(=[2]O)-[:-30](-[6])-SePh)-=-)}
            \arrow{->[mCPBA]}[,1.2]
            % \chemfig{*6(=-=(-(=[2]O)-[:-30](-[6]-[:-30]H)-Se(-[2]Ph)(=[:-30]O))-=-)}
            \chemfig{*6(=-=(-(=[2]O)-[:-30]*5([:-30,1.2]-[@{sb2a}]-[@{sb2b}]@{H2}H-[,,,,white]O=[@{db2}]@{Se2}Se(-Ph)-[@{sb2c}]))-=-)}
            \arrow
            \chemfig{*6(=-=(-(=[2]O)-[:-30]=_[6])-=-)}
            \arrow{0}[,0.1]\+{,,0.7em}
            \chemfig{Ph-[:30]Se-[:-30]OH}
        \schemestop
        \chemmove{
            \draw [curved arrow={3pt}{2pt}] (db2) to[out=-138,in=110] (H2);
            \draw [curved arrow={2pt}{2pt}] (sb2b) to[bend right=60,looseness=1.8] (sb2a);
            \draw [curved arrow={2pt}{2pt}] (sb2c) to[bend right=60,looseness=1.8] (Se2);
        }
        \caption{Phenyl selenide elimination mechanism.}
        \label{fig:mechanismPhenylSelenideE2}
    \end{figure}
    \begin{itemize}
        \item The first intermediate is a \textbf{selenoxide}.
    \end{itemize}
    \item Selectivity.
    \begin{figure}[h!]
        \centering
        \footnotesize
        \schemestart
            \chemfig{-[:-30]-[:30](=[2]O)-[:-30](-[6])-[:30]}
            \arrow(a--b){->[\ce{OH-}][\ce{Br2}]}
            \chemfig{-[:-30]-[:30](=[2]O)-[:-30](=[6])-[:30]}
            \arrow(@a--c){->[*{0.-90}
                \begin{tabular}{l}
                    1. LDA\\
                    2. \ce{PhSeBr}\\
                    3. mCPBA\\
                \end{tabular}
            ]}[180,1.5]
            \chemfig{=^[:-30]-[:30](=[2]O)-[:-30](-[6])-[:30]}
        \schemestop
        \caption{Selectivity in the formation of $\alpha,\beta$ unsaturated compounds.}
        \label{fig:alphaBetaE2Selectivity}
    \end{figure}
    \begin{itemize}
        \item We use the thermodynamic enolate (accessible via reversible hydroxide) for the right side and the kinetic enolate (accessible via irreversible LDA) for the left side.
    \end{itemize}
    \item Applications to carboxylic acid derivatives.
    \begin{figure}[H]
        \centering
        \footnotesize
        \begin{subfigure}[b]{\linewidth}
            \centering
            \schemestart
                \chemfig{-[:-30]O-[:30](=[2]O)-[:-30]-[:30]H}
                \arrow{->[LDA]}
                \chemfig{-[:-30]O-[:30](-[2]\charge{45:1pt=$\ominus$}{O})=[:-30]}
                \arrow{->[\ce{E+}]}
                \chemfig{-[:-30]O-[:30](=[2]O)-[:-30]-[:30]E}
            \schemestop
            \caption{Esters.}
            \label{fig:carboxylicEnolatea}
        \end{subfigure}\\[2em]
        \begin{subfigure}[b]{\linewidth}
            \centering
            \schemestart
                \chemfig{-[:-30]N(-[6])-[:30](=[2]O)-[:-30]-[:30]H}
                \arrow{->[LDA]}
                \chemfig{-[:-30]N(-[6])-[:30](-[2]\charge{45:1pt=$\ominus$}{O})=[:-30]}
                \arrow{->[\ce{E+}]}
                \chemfig{-[:-30]N(-[6])-[:30](=[2]O)-[:-30]-[:30]E}
            \schemestop
            \caption{Amides.}
            \label{fig:carboxylicEnolateb}
        \end{subfigure}
    \end{figure}
    \begin{figure}[h!]
        \ContinuedFloat
        \footnotesize
        \begin{subfigure}[b]{\linewidth}
            \centering
            \schemestart
                \chemfig{N~[:-30]C-[:-30]-[:30]H}
                \arrow{->[LDA]}
                \chemname{\chemfig{\charge{90:3pt=$\ominus$}{N}=[:-30]C=[:-30]}}{Ketene imidate}
                \arrow{->[\ce{E+}]}
                \chemfig{N~[:-30]C-[:-30]-[:30]E}
            \schemestop
            \chemnameinit{}
            \caption{Nitriles.}
            \label{fig:carboxylicEnolatec}
        \end{subfigure}
        \caption{Carboxylic acid derivatives as enolates.}
        \label{fig:carboxylicEnolate}
    \end{figure}
    \item Comparing the nucleophilicity of ketone enolates, ester enolates, and amide enolates.
    \begin{figure}[H]
        \centering
        \footnotesize
        \begin{subfigure}[b]{\linewidth}
            \centering
            \schemestart
                \chemfig{-[:30](-[@{sb1}2]@{O1}\charge{180=\:,45:1pt=$\ominus$}{O})=[@{db1}:-30]@{C1}}
                \arrow{<->}
                \chemfig{-[:30](=[2]O)-[:-30]\charge{45:1pt=$\ominus$}{}}
            \schemestop
            \chemmove{
                \draw [curved arrow={6pt}{2pt},blx] (O1) to[bend right=90,looseness=3] (sb1);
                \draw [curved arrow={3pt}{3pt},blx] (db1) to[bend left=90,looseness=4] (C1);
            }
            \caption{Ketone enolate resonance.}
            \label{fig:carboxylicEnolateResonancea}
        \end{subfigure}\\[2em]
        \begin{subfigure}[b]{\linewidth}
            \centering
            \schemestart
                \chemfig{-[:-30]@{O1}\charge{90:3pt=$\oplus$}{O}=[@{db1}:30](-[2]\charge{45:1pt=$\ominus$}{O})-[@{sb1}:-30]@{C1}\charge{45:1pt=$\ominus$}{}}
                \arrow{<->}
                \chemfig{-[:-30]O-[:30](-[@{sb2}2]@{O2}\charge{180=\:,45:1pt=$\ominus$}{O})=[@{db2}:-30]@{C2}}
                \arrow{<->}
                \chemfig{-[:-30]O-[:30](=[2]O)-[:-30]\charge{45:1pt=$\ominus$}{}}
            \schemestop
            \chemmove{
                \draw [curved arrow={10pt}{2pt},blx] (C1) to[bend right=90,looseness=5] (sb1);
                \draw [curved arrow={3pt}{2pt},blx] (db1) to[bend left=90,looseness=3] (O1);
                \draw [curved arrow={6pt}{2pt},blx] (O2) to[bend right=90,looseness=3] (sb2);
                \draw [curved arrow={3pt}{3pt},blx] (db2) to[bend left=90,looseness=4] (C2);
            }
            \caption{Ester enolate resonance.}
            \label{fig:carboxylicEnolateResonanceb}
        \end{subfigure}
        \caption{An extra resonance form for carboxylic acid derivative enolates.}
        \label{fig:carboxylicEnolateResonance}
    \end{figure}
    \begin{itemize}
        \item Nucleophilicity depends on how electron-rich the $\pi$ system is.
        \item Oxygen and nitrogen both donate their lone pairs to the $\pi$ system.
        \item The additional resonance form makes the carboxylic acid derivative enolates more nucleophilic.
        \item Nitrogen is the most nucleophilic (because of its lower electronegativity relative to oxygen), then oxygen, then carbon (of these three).
    \end{itemize}
    \item We will not be asked to compare the nucleophilicity of ketene imidates to ketone, ester, or amide enolates.
    \item Selectivity is nice for all carboxylic acid derivatives; there's at most one set of $\alpha$-hydrogens for all of them.
    \item Compounds whose enolates are less useful.
    \begin{itemize}
        \item Carboxylic acids: These will become carboxylates upon the first deprotonation. The second deprotonation takes a much stronger base, forms a dianion, and doesn't work too well.
        \item Amides with hydrogens: These deprotonate first as well and then run into the same dianion problem.
        \item Acid chlorides: These kick out the chloride along with deprotonation, forming a \textbf{ketene}. There are things we can do with ketenes, but we won't talk about them since they aren't as useful as enolates.
        \item Aldehydes: These will dimerize. In particular, one deprotonated aldehyde will engage in a nucleophilic attack on another.
        \begin{itemize}
            \item This will form most of the rest of the class.
        \end{itemize}
    \end{itemize}
\end{itemize}



\section[Reactions at the \texorpdfstring{$\alpha$}{TEXT}-Carbon of Carbonyl Compounds 3]{Reactions at the \texorpdfstring{$\bm{\alpha}$}{TEXT}-Carbon of Carbonyl Compounds 3}
\begin{itemize}
    \item \marginnote{4/28:}Last time.
    \begin{itemize}
        \item Enolates derived from ketones that are the major species in solution.
        \item Specifically, ones that are generated selectively.
        \item Enolates can be used to make new \ce{C-C} bonds (but in a limited number of cases).
    \end{itemize}
    \item Today.
    \begin{itemize}
        \item Expanding the utility of enolates in \ce{C-C} bond forming reactions.
        \item In particular, developing solutions to the following issues.
    \end{itemize}
    \item Problems with enolates.
    \begin{enumerate}
        \item Enolates are basic.
        \begin{itemize}
            \item This means that enolates preferentially eliminate secondary and tertiary alkyls instead of adding into them via S\textsubscript{N}2.
        \end{itemize}
        \item LDA is not regioselective for similar sites.
        \begin{itemize}
            \item Recall 1-phenylhept-4-one.
        \end{itemize}
        \item Aldehyde enolates self-attack.
        \begin{itemize}
            \item Weixin will talk at length about how to control this "problem" and actually utilize it.
        \end{itemize}
    \end{enumerate}
    \item The solutions.
    \begin{itemize}
        \item There is no unified solution; rather, we will discuss two partial solutions.
        \item Both of these solutions solve problem 1. In addition, one solves problem 2, and the other solves problem 3.
    \end{itemize}
    \item Generalizing about the problems we face.
    \begin{itemize}
        \item The overall problem is that enolates are too reactive.
        \item Solution: Use the enol --- it's less reactive than the enolate because it's neutral.
        \item However, enols are the minor species in solution relative to their ketone tautomers, and since these two molecules are constitutional isomers (i.e., nothing is gained or lost in the tautomerization), there is no way to push the equilibrium to one side or the other with Le Ch\^{a}telier's principle.
    \end{itemize}
    \item Solution 1 (to problems 1 and 3): Enamines.
    \item Levin reviews enamine formation.
    \begin{itemize}
        \item See the general form below Figure \ref{fig:enamine}.
        \item Levin notes that removing water can further drive the reaction in the forward direction.
    \end{itemize}
    \item As one might assume from the structural homology between enols and enamines, the two compounds do indeed have similar reactivity.
    \item Comparing enolate, enamine, and enol reactivity.
    \begin{itemize}
        \item Enolates are more reactive than enamines, which are more reactive than enols.
        \begin{itemize}
            \item In layman's terms, this is due to the presence/lack thereof of formal charges and the relative electronegativities of nitrogen and oxygen, respectively.
            \item More specifically, when we draw the resonance forms for all three of these compounds that put the negative charge on the $\alpha$-carbon, we note two things.
            \item First, enamines and enols both have a counterbalancing positive charge on their heteroatom. This makes their $\alpha$-carbons significantly less basic (hence less reactive) than enolates'.
            \item Second, oxygen is more electronegative than nitrogen. Thus, the positively charged oxygen withdraws electron density to an even greater extent than the positively charged nitrogen. Consequently, the enol's $\alpha$-carbon is less basic than the enamine's.
        \end{itemize}
        \item Therefore, enamines are Goldilocks nucleophiles (with reactivity between enols and enolates).
        \item Advantages of enamines over enolates.
        \begin{itemize}
            \item Less reactive.
            \item Still reactive enough.
        \end{itemize}
        \item Advantages of enamines over enols.
        \begin{itemize}
            \item Enamines can be stoichiometrically generated.
            \item We can use them as nucleophiles.
        \end{itemize}
    \end{itemize}
    \item Using enamines to alkylate carbonyls.
    \item General form.
    \begin{center}
        \footnotesize
        \setchemfig{atom sep=1.4em}
        \schemestart
            \chemfig{R-[:30](=[2]O)-[:-30]}
            \arrow{->[\begin{tabular}{l}
                1. $2^\circ$ amine, $[-\ce{H2O}]$, \ce{H+}\\
                2. \ce{{}^{$i$}PrI}\\
                3. \ce{H3O+}\\
            \end{tabular}]}[,2.7]
            \chemfig{R-[:30](=[2]O)-[:-30]-[:30](-[2])-[:-30]}
        \schemestop
    \end{center}
    \begin{itemize}
        \item This procedure permits secondary $\alpha$-alkylation of carbonyl compounds, solving problem 1.
        \item $\ce{R}=\ce{H,C}$.
        \begin{itemize}
            \item Thus, this procedure $\alpha$-alkylates both ketones and aldehydes, solving problem 3.
        \end{itemize}
        \item We can any secondary amine we like. Some examples are pyrrolidine or morpholine (the latter is used in Figure \ref{fig:enamineAlkylationMechanism}).
    \end{itemize}
    \item Mechanism.
    \begin{figure}[h!]
        \centering
        \footnotesize
        \schemestart
            \chemfig{-[:30](=[2]O)-[:-30]}
            \arrow{0}[,0.1]\+
            \chemfig{*6(-\chembelow{N}{H}---O--)}
            \arrow{->[\ce{H+}][$[-\ce{H2O}]$]}[,1.2]
            \chemfig{*6([:60]\vphantom{O}-[,,,,white]-[,,,,white]-[,,,,white](-[:30](-[2]N*6(---O---))=^[:-30]))}
            \arrow{->[\chemfig[atom sep=1.4em]{-[:30](-[2]I)-[:-30]}]}
            \chemfig{*6([:60]\vphantom{O}-[,,,,white]-[,,,,white]-[,,,,white](-[:30](=[2]\charge{90:3pt=$\oplus$}{N}*6(---O---)-[0,0.6,,,white]\charge{45:1pt=$\ominus$}{I})-[:-30]-[:30](-[2])-[:-30]))}
            \arrow{->[\ce{H3O+}]}
            \chemfig{-[:30](=[2]O)-[:-30]-[:30](-[2])-[:-30]}
            \arrow{0}[,0.1]\+
            \chemfig{*6(-\chembelow{N}{H}---O--)}
        \schemestop
        \vspace{-3em}
        \caption{$\alpha$-alkylating carbonyl compounds using enamines mechanism.}
        \label{fig:enamineAlkylationMechanism}
    \end{figure}
    \begin{itemize}
        \item The $\alpha$-alkylated iminium generated as the second intermediate above is stable until workup.
        \begin{itemize}
            \item This is fairly remarkable since it is difficult (requires removing water) to generate iminiums from ketones.
            \item However, since we have already isolated the enamine before the second reaction above, there is no chance of it reacting backwards. This is what leads to the stability of the iminium intermediates.
        \end{itemize}
        \item That being said, adding the water back in (in the form of an acid workup) readily hydrolyzes the iminium back down to a ketone.
        \begin{itemize}
            \item Recall, wrt. the third step, that imines are prone to hydrolysis (see Lecture 2).
        \end{itemize}
    \end{itemize}
    \item Aside (will not be tested).
    \begin{itemize}
        \item Since the amine reacts in Figure \ref{fig:enamineAlkylationMechanism} but is regenerated at the end, it is technically a catalyst.
        \item More generally, catalytic alkylations and electrofunctionalizations can be accomplished via the above mechanism. Thus, instead of introducing a stoichiometric amount of the amine, we can cycle through a small, catalytic quantity of amine.
        \item Moreover, if we use a chiral amine, we can influence the stereocenter in the product.
        \item This is what the 2021 Nobel Prize was awarded for: Asymmetric organocatalysis.
        \item Dave Macmillan of Princeton (one of the Nobel laureates) is a great chemist but also a master salesman, so what he realized and sold was that you can use small organic molecules as catalysts (or \textbf{organocatalysts}).
        \begin{itemize}
            \item This was revolutionary because it was thought in the early 2000s that catalysts had to either be transition metals or enzymes.
        \end{itemize}
        \item A common organocatalyst is proline.
        % \begin{figure}[h!]
        %     \centering
        %     \begin{tikzpicture}
        %         \footnotesize
        %         \node (C0) at (90:6) {\chemfig{*5([:-18]-\chembelow{N}{H}-\charge{45=*}{}(-[0](=[::60]O)-[::-60]OH)---)}};
        %         \node (C1) at (90:3) {\chemfig{*5([:-18]-\charge{90:3pt=$\oplus$}{N}(-[:-70]{\color{grx}H})(-[:-110]H)-\charge{45=*}{}(-[0](=[::60]O)-[::-60]\charge{45:1pt=$\ominus$}{O})---)}};
        %         \node (C2) at (-30:3) {\chemfig{*5([:-18]-N(-(-[::-60]H)=^[::60]-[::60])-\charge{45=*}{}(-[0](=[::60]O)-[::-60]OH)---)}};
        %         \node (C3) at (-150:3) {\chemfig{*5([:-18]-\charge{90:3pt=$\oplus$}{N}(=(-[::-60]H)-[::60]\charge{90=*}{}(-[::-60]E)-[::60])-\charge{45=*}{}(-[0](=[::60]O)-[::-60]OH)---)}};
        
        %         \draw [blo,semithick,arrows={-CF[harpoon]}] (C1.92) -- (C0.-93);
        %         \draw [blo,semithick,arrows={-CF[harpoon]}] (C0.-87) -- (C1.88);
        %         \draw [blo,semithick,-CF] (C1) to[bend left=40] (C2);
        %         \draw [blo,semithick,-CF] (C2) to[bend left=35] ([xshift=1cm,yshift=-1.2cm]C3.center);
        %         \draw [blo,semithick,-CF] (C3) to[bend left=35] (C1);
        
        %         \node (R1) at (50:4.5) {\chemfig{H-[:30](=[2]{\color{grx}O})-[:-30](-[6]{\color{grx}H})-[:30]}};
        %         \node (P1) at (10:4.5) {\color{grx}\ce{H2O}};
        %         \node (R2) at (-70:4.5) {\chemfig{\charge{45:1pt=$\oplus$}{E}}};
        %         \node (R3) at (170:4.2) {\ce{H2O}};
        %         \node (P3) at (135:4.5) {\chemfig{H-[:30](=[2]O)-[:-30]\charge{90=*}{}(-[6]E)-[:30]}};
        
        %         \draw [blo,semithick,-CF] ([xshift=-3mm,yshift=-5mm]R1.center) to[bend right=60,looseness=1.24] (P1);
        %         \draw [blo,semithick] (R2) to[out=120,in=0] (-90:3.253);
        %         \draw [blo,semithick,-CF] (R3) to[bend right=55,looseness=1.5] (P3);
        %     \end{tikzpicture}
        %     \caption{Proline organocatalysis.}
        %     \label{fig:organocatalysis}
        % \end{figure}
        \begin{figure}[h!]
            \centering
            \begin{tikzpicture}
                \footnotesize
                \node (C0) at (90:6) {\chemfig{*5([:-18]-\chembelow{N}{H}-\charge{45=*}{}(-[0](=[::60]O)-[::-60]OH)---)}};
                \node (C1) at (90:3) {\chemfig{*5([:-18]-\charge{90:3pt=$\oplus$}{N}(-[:-70]{\color{grx}H})(-[:-110]H)-\charge{45=*}{}(-[0](=[::60]O)-[::-60]\charge{45:1pt=$\ominus$}{O})---)}};
                \node (C2) at (-30:3) {\chemfig{*5([:-18]-N(-(-[::-60]H)=^[::60]-[::60])-\charge{45=*}{}(-[0](=[::60]O)-[::-60]OH)---)}};
                \node (C3) at (-150:3) {\chemfig{*5([:-18]-\charge{90:3pt=$\oplus$}{N}(=(-[::-60]H)-[::60]\charge{90=*}{}(-[::-60]E)-[::60])-\charge{45=*}{}(-[0](=[::60]O)-[::-60]OH)---)}};
        
                \draw [blo,semithick,arrows={-CF[harpoon]}] (C1.92) -- (C0.-93);
                \draw [blo,semithick,arrows={-CF[harpoon]}] (C0.-87) -- (C1.88);
                \draw [blo,semithick,-CF] (60:3) arc[start angle=60,end angle=-5,radius=3cm];
                \draw [blo,semithick,-CF] (-70:3) arc[start angle=-70,end angle=-130,radius=3cm];
                \draw [blo,semithick,-CF] (180:3) arc[start angle=180,end angle=115,radius=3cm];
        
                \node (R1) at (50:4.5) {\chemfig{H-[:30](=[2]{\color{grx}O})-[:-30](-[6]{\color{grx}H})-[:30]}};
                \node (P1) at (10:4.5) {\color{grx}\ce{H2O}};
                \node (R2) at (-70:4.5) {\chemfig{\charge{45:1pt=$\oplus$}{E}}};
                \node (R3) at (170:4.2) {\ce{H2O}};
                \node (P3) at (135:4.8) {\chemfig{H-[:30](=[2]O)-[:-30]\charge{90=*}{}(-[6]E)-[:30]}};
        
                \draw [blo,semithick] (R2) to[out=110,in=0] (-90:3);
                \draw [blo,semithick,-CF] ([xshift=-3mm,yshift=-5mm]R1.center) to[bend right=55,looseness=1.22] (P1);
                \draw [blo,semithick,-CF] (R3) to[bend right=55,looseness=1.2] ([xshift=7mm,yshift=-5mm]P3.center);
            \end{tikzpicture}
            \caption{Proline organocatalysis.}
            \label{fig:organocatalysis}
        \end{figure}
        \begin{itemize}
            \item The substrate is any simple aldehyde.
            \item Because proline (a simple, cheap amino acid) is chiral, its stereocenter influences the final one by favoring one face of the substrate for electrophilic attack over the other.
            \item However, you need to use a lot of it.
        \end{itemize}
        \item \textbf{Macmillan's catalyst} is drawn as well.
        \begin{figure}[h!]
            \centering
            \footnotesize
            \chemfig{*5((<(-[::60])(-[::-60])-)-\chembelow{N}{H}-(<-[::60]Ph)-(=O)-HN-[,,2])}
            \caption{Macmillan's catalyst.}
            \label{fig:macmillanCatalyst}
        \end{figure}
        \begin{itemize}
            \item Macmillan's catalyst allows much lower loadings while retaining high levels of stereocontrol.
        \end{itemize}
    \end{itemize}
    \item Enamines don't solve problem 2 (regioselectivity for similar sites), however.
    \item Solution 2 (to problems 1 and 2): $\beta$-dicarbonyl compounds.
    \item $\beta$-dicarbonyl compounds.
    \begin{figure}[h!]
        \centering
        \footnotesize
        \begin{subfigure}[b]{0.25\linewidth}
            \centering
            \chemfig{-[:30](=[2]O)-[:-30]-[:30](=[2]O)-[:-30]}
            \caption{$\beta$-diketone.}
            \label{fig:betaDicarbonylsa}
        \end{subfigure}
        \begin{subfigure}[b]{0.25\linewidth}
            \centering
            \chemfig{RO-[:30](=[2]O)-[:-30]-[:30](=[2]O)-[:-30]}
            \caption{$\beta$-ketoester.}
            \label{fig:betaDicarbonylsb}
        \end{subfigure}
        \begin{subfigure}[b]{0.25\linewidth}
            \centering
            \chemfig{RO-[:30](=[2]O)-[:-30]-[:30](=[2]O)-[:-30]OR'}
            \caption{Malonate.}
            \label{fig:betaDicarbonylsc}
        \end{subfigure}
        \caption{$\beta$-dicarbonyl compounds.}
        \label{fig:betaDicarbonyls}
    \end{figure}
    \begin{itemize}
        \item $\beta$-dicarbonyls are referred to as such because relative to either carbonyl functional group, the other carbonyl is on the original carbonyl's $\beta$-carbon (two away along the chain from the carbon involved in the functional group).
        \item Malonates could be called $\beta$-diesters, but nobody refers to them as such.
    \end{itemize}
    \item $\beta$-dicarbonyls are useful because $9\leq\pKa\leq 11$ for the hydrogens on the central $\alpha$-carbon.
    \begin{itemize}
        \item As specific examples, pentane-2,4-dione (Figure \ref{fig:betaDicarbonylsa}) has $\pKa=9$, dimethyl malonate has $\pKa=11$, and methyl acetoacetate is somewhere in the middle.
        \item The implication is that we can deprotonate $\beta$-dicarbonyls far easier than regular ketones.
    \end{itemize}
    \item In particular, while we need LDA for acetone, we can use methoxide for $\beta$-dicarbonyls.
    \begin{figure}[h!]
        \centering
        \footnotesize
        \schemestart
            \chemfig{MeO-[:30](=[2]O)-[:-30](-[@{sb1a}6]@{H1}H)-[@{sb1b}:30](=[@{db1}2]@{O1}O)-[:-30]}
            \arrow{->[\chemfig{@{O2}\charge{180=\:,90:3pt=$\ominus$}{O}Me} / \ce{MeOH}]}[,1.9]
            \chemleft{[}
                \subscheme{
                    \chemfig{MeO-[:30](=[2]O)-[:-30]@{C3}=^[@{db3}:30](-[@{sb3}2]@{O3}\charge{45:1pt=$\ominus$}{O})-[:-30]}
                    \arrow{<->}
                    \chemfig{MeO-[:30](=[@{db4}2]@{O4}O)-[@{sb4}:-30]@{C4}\charge{-90:3pt=$\ominus$}{}-[:30](=[2]O)-[:-30]}
                    \arrow{<->}
                    \chemfig{MeO-[:30](-[2]\charge{45:1pt=$\ominus$}{O})=^[:-30]-[:30](=[2]O)-[:-30]}
                }
            \chemright{]}
        \schemestop
        \chemmove{
            \draw [curved arrow={6pt}{2pt}] (O2) to[out=180,in=0,out looseness=0.8,in looseness=2] (H1);
            \draw [curved arrow={2pt}{2pt}] (sb1a) to[bend right=70,looseness=2] (sb1b);
            \draw [curved arrow={3pt}{2pt}] (db1) to[bend left=90,looseness=3] (O1);
            \draw [curved arrow={10pt}{2pt},blx] (O3) to[out=45,in=0,looseness=4] (sb3);
            \draw [curved arrow={4pt}{4pt},blx] (db3) to[bend right=80,looseness=4] (C3);
            \draw [curved arrow={11pt}{2pt},blx] (C4) to[out=-90,in=-120,looseness=6] (sb4);
            \draw [curved arrow={3pt}{2pt},blx] (db4) to[bend left=90,looseness=3] (O4);
        }
        \caption{Deprotonating $\beta$-dicarbonyls.}
        \label{fig:betaDicarbonylDeprotonation}
    \end{figure}
    \begin{itemize}
        \item It makes sense that $\beta$-dicarbonyls are more acidic than regular carbonyl compounds because their conjugate bases have three resonance forms, as above, compared to the two of regular carbonyls (as in Figure \ref{fig:enolate}).
        \begin{itemize}
            \item Recall that carbonyl compounds are in turn more acidic than alkanes because alkanes only have one resonance form.
        \end{itemize}
        \item Since $\pKa=10$ for methyl acetoacetate and $\pKa=15$ for \ce{MeOH}, only 1 in every $10^5$ $\beta$-dicarbonyls is not deprotonated, so the above equilibrium lies fairly far to the right.
        \item Note that we do need to "match" the alkyl ester group(s), organic base, and solvent so that background competitive nucleophilic acyl substitution does not become observable.
        \begin{itemize}
            \item For example, in the above reaction, we use \emph{methyl} acetoacetate, \emph{meth}oxide, and \emph{meth}anol.
        \end{itemize}
    \end{itemize}
    \item The reason for performing this reaction in alcholic solvent is purely practical.
    \begin{itemize}
        \item Sodium alkoxides can be bought, but they're air-sensitive and prone to decomposition.
        \item Thus, we prefer to generate them fresh.
        \item To do so, we treat some quantity of alcohol with sodium hydride or sodium metal. Either way, we form solvated sodium alkoxide in methanol and releasing \ce{H2} gas.
        \item Performing this reaction in an excess of methanol allows us to easily proceed with the reaction in liquid media, just with the necessary condition that the solvent is the alcohol.
    \end{itemize}
    \item Other ways of deprotonating $\beta$-dicarbonyls.
    \begin{itemize}
        \item Since protonated triethyl amine has a comparable $\pKa$ to methyl acetoacetate, mixing \ce{NEt3} with methyl acetoacetate generates the deprotonated form reversibly.
        \item In principle, we could also use LDA. It's overkill (it would lead to $10^{26}$-fold deprotonation), but there's nothing chemically wrong with it.
    \end{itemize}
    \item We now get into the reactions of $\beta$-dicarbonyls.
    \item Monoalkylation.
    \item General form.
    \begin{center}
        \footnotesize
        \setchemfig{atom sep=1.4em}
        \schemestart
            \chemfig{-[:30](=[2]O)-[:-30](-[6]H)-[:30](=[2]O)-[:-30]OMe}
            \arrow{->[1. \ce{NaOMe} / \ce{MeOH} (1 equiv.)][2. \ce{MeI}\rule{3.07cm}{0pt}]}[,3.1]
            \chemfig{-[:30](=[2]O)-[:-30](-[6]Me)-[:30](=[2]O)-[:-30]OMe}
        \schemestop
    \end{center}
    \item Mechanism.
    \begin{itemize}
        \item We generate the enolate (as in Figure \ref{fig:betaDicarbonylDeprotonation}). It subsequently attacks methyl iodide from the backside via an S\textsubscript{N}2 mechanism.
    \end{itemize}
    \item Dialkylation.
    \item General form.
    \begin{center}
        \footnotesize
        \setchemfig{atom sep=1.4em}
        \schemestart
            \chemfig{-[:30](=[2]O)-[:-30](-[6]H)-[:30](=[2]O)-[:-30]OMe}
            \arrow{->[\ce{NaOMe} / \ce{MeOH} (excess)][\ce{MeI} (excess)]}[,2.7]
            \chemfig{-[:30](=[2]O)-[:-30](-[6]Me)-[:30](=[2]O)-[:-30]OMe}
        \schemestop
    \end{center}
    \item Mechanism.
    \begin{itemize}
        \item The first two steps are the same as the monoalkylation mechanism.
        \item The third and fourth steps are deprotonation of the monoalkylated product followed by S\textsubscript{N}2.
    \end{itemize}
    \item Note that monoalkylation proceeds to completion (instead of generating one-half equivalent of dialkylated product) because the alkylated product is less reactive than the starting material (methyl groups are electron-donating through induction, so they destabilize the enolate intermediate).
    \item Cyclization (with alkyl halides).
    \item General form.
    \begin{center}
        \footnotesize
        \setchemfig{atom sep=1.4em}
        \schemestart
            \chemfig{-[:30](=[2]O)-[:-30]@{C1}-[:30](=[2]O)-[:-30]OMe}
            \arrow{->[\ce{NaOMe} / \ce{MeOH} (excess)][\chemfig[atom sep=1.2em]{Br-[:30]@{C2a}-[:-30]@{C2b}-[:30]@{C2c}-[:-30]Br}]}[,2.7]
            \chemfig{-[:30](=[2]O)-[:-30]@{C3a}(*4([6]-@{C3b}-@{C3c}-@{C3d}-))-[:30](=[2]O)-[:-30]OMe}
        \schemestop
        \chemmove{
            \node [below,font=\scriptsize,orx] at (C1)  {4};
            \node [below,font=\scriptsize,orx] at (C2a) {1};
            \node [below,font=\scriptsize,orx] at (C2b) {2};
            \node [below,font=\scriptsize,orx] at (C2c) {3};
            \node [above,font=\scriptsize,orx] at (C3a) {4};
            \node [left ,font=\scriptsize,orx] at (C3b) {1};
            \node [below,font=\scriptsize,orx] at (C3c) {2};
            \node [right,font=\scriptsize,orx] at (C3d) {3};
        }
    \end{center}
    \begin{itemize}
        \item We use one equivalent of the dibromide.
    \end{itemize}
    \item Mechanism.
    \begin{itemize}
        \item The first three steps are the same as the dialkylation mechanism.
        \item The fourth step is that the enolate attacks the other side of the alkyl bromide \emph{intramolecularly} instead of attacking a new molecule.
        \begin{itemize}
            \item The chelate effect is what prefers an intramolecular attack over an intermolecular attack.
            \item There's no such thing as higher rates of collisions than intramolecular.
        \end{itemize}
        \item This is highly effective, great chemistry.
    \end{itemize}
    \item Cyclization (with alcohols).
    \item General form.
    \begin{center}
        \footnotesize
        \setchemfig{atom sep=1.4em}
        \schemestart
            \chemfig{HO-[:30]@{C1a}-[:-30]@{C1b}-[:30]@{C1c}-[:-30]@{C1d}-[:30]@{C1e}-[:-30]OH}
            \arrow{->[1. \ce{TsCl} / \ce{Py}\rule{2.97cm}{0pt}][2. \ce{NaOMe} / \ce{MeOH} (excess), Hacac]}[,3.6]
            \chemfig{-[:30](=[2]O)-[:-30](*6([6]------))-[:30](=[2]O)-[:-30]}
        \schemestop
    \end{center}
    \begin{itemize}
        \item This reaction has broad synthetic utility\footnote{We've learned a lot of reactions that make alcohols. A problem combining those reactions with $\beta$-dicarbonyl chemistry via this reactions would be great, in Levin's opinion.}.
    \end{itemize}
    \item Mechanism.
    \begin{itemize}
        \item The first step takes place as in Figure 9.4 of \textcite{bib:CHEM22000Notes} and Figure 8.3 of \textcite{bib:CHEM22100Notes}.
        \item The second step takes place as in the cyclization of an alkyl halide above, except that tosylate is our leaving group instead of bromide.
    \end{itemize}
    \item Transforming $\beta$-diketoesters to carbonyls.
    \item General form.
    \begin{center}
        \footnotesize
        \setchemfig{atom sep=1.4em}
        \schemestart
            \chemfig{-[:30](=[2]O)-[:-30](*4([6]----))-[:30](=[2]O)-[:-30]OR}
            \arrow{->[1. \ce{NaOH}][2. \ce{H3O+}]}[,1.3]
            \chemfig{-[:30](=[2]O)-[:-30]*4(----)}
        \schemestop
    \end{center}
    \begin{itemize}
        \item We use one equivalent of sodium hydroxide.
    \end{itemize}
    \item Mechanism.
    \begin{figure}[h!]
        \centering
        \footnotesize
        \schemestart
            \chemfig{-[:30](=[2]O)-[:-30](*4([6]----))-[:30](=[2]O)-[:-30]OMe}
            \arrow{->[\ce{NaOH}]}
            \chemfig{-[:30](=[2]O)-[:-30](*4([6]----))-[:30](=[2]O)-[:-30]@{O2}\charge{90:3pt=$\ominus$}{O}}
            \arrow{->[\chemfig[atom sep=1.4em]{@{H3}H-[@{sb3}]@{O3}\charge{90:3pt=$\oplus$}{O}H_2}][-\ce{H2O}]}[,1.3]
            \chemfig{*6([,1.25](-[,1])(=[@{db4}2,1]O)-[@{sb4a}](*4([:-105,1]----))-[@{sb4b}](=[,1]O)-[@{sb4c}]O-[@{sb4d}]@{H4}H)}
            \arrow{->[][-\ce{CO2}]}
            \chemfig{-[:30](-[2]OH)=^[:-30]*4(----)}
            \arrow
            \chemfig{-[:30](=[2]O)-[:-30]*4(----)}
        \schemestop
        \chemmove{
            \draw [curved arrow={11pt}{2pt}] (O2) to[out=90,in=90,looseness=3] (H3);
            \draw [curved arrow={2pt}{2pt}] (sb3) to[out=110,in=130,looseness=3] (O3);
            \draw [curved arrow={3pt}{2pt}] (db4) to[bend right=40,looseness=1.1] (H4);
            \draw [curved arrow={2pt}{2pt}] (sb4d) to[bend right=50,looseness=1.3] (sb4c);
            \draw [curved arrow={2pt}{2pt}] (sb4b) to[bend right=50,looseness=1.3] (sb4a);
        }
        \caption{Transforming $\beta$-diketoesters to carbonyls mechanism.}
        \label{fig:diToMonoCarbonylMechanism}
    \end{figure}
    \begin{itemize}
        \item The first step above proceeds via the saponification mechanism to yield a stable carboxylate (under the given conditions).
        \item The last step above proceeds via the reverse keto-enol tautomerization mechanism (Figure \ref{fig:enolFormationAcid} depicts the forward version), as dictated by the principle of microscopic reversibility and with \ce{H3O+} as the acid.
    \end{itemize}
    \item Transforming malonates to carbonyls.
    \item General form.
    \begin{center}
        \footnotesize
        \setchemfig{atom sep=1.4em}
        \schemestart
            \chemfig{RO-[:30](=[2]O)-[:-30](-[:-70])(-[:-110])-[:30](=[2]O)-[:-30]OR}
            \arrow{->[1. \ce{NaOH} (2 equiv.)][2. \ce{H3O+}\rule{1.37cm}{0pt}]}[,2.2]
            \chemfig{HO-[:30](=[2]O)-[:-30](-[6])-[:30]}
        \schemestop
    \end{center}
    \item Mechanism.
    \begin{itemize}
        \item Double saponification yields a dicarboxylate in the first step.
        \item From here, we protonate both carboxylates to form a dicarboxylic acid.
        \item With free rotation about both $\alpha$-$\beta$ axes, one of the carboxylic acids will eventually rotate into a suitable position for step 3 of Figure \ref{fig:diToMonoCarbonylMechanism} to happen.
        \item We will then have a final keto-enol tautomerization. 
    \end{itemize}
    \item Malonic acid reactions.
    \begin{figure}[h!]
        \centering
        \footnotesize
        \begin{subfigure}[b]{\linewidth}
            \centering
            \schemestart
                \chemfig{HO-[:30](=[2]O)-[:-30]-[:30](=[2]O)-[:-30]OH}
                \arrow{->[$\Delta$]}
                \chemfig{HO-[:30](=[2]O)-[:-30]-[:30,,,,white]}
                \arrow{0}[,0.1]\+
                \chemfig{CO_2}
            \schemestop
            \caption{Slowest.}
            \label{fig:malonicAcidDecompa}
        \end{subfigure}\\[2em]
        \begin{subfigure}[b]{\linewidth}
            \centering
            \schemestart
                \chemfig{HO-[:30](=[2]O)-[:-30](-[6])-[:30](=[2]O)-[:-30]OH}
                \arrow{->[$\Delta$]}
                \chemfig{HO-[:30](=[2]O)-[:-30]-[:30]}
                \arrow{0}[,0.1]\+
                \chemfig{CO_2}
            \schemestop
            \caption{Faster.}
            \label{fig:malonicAcidDecompb}
        \end{subfigure}\\[2em]
        \begin{subfigure}[b]{\linewidth}
            \centering
            \schemestart
                \chemfig{HO-[:30](=[2]O)-[:-30](-[:-70])(-[:-110])-[:30](=[2]O)-[:-30]OH}
                \arrow{->[\SI{25}{\celsius}]}
                \chemfig{HO-[:30](=[2]O)-[:-30](-[6])-[:30]}
                \arrow{0}[,0.1]\+
                \chemfig{CO_2}
            \schemestop
            \caption{Fastest.}
            \label{fig:malonicAcidDecompc}
        \end{subfigure}
        \caption{Decomposition of alkylated malonic acid.}
        \label{fig:malonicAcidDecomp}
    \end{figure}
    \begin{itemize}
        \item When heated, malonic acid decomposes to release \ce{CO2}.
        \item Its methylated forms, however, react much more quickly. Monomethylation confers a noticeable increase in rate, while dimethylation has the reaction proceed at rapidly at room temperature.
        \item The reason for this acceleration is that the carbonyl oxygen and alcohol hydrogen that will react as per Figure \ref{fig:diToMonoCarbonylMechanism} can freely rotate, but we need them close together before the reaction can happen. Indeed, the methyls squeeze the hydrogen and oxygen together because of their steric clash with the neighboring carbonyl and alcohol groups.
    \end{itemize}
    \item Since $\beta$-diketoester enolates are less basic than normal enolates, they permit $\alpha$-alkylation of secondary carbonyl compounds, solving problem 1.
    \item $\beta$-diketoesters are still too reactive to work with aldehydes, so they do \emph{not} solve problem 3.
    \item They do solve problem 2 (regioselectivity), however.
    \begin{figure}[h!]
        \centering
        \footnotesize
        \schemestart
            \chemfig{-[:30]-[:-30]-[:30](=[2]O)-[:-30](-[6]-[:-30]-[6]Ph)-[:30](=[2]O)-[:-30]OMe}
            \arrow{->[\begin{tabular}{l}
                1. \ce{NaOMe}, \ce{MeOH}\\
                2. \ce{MeI}\\
                3. \ce{NaOH}\\
                4. \ce{H3O+}\\
            \end{tabular}]}[,2.1]
            \chemfig{-[:30]-[:-30]-[:30](=[2]O)-[:-30](-[6]Me)-[:30]-[:-30]-[:30]Ph}
        \schemestop
        \caption{$\beta$-ketoesters and regioselectivity.}
        \label{fig:betaRegio}
    \end{figure}
    \begin{itemize}
        \item Suppose we're asked to make the product above from \ce{MeI} and any other compound(s) of our choosing.
        \item Performing a retrosynthetic disconnection yields 1-phenylhept-4-one, a carbonyl that would not be regioselective to LDA as discussed last lecture.
        \item We can, however, start from a $\beta$-ketoester, methylate once (steps 1-2), and then remove the ester (steps 3-4) to yield our final product.
    \end{itemize}
    \item One last regioselectivity tool.
    \begin{figure}[h!]
        \centering
        \footnotesize
        \schemestart
            \chemfig{*6(--=-(=O)--)}
            \arrow{->[*{0} \ce{LiCuMe2}]}[-90]
            \chemfig{*6((-[,,,,white]\phantom{Me})--(-Me)-=(-\charge{45:1pt=$\ominus$}{O})--)}
            \arrow(b--){->[\begin{tabular}{l}
                1. \ce{H3O+}\\
                2. LDA\\
                3. \ce{MeI}\\
            \end{tabular}]}[,1.3]
            \chemfig{*6(--(-Me)--(=O)-(-Me)-)}
            \arrow(@b--){->[\ce{MeI}]}[180]
            \chemfig{*6(--(-Me)-(-Me)-(=O)--)}
        \schemestop
        \caption{Regioselectivity with $\alpha,\beta$-unsaturated compounds.}
        \label{fig:regioselectivityAlphaBeta}
    \end{figure}
    \begin{itemize}
        \item When adding a cuprate to an $\alpha,\beta$-unsaturated compound, 1,4-addition generates an enolate before aqueous workup.
        \item If we proceed with the aqueous workup and then use LDA, we will selectively deprotonate the less sterically encumbered hydrogens, leading to methylation on the carbonyl's left $\alpha$-carbon when \ce{MeI} is introduced.
        \item If we use the existing enolate, we methylate the carbonyl's right $\alpha$-carbon when \ce{MeI} is introduced.
    \end{itemize}
\end{itemize}




\end{document}