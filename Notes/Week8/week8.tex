\documentclass[../notes.tex]{subfiles}

\pagestyle{main}
\renewcommand{\chaptermark}[1]{\markboth{\chaptername\ \thechapter\ (#1)}{}}
\setcounter{chapter}{7}

\begin{document}




\chapter{Amine Reactions and Carbohydrate Structure}
\section{Amines 2}
\begin{itemize}
    \item \marginnote{5/17:}Midterm 2.
    \begin{itemize}
        \item Scores back after class.
        \item Request a regrade (of your whole exam) ASAP if needed.
        \item Raw score: $56\pm 24$ (median 59).
        \item Range: 0-99.
        \item Adjusted: $70\pm 10$.
    \end{itemize}
    \item Today's lecture content in \textcite{bib:SolomonsEtAl}.
    \begin{itemize}
        \item Today: Sections 20.4, 20.12, and 20.6-20.7.
        \item Next time: Sections 22.1-22.2, 22.9A.
        \item Practice problems: 20.19-20.24, 20.26, 20.34-20.36.
    \end{itemize}
    \item Review of last lecture.
    \begin{itemize}
        \item Basicity of amines.
        \item Higher $\pKa(\ce{RNH3+})$ means more basic \ce{RNH2}.
        \item Key: How willing is \ce{N} to share its lone pair.
    \end{itemize}
    \item Today, we will cover the following.
    \begin{enumerate}[label={\Roman*.}]
        \item Properties of amines.
        \begin{enumerate}[label={\Alph*.}]
            \stepcounter{enumii}
            \item Acid-base properties (cotd.).
        \end{enumerate}
        \item Preparation of amines.
        \begin{enumerate}[label={\Alph*.}]
            \item Alkylation.
            \item Reduction.
            \item Hofmann rearrangement.
            \item Curtius rearrangement (review).
        \end{enumerate}
        \item Reactions of amines.
        \begin{enumerate}[label={\Alph*.}]
            \item Hofmann elimination.
            \item Cope elimination.
        \end{enumerate}
    \end{enumerate}
    \item Acid-base properties (cotd.).
    \item Additional example amine $\pKa$'s.
    \begin{itemize}
        \item \ce{Py} has $\pKa(\ce{PyH+})=5.3$.
        \begin{itemize}
            \item Its basicity is intermediate between \ce{NH3} and \ce{PhNH2} due to its $sp^2$ hybridization.
        \end{itemize}
        \item Pyrrole has $\pKa(\ce{RH+})=0.4$.
        \begin{itemize}
            \item Since nitrogen's lone pair here is fully incorporated into the aromatic system, it is not basic.
            \item In fact, pyrrole has $\pKa(\ce{R})=16.5$.
            \item This means that its amine hydrogen is actually mildly acidic (about equivalent to ethanol's hydroxyl hydrogen).
        \end{itemize}
        \item Indole (left to us).
        \begin{itemize}
            \item Indole is like pyrrole: To have $4n+2$ aromatic electrons, it needs nitrogen's lone pair.
        \end{itemize}
        \item See the Aromaticity 2 lecture from \textcite{bib:CHEM22100Notes} for more on aromatic $\pKa$'s.
        \item Amides.
        \begin{itemize}
            \item An amide will coordinate a proton at its oxygen, not its nitrogen.
            \item This protonated species will have $\pKa=0$.
            \item The reason for coordination at oxygen is that the resonance structure with a negative charge on oxygen makes a significant contribution to the overall molecule (oxygen is more electronegative than nitrogen). In fact, this resonance structure implies that the \ce{C-N} bond in an amide is not rotatable, and thus the six atoms \ce{C-C(=O)-NH2} are coplanar.
            \item Additionally, the nitrogen protons are slightly acidic with $\pKa(\ce{RNH2})=18$.
            \item Hence, if we react an amide with a Grignard, we will deprotonate the \ce{NH2} portion.
        \end{itemize}
    \end{itemize}
    \item A note on how protonation can be used to isolate amines (and other basic species) when synthesizing them in the lab.
    \begin{itemize}
        \item Begin by protonating the amines and performing an extraction.
        \item The protonated amines will be attracted to the polar aqueous layer and all other organic compounds can be separated out with the organic layer.
        \item Then we can deprotonate to recover our desired amines.
    \end{itemize}
    \item Preparation of amines.
    \item Alkylation (direct).
    \item General form.
    \begin{equation*}
        \ce{NH3 ->[1. MeI][2. NaOH] MeNH2}
    \end{equation*}
    \item Mechanism.
    \begin{itemize}
        \item The first step proceeds via an S\textsubscript{N}2 mechanism to yield a quaternary ammonium salt.
        \item The second step (a basic workup) removes one of the three nitrogen protons, yielding \ce{H2O + NaI} as side products.
    \end{itemize}
    \item Problems with direct alkylation:
    \begin{itemize}
        \item Even before the basic workup, we have base in solution (\ce{NH3}). This base can accomplish the second-step deprotonation, introducing \ce{MeNH2} into our initial reaction mixture.
        \item But adding alkyl groups (EDGs) creates more reactive amines, so \ce{MeNH2} will preferentially attack \ce{CH3I} compared with \ce{NH3}.
        \item Thus, with direct alkylation, we cannot stop at one particular stage; we will always get a mixture of \ce{NH3}, \ce{MeNH2}, \ce{Me2NH}, \ce{Me3N}, and \ce{Me4NI}.
    \end{itemize}
    \item One potential solution.
    \begin{itemize}
        \item In some cases, we can use excess amine and a bulky alkyl halide.
        \item For example, mixing approx. 20 equivalents of \ce{MeNH2} with \ce{BnCl} yields fairly pure \ce{BnNMeH}.
    \end{itemize}
    \item Alkylation (Gabriel synthesis).
    \item General form.
    \begin{center}
        \footnotesize
        \setchemfig{atom sep=1.4em}
        \schemestart
            \chemfig{*6(=-(*5(-(=O)-NH-(=O)-))=-=-)}
            \arrow{->[1. reagents][2. \ce{MeI}\rule{5.5mm}{0pt}]}[,1.5]
            \chemfig{*6(=-(*5(-(=O)-N(-)-(=O)-))=-=-)}
        \schemestop
    \end{center}
    \begin{itemize}
        \item The Gabriel synthesis prepares primary amines.
        \item The starting material is called \textbf{phthalimide}.
        \item Reagents is either \ce{NaH} (nice because it liberates \ce{H2_{(g)}} as an additional driving force) or \ce{K2CO3} (nice because it's not as strong as \ce{NaH}).
    \end{itemize}
    \item \textbf{Phthalimide}: A $2^\circ$ amine, the lone hydrogen of which has has $\pKa=8.3$ since it is subject to \emph{two} EWG carbonyls and additional resonance with the aromatic ring. \emph{Structure} see above left.
    \item Mechanism.
    \begin{itemize}
        \item The first step is a deprotonation.
        \item The second step proceeds via an S\textsubscript{N}2 mechanism.
        \begin{itemize}
            \item Thus, we preferentially use it in conjunction with primary alkyl halides.
            \item Secondary, allylic, and benzylic alkyl halides will work.
            \item An attempt to run this reaction with a tertiary alkyl halide will lead to elimination.
        \end{itemize}
        \item Notice that the product is a $3^\circ$ amide and thus cannot react any further.
    \end{itemize}
    \item There are three ways to recover the primary amine from the product above.
    \begin{enumerate}
        \item Use \ce{H2SO4}, \ce{H2O}, and heat.
        \begin{itemize}
            \item This amide hydrolysis proceeds analogously to the last several steps of Figure \ref{fig:mechanismNitrileHydrolysis}.
            \item A subsequent deprotonation of \ce{MeNH3+} will be required.
        \end{itemize}
        \item Use \ce{NaOH}, \ce{H2O}, and heat.
        \begin{itemize}
            \item This amide hydrolysis proceeds analogous to the saponification mechanism.
        \end{itemize}
        \item Use \ce{H2NNH2} and reflux.
        \begin{itemize}
            \item See \textcite{bib:SolomonsEtAl} for the mechanism.
        \end{itemize}
    \end{enumerate}
    \item Reduction.
    \begin{itemize}
        \item This method of preparation can proceed from a number of starting materials.
    \end{itemize}
    \item From azides.
    \begin{center}
        \footnotesize
        \setchemfig{atom sep=1.4em}
        \schemestart
            \chemfig{*5([:18]--(--[::-60]-Br)---)}
            \arrow{->[1. \ce{NaN3}\rule{3.3mm}{0pt}][2. reagents]}[,1.5]
            \chemfig{*5([:18]--(--[::-60]-NH_2)---)}
        \schemestop
    \end{center}
    \begin{itemize}
        \item Begin with the desired alkyl group as an alkyl halide.
        \item React it with an azide nucleophile via an S\textsubscript{N}2 mechanism.
        \begin{itemize}
            \item Azide is one of the few nucleophiles that is a very poor base, so it is very good for S\textsubscript{N}2.
        \end{itemize}
        \item Reagents is either \ce{LiAlH4} followed by an acidic workup or hydrogenation (\ce{H2 + Pd/C}).
    \end{itemize}
    \item From nitriles.
    \begin{center}
        \footnotesize
        \setchemfig{atom sep=1.4em}
        \schemestart
            \chemfig{*6(--(--[:30]Br)----)}
            \arrow{->[\begin{tabular}{l}
                1. \ce{NaCN}\\
                2. \ce{LiAlH4}\\
                3. \ce{H3O+}\\
            \end{tabular}]}[,1.4]
            \chemfig{*6(--(--[:30]-NH_2)----)}
        \schemestop
    \end{center}
    \begin{itemize}
        \item Take the desired alkyl group, S\textsubscript{N}2 it with a cyanide nucleophile, and then reduce with \ce{LiAlH4} as in Chapter 17.
        \item Notice that this reaction adds an extra carbon before the amide, unlike with azides.
    \end{itemize}
    \item From amides.
    \begin{center}
        \footnotesize
        \setchemfig{atom sep=1.4em}
        \schemestart
            \chemfig{R-[:30](=[2]O)-[:-30]NR'R''}
            \arrow{->[1. \ce{LiAlH4}][2. \ce{H3O+}\rule{1.4mm}{0pt}]}[,1.4]
            \chemfig{R-[:30]-[:-30]NR'R''}
        \schemestop
    \end{center}
    \begin{itemize}
        \item This is a review reaction; see the discussion associated with Figure \ref{fig:mechanismAmideReduction}.
    \end{itemize}
    \item From iminium ions (reductive amination).
    \begin{center}
        \footnotesize
        \setchemfig{atom sep=1.4em}
        \schemestart
            \chemfig{*6(----(=O)--)}
            \arrow{0}[,0.1]\+
            \chemfig{\chembelow{N}{H}(-[:30])(-[:150])}
            \arrow{->[reagents]}[,1.2]
            \chemfig{*6(----(-N(-[:30])(-[:150]))--)}
        \schemestop
    \end{center}
    \begin{itemize}
        \item This is a very useful reaction in the pharmaceutical industry.
        \item Depending on the reagents, we can accomplish this reaction in a stepwise fashion or all at once.
        \item Stepwise reagents.
        \begin{itemize}
            \item Use mild \ce{H+} followed by a mild hydride source, such as \ce{NaBH4}.
            \item In the first step, we create an enamine in equilibrium with the corresponding iminium ion.
            \item In the second step, hydride attacks the iminium ion's carbon, leading to the final product.
        \end{itemize}
        \item This set of reagents explains the name of the reaction: It is \emph{amination} because we are replacing an oxygen with a nitrogen and \emph{reductive} because we are reducing the iminium ion's double bond.
        \item If we use these reagents, we must (in theory) perform the reaction stepwise because \ce{NaBH4} can reduce any unreacted ketone.
        \begin{itemize}
            \item In reality, there is a trick we can use to do this reaction all at once with these reagents.
        \end{itemize}
        \item One-step reagents.
        \begin{itemize}
            \item Use sodium cyanoborohydride (\ce{NaBH3CN}) in alcoholic solvent (\ce{EtOH} or \ce{MeOH}).
        \end{itemize}
        \item \ce{NaBH3CN} is a weaker hydride source (cyano groups are EWGs), so it can't react with the ketone because it's not electrophilic enough (the charged iminium ion is much more electrophilic).
    \end{itemize}
    \item Reductive amination describes the above reaction of a relatively complicated ketone with a relatively simple amine. If we use, instead, a relatively simple ketone and a relatively complicated amine, the reaction is called\dots
    \item Reductive alkylation.
    \begin{center}
        \footnotesize
        \setchemfig{atom sep=1.4em}
        \schemestart
            \chemfig{*5([:18]--(-\chemabove{N}{H}-[::-60]-Ph)---)}
            \arrow{->[\ce{HCHO}, \ce{MeOH}, \ce{NaBH3CN}]}[,2.8]
            \chemfig{*5([:18]--(-N(-[2])-[::-60]-Ph)---)}
        \schemestop
    \end{center}
    \begin{itemize}
        \item Remember that \ce{HCHO} is formaldehyde, which is our carbon source here.
    \end{itemize}
    \item Reductive amination/alkylation can be more controlled than alkylation.
    \begin{itemize}
        \item This is because with alkylation, our final $3^\circ$ amine could still form a quaternary ammonium salt in the presence of excess \ce{MeI}.
        \item However, a $3^\circ$ amine can never form another iminium ion.
    \end{itemize}
    \item From nitro groups.
    \begin{center}
        \footnotesize
        \setchemfig{atom sep=1.4em}
        \schemestart
            \chemfig{*6(=-=(-NO_2)-=-)}
            \arrow{->[reagents]}[,1.2]
            \chemfig{*6(=-=(-NH_2)-=-)}
        \schemestop
    \end{center}
    \begin{itemize}
        \item Reagents is \ce{H2 + Pd/C}, \ce{Fe + HCl}, or \ce{Zn(Hg) + HCl}.
    \end{itemize}
    \item Hofmann rearrangement.
    \item General form.
    \begin{center}
        \footnotesize
        \setchemfig{atom sep=1.4em}
        \schemestart
            \chemfig{R-[:30](=[2]O)-[:-30]NH_2}
            \arrow{->[\ce{NaOH}, \ce{Br2}][\ce{H2O}]}[,1.5]
            \chemfig{R-NH_2}
        \schemestop
    \end{center}
    \begin{itemize}
        \item Whereas with azides and amides kept the number of carbons constant and nitriles added a carbon, here we lose a carbon.
        \item This reaction is similar to the Curtius rearrangement.
        \item The conditions are identical to those used in the haloform reaction, and we will see that there are homologies in the mechanisms, too.
    \end{itemize}
    \item Mechanism.
    \begin{figure}[h!]
        \centering
        \footnotesize
        \schemestart
            \chemfig{R-[:30](=[2]O)-[:-30]@{N1}N(-[6]H)-[@{sb1}:30]@{H1}H}
            \arrow{->[\chemfig{@{O2}\charge{90:3pt=$\ominus$}{O}H}][-\ce{H2O}]}
            \chemleft{[}
                \subscheme{
                    \chemfig{R-[:30](=[2]O)-[:-30]@{N3}\charge{90:3pt=$\ominus$}{N}-[6]H}
                    \arrow{<->}[-90]
                    \chemfig{R-[:30](-[2]\charge{135:1pt=$\ominus$}{O})=^[:-30]N-[6]H}
                }
            \chemright{]}
            \arrow{->[\chemfig[atom sep=1.4em]{@{Br5a}Br-[@{sb5}]@{Br5b}Br}][-\ce{Br-}]}[,1.2]
            \chemfig{R-[:30](=[2]O)-[:-30]@{N6}N(-[6]Br)-[@{sb6}:30]@{H6}H}
            \arrow{->[\chemfig{@{O7}\charge{90:3pt=$\ominus$}{O}H}][-\ce{H2O}]}
            \chemleft{[}
                \subscheme{
                    \chemfig{R-[@{sb8a}:30](=[2]O)-[@{sb8b}:-30]@{N8}\charge{90:3pt=$\ominus$}{N}-[@{sb8c}6]@{Br8}Br}
                    \arrow{<->}[-90]
                    \chemfig{R-[@{sb9a}:30](-[@{sb9b}2]@{O9}\charge{135:1pt=$\ominus$}{O})=^[:-30]@{N9}N-[@{sb9c}6]@{Br9}Br}
                }
            \chemright{]}
            \arrow{->[][-\ce{Br-}]}
            \chemfig{R-[:30]N=[:-30]C=[:-30]O}
        \schemestop
        \chemmove{
            \draw [curved arrow={11pt}{2pt}] (O2) to[out=90,in=90,looseness=3] (H1);
            \draw [curved arrow={2pt}{2pt}] (sb1) to[out=120,in=90,looseness=2.5] (N1);
            \draw [curved arrow={11pt}{2pt}] (N3) to[out=90,in=90,out looseness=1.5,in looseness=3] (Br5a);
            \draw [curved arrow={2pt}{2pt}] (sb5) to[bend left=90,looseness=3] (Br5b);
            \draw [curved arrow={11pt}{2pt}] (O7) to[out=90,in=90,looseness=3] (H6);
            \draw [curved arrow={2pt}{2pt}] (sb6) to[out=120,in=90,looseness=2.5] (N6);
            \draw [curved arrow={11pt}{2pt},densely dashed] (N8) to[out=90,in=60,looseness=3] (sb8b);
            \draw [curved arrow={2pt}{2pt},densely dashed] (sb8a) to[out=-60,in=-150,looseness=1.3] (N8);
            \draw [curved arrow={2pt}{2pt},densely dashed] (sb8c) to[bend right=90,looseness=3] (Br8);
            \draw [curved arrow={11pt}{2pt}] (O9) to[out=135,in=180,looseness=4] (sb9b);
            \draw [curved arrow={2pt}{2pt}] (sb9a) to[out=-60,in=-150,looseness=1.3] (N9);
            \draw [curved arrow={2pt}{2pt}] (sb9c) to[bend right=90,looseness=3] (Br9);
        }
        \caption{Hofmann rearrangement mechanism (isocyanate formation).}
        \label{fig:mechanismHofmannRearr}
    \end{figure}
    \begin{itemize}
        \item Many of these reactions are reversible, but the equilibria are not that important here.
        \item The first two brominations proceed analogously to those in Figure \ref{fig:mechanismHaloformRxn}.
        \begin{itemize}
            \item Recall that the second bromination happens more readily because having bromine (an EWG) on the nitrogen makes the remaining hydrogen more acidic.
        \end{itemize}
        \item There are two possible rearrangement mechanisms after this for forming the isocyanate.
        \begin{itemize}
            \item The two proceed from different resonance structures.
            \item The one drawn in dashed lines is advocated for by \textcite{bib:SolomonsEtAl}. In it, the \emph{nitrogen} lone pair kicks in, the alkyl group migrates to the nitrogen, and bromine leaves.
            \item The one drawn in solid lines is advocated for by Tang. In it, the \emph{oxygen} lone pair kicks in, the alkyl group migrates to the nitrogen, and bromine leaves.
            \item Tang will accept either on a test despite her preference for the latter.
        \end{itemize}
        \item Once we have an isocyanate, we remove it exactly as in Figure \ref{fig:mechanismCurtiusb}.
        \begin{itemize}
            \item The \ce{NHCOOH} intermediate (intermediate 2 in Figure \ref{fig:mechanismCurtiusb}) is a \textbf{carbamic acid}.
            \item A possible intermediate between intermediate 1 and the carbamic acid is a resonance form of the former wherein we have kicked the oxygen lone pair in and used the double bond to create a lone pair on nitrogen, negatively charging it.
            \item Note that \textcite{bib:SolomonsEtAl} uses a simplified mechanism for these first two steps (isocyanate to carbamic acid). Therein the hydroxide attacks the isocyanate carbon and kicks the \ce{N=C} electrons back onto nitrogen, forming the negatively charged nitrogen intermediate described above in one go. From here, the negative nitrogen can attack water to form the carbamic acid.
            \item The mechanism of \textcite{bib:SolomonsEtAl} is inaccurate, though, because when displaced the electrons will preferentially move toward the more electronegative oxygen.
            \item Regardless, both mechanisms will be accepted as correct in this course.
        \end{itemize}
        \item Other comments.
        \begin{itemize}
            \item Whereas we can isolate the isocyanate intermediate in the Curtius rearrangement, the conditions of the Hofmann rearrangement are such that it will continue reacting immediately upon being formed.
            \item Even though \ce{CO2} is released by this mechanism, we will not observe bubbling in the reaction mixture because the gas is absorbed by the basic media.
            \item Overall, we form isocyanate and then perform two consecutive types of nucleophilic acyl substitution.
        \end{itemize}
    \end{itemize}
    \item An advantage of the Hofmann rearrangement is that it maintains the chirality in the \ce{R} group.
    \begin{itemize}
        \item In particular, we preserve the chirality at the carbon that ends up being $\alpha$ to the amine.
        \item This differs from any of the reductive pathways that use S\textsubscript{N}2, for instance.
    \end{itemize}
    \item Comments on the Curtius rearrangement.
    \begin{itemize}
        \item In the first step, heat is used to transform the (relatively stable) acyl azide into the isocyanate and liberate \ce{N2} gas.
        \item This detail was not mentioned in Lecture 6 and is not shown in Figure \ref{fig:mechanismCurtiusa}.
        \item You can hydrolyze the isocyanate with alcohol instead of water, leading to different products. We will explore this in PSet 5.
    \end{itemize}
    \item Reactions of amines.
    \item Hofmann elimination.
    \item General form.
    \begin{center}
        \footnotesize
        \setchemfig{atom sep=1.4em}
        \schemestart
            \chemfig{-[:-30]-[:30]-[:-30]NH_2}
            \arrow{->[1. \ce{MeI} (excess), base][2. \ce{Ag2O}, \ce{H2O}, $\Delta$\rule{4mm}{0pt}]}[,2.3]
            \chemfig{-[:-30]=^[:30]}
        \schemestop
    \end{center}
    \begin{itemize}
        \item This reaction solves the problem of how to turn \ce{NH2} into a good leaving group so that we can eliminate it.
        \item Example bases are \ce{NEt3} or a \ce{NaOH} pellet (it doesn't even have to be dissolved).
        \item Yields the non-Zaitsev product\footnote{This is why Mrs. Meer introduced the Zaitsev v. Hofmann product!} (less substituted alkene).
    \end{itemize}
    \item Mechanism.
    \begin{itemize}
        \item The first step makes the amide \ce{NH2-} into a good leaving group by transforming it into a quaternary ammonium salt.
        \item The second step causes the elimination. How it works centers around the dual role \ce{Ag2O} serves.
        \begin{itemize}
            \item First, it relinquishes a silver cation to precipitate the iodide anion of the ammonium salt\footnote{Silver and iodide ions preferentially bond because of the HSAB principle from \textcite{bib:CHEM20100Notes}.}.
            \item Second, the remaining \ce{AgO-} species acts as a strong bulky base.
        \end{itemize}
    \end{itemize}
    \item If we use a non-Hofmann elimination base (e.g., \ce{NaOEt}) after forming the quaternary ammonium salt, then we get a mix of products with the Zaitsev product as the major product.
    \item Cope elimination.
    \item General form.
    \begin{center}
        \footnotesize
        \setchemfig{atom sep=1.4em}
        \schemestart
            \chemfig{R-[:-30]-[:30]-[:-30]NMe_2}
            \arrow{->[1. reagents][2. $\Delta$\rule{8.3mm}{0pt}]}[,1.5]
            \chemfig{R-[:-30]=^[:30]}
        \schemestop
    \end{center}
    \begin{itemize}
        \item Reagents is mCPBA or \ce{H2O2}.
        \item We need heat around \SI{150}{\celsius} in the second step.
    \end{itemize}
    \item Mechanism.
    \begin{figure}[h!]
        \centering
        \footnotesize
        \schemestart
            \chemfig{R-[:-30]-[:30]-[:-30]NMe_2}
            \arrow{->[mCPBA]}[,1.2]
            \chemfig{*6([:120,1.25]@{O2}\charge{180=\:,45:1pt=$\ominus$}{O}-@{N2}\charge{90:3pt=$\oplus$}{N}Me_2-[@{sb2a}]-[@{sb2b}](-[,1]R)-[@{sb2c}]@{H2}H)}
            \arrow{->[$\Delta$]}
            \chemfig{R-[:-30]=^[:30]}
        \schemestop
        \chemmove{
            \draw [curved arrow={6pt}{2pt}] (O2) to[bend right=20] (H2);
            \draw [curved arrow={2pt}{2pt}] (sb2c) to[bend right=60,looseness=1.8] (sb2b);
            \draw [curved arrow={2pt}{2pt}] (sb2a) to[bend right=80,looseness=2.5] (N2);
        }
        \caption{Cope elimination mechanism.}
        \label{fig:mechanismCope}
    \end{figure}
    \begin{itemize}
        \item A concerted second step; hence, this is syn elimination.
    \end{itemize}
    \item The Cope elimination is regioselective.
    \begin{figure}[h!]
        \centering
        \footnotesize
        \schemestart
            \chemfig{*6(-(<(-[::60])(-[::-60]))-(<N(-[::60])(-[::-60]))--(<:)--)}
            \arrow{->[1. mCPBA][2. $\Delta$\rule{8.5mm}{0pt}]}[,1.5]
            \chemfig{*6(-(<(-[::60])(-[::-60]))-=-(<:)--)}
        \schemestop
        \caption{Cope elimination regioselectivity.}
        \label{fig:copeRegioselectivity}
    \end{figure}
    \begin{itemize}
        \item The hydrogen and oxygen need to be able to align (i.e., in the transition state). Thus, if they cannot, we will not get elimination there.
        \item Guiding principle: The proton that you pull off has to point in the same direction as the nitrogen.
    \end{itemize}
\end{itemize}



\section{Carbohydrates 1}
\begin{itemize}
    \item \marginnote{5/19:}Today's lecture content in \textcite{bib:SolomonsEtAl}.
    \begin{itemize}
        \item Today: Sections 22.1-22.2 and 22.9A.
        \item Next time: Sections 22.3-22.4, 22.6-22.7, and 22.9B. Read Sections 22.10-22.11.
        \item Practice problems: 22.21, 22.24, 22.28, 22.30.
    \end{itemize}
    \item At this point, we have learned 98\% of all of the reactions we'll learn for Organic Chemistry III.
    \begin{itemize}
        \item We'll see some new reactions in this chapter, but they're all fairly obvious mechanistic analogues of previous reactions.
    \end{itemize}
    \item Review of last lecture.
    \begin{itemize}
        \item You don't need to be able to identify major and minor Cope elimination products for this course, just which ones can form and which ones cannot form.
    \end{itemize}
    \item \textbf{Hydrolysis}: The addition of a water molecule to the starting material and subsequent breaking of some bond.
    \item \textbf{Alcoholysis}: The addition of an alcohol molecule to the starting material and subsequent breaking of some bond.
    \begin{itemize}
        \item Note that neither hydrolysis nor alcoholysis has to involve the formation of two products from one SM (i.e., they don't need the bond broken to have been the only one holding two molecular fragments together).
    \end{itemize}
    \item Tang is dividing this chapter into two parts.
    \begin{enumerate}[label={\Roman*.}]
        \item Names, structures, and properties.
        \item Reactions.
    \end{enumerate}
    \item Today, we will cover the following.
    \begin{enumerate}[label={\Roman*.}]
        \item Names, structures, and properties.
        \begin{enumerate}[label={\Alph*.}]
            \item Definitions.
            \item Structures.
        \end{enumerate}
        \item Reactions.
        \begin{enumerate}[label={\Alph*.}]
            \item Kiliani-Fischer synthesis.
        \end{enumerate}
    \end{enumerate}
    \item The following, up until stated otherwise, is not testable material.
    \item \textbf{Carbohydrate}: Sugar molecules, both simple and complicated.
    \item The etymology of the term, "carbohydrate."
    \begin{itemize}
        \item Before NMR and other characterization methods, chemists determined the molecular formulas of compounds by burning them and measuring how much \ce{O2} is consumed, how much \ce{CO2} is formed, and how much \ce{H2O} is formed.
        \item For a certain class of compounds, they determined that the formulas are of the form \ce{(CH2O)_{$n$}}.
        \item Since these compounds all have a $1:1$ ratio of carbon to water, i.e., their unit structure is a \underline{carbo\vphantom{y}}n \underline{hydrate}, chemists chose the name \emph{carbohydrate}.
    \end{itemize}
    \item The number of degrees of unsaturation of a simple sugar.
    \begin{itemize}
        \item We can discount oxygen from the empirical formula, learning that the simple sugar \ce{(CH2O)_{$n$}} has the same number of degrees of unsaturation as the hydrocarbon \ce{C_{$n$}H_{$2n$}}.
        \item Thus, since a fully saturated hydrocarbon has empirical formula \ce{C_{$n$}H_{$2n+2$}}, we know that a simple sugar has \emph{one} degree of unsaturation.
    \end{itemize}
    \item Photosynthesis and cellular respiration.
    \begin{itemize}
        \item Plants can synthesize glucose from carbon dioxide and water.
        \begin{itemize}
            \item They us an additional special organelle with chlorophyll.
            \item The energy source is sunlight.
            \item Plants can also burn glucose for energy.
        \end{itemize}
        \item Humans cannot synthesize glucose from simpler molecules.
        \begin{itemize}
            \item We can synthesize more complicated carbohydrates from glucose, however.
            \item We solely burn glucose for energy, producing ATP and heat.
        \end{itemize}
    \end{itemize}
    \item Function of carbohydrates.
    \begin{enumerate}
        \item An energy source for humans.
        \begin{itemize}
            \item We can digest glucose.
            \item We can also digest a number more types of sugar, e.g., sucrose (table sugar), starch (from bread), and maltose.
            \item There are types of sugar that we cannot digest, e.g., cellulose.
            \begin{itemize}
                \item Cows and sheep can digest cellulose, however, thanks to specialized bacteria in their gut.
            \end{itemize}
            \item The reason for the difference in digestibility between starch and cellulose hails from the type of linkage used between the sugar monomers.
        \end{itemize}
        \item Other roles.
        \begin{itemize}
            \item Structure materials (cellulose is structural in cell walls).
            \item Components of nucleic acids (think ATP, as well as the sugar-phosphate backbone of DNA).
            \item Many others (the following are a few specific examples).
            \begin{itemize}
                \item The core structure of vitamin C.
                \item Mediation of antibody-antigen recognition.
            \end{itemize}
        \end{itemize}
    \end{enumerate}
    \item Linus Pauling and vitamin C.
    \begin{itemize}
        \item A great chemist who won (solo) the Nobel Prizes for both chemistry and peace.
        \item Every day, he consumed multiple grams of vitamin C --- he believed it would slow the aging process.
        \item Pauling did live to 93, but one wonders if he would have lived a lot longer without such a surplus of vitamin C.
        \item Modern-day vitamin doses.
        \begin{itemize}
            \item We slightly overdose to compensate for the fact that only part of the dose will be absorbed.
            \item As a particular example, shortly after taking a dose of vitamin B, your urine will become bright yellow and have a special smell. This is the excess being flushed out of your body.
        \end{itemize}
    \end{itemize}
    \item Antibody-antigen recognition.
    \begin{itemize}
        \item This is the most important role of sugars in our body.
        \item Most of the mediation is actually done by sugars instead of the primary amino-acid sequence of the antigens and antibodies.
        \item Specific examples.
        \begin{itemize}
            \item The COVID-19 vaccine gives us spike protein antibodies; having the disease gives us more. Discusses how glycoxylation of the COVID-19 spike protein allows our antibodies to recognize it and thus neutralize the virus.
            \item HIV is similar.
            \item A/B/O blood types also work much the same way. These work via modification of lipids. O-type blood has galactose connected to fructose as the terminal of the lipid glycin (this does not generate antibodies). A-type is the same as O-type, except with an additional galactosamine on the galactose. B-type is the same as O-type, except with an additional galactose. AB-type has both the A-type and B-type modifications to O-type.
        \end{itemize}
    \end{itemize}
    \item \textbf{Monosaccharide}: A single sugar monomer.
    \item \textbf{Disaccharide}: Two sugar monomers connected together.
    \item \textbf{Trisaccharide}: A molecule with a three sugar monomers.
    \item We can continue this pattern.
    \item \textbf{Oligosaccharide}: A chain of monosaccharides that is not too long.
    \begin{itemize}
        \item The definition is very vague.
    \end{itemize}
    \item \textbf{Polysaccharide}: A (longer) chain of monosaccharides.
    \item We now begin listing testable material.
    \item Structure.
    \item Fischer projections.
    \begin{figure}[h!]
        \centering
        \footnotesize
        \begin{tikzpicture}
            \node (1) {
                \chemname{\chemfig{HO-C(-[2]H)(-[6]H)-C(-[2]OH)(-[6]H)-C(=[2]O)-H}}{Glyceraldehyde}
            };
    
            \node (2t) [right=1cm of 1,yshift=2cm] {
                \chemfig{HO-[:-30]-[:30](<:[:70]OH)(<[:110]H)-[:-30](=[6]O)-[:30]H}
            };
            \node (2b) [right=1cm of 1,yshift=-2cm] {
                \chemfig{HO-[:-30]-[:30](<:[:70]H)(<[:110]HO)-[:-30](=[6]O)-[:30]H}
            };
            \draw [-CF,shorten <=6pt,shorten >=6pt] (1) -- (1 |- 2t) -- node[above]{(S) enantiomer} (2t);
            \draw [-CF,shorten <=6pt,shorten >=6pt] (1) -- (1 |- 2b) -- node[below]{(R) enantiomer} (2b);
            \draw ($(2t)+(-0.08,1.5)$)
                arc[start angle=-130,end angle=-50,radius=3mm]
                arc[start angle=-50,end angle=-60,radius=3mm]
                -- ++(120:0.3)
                -- ++(-120:0.3)
            ;
            \fill ($(2t)+(0.18,1.44)$) arc[start angle=0,end angle=180,radius=2pt];
            \draw [dashed] ($(2t)+(0.11,1.35)$) -- ++(0,-0.6);
            \draw ($(2b)+(-0.08,1.5)$)
                arc[start angle=-130,end angle=-50,radius=3mm]
                arc[start angle=-50,end angle=-60,radius=3mm]
                -- ++(120:0.3)
                -- ++(-120:0.3)
            ;
            \fill ($(2b)+(0.18,1.44)$) arc[start angle=0,end angle=180,radius=2pt];
            \draw [dashed] ($(2b)+(0.11,1.35)$) -- ++(0,-0.6);
    
            \node (3t) [right=1cm of 2t] {
                \chemfig{CHO>:[6](<H)(<[4]HO)<:[6]-OH}
            };
            \node (3b) [right=1cm of 2b] {
                \chemfig{CHO>:[6](<OH)(<[4,,,,white]\phantom{HO})(<[4]H)<:[6]-OH}
            };
            \node at ($(2t.east)!0.5!(3t.west)$) {\small$\equiv$};
            \node at ($(2b.east)!0.5!(3b.west)$) {\small$\equiv$};
    
            \node (4t) [right=1cm of 3t] {
                \chemfig{CHO-[6](-H)(-[4]HO)-[6]-OH}
            };
            \node (4b) [right=1cm of 3b] {
                \chemfig{CHO-[6](-OH)(-[4,,,,white]\phantom{HO})(-[4]H)-[6]-OH}
            };
            \node at ($(3t.east)!0.5!(4t.west)$) {\small$\equiv$};
            \node at ($(3b.east)!0.5!(4b.west)$) {\small$\equiv$};
        \end{tikzpicture}
        \chemnameinit{}
        \caption{Interpreting the Fischer projections of glyceraldehyde.}
        \label{fig:FischerGlyceraldehyde}
    \end{figure}
    \begin{itemize}
        \item We may have learned Fischer projections first quarter, but we've never used them up until now.
        \begin{itemize}
            \item This makes sense because Fischer projections are only used for sugars nowadays.
        \end{itemize}
        \item The elements of Figure \ref{fig:FischerGlyceraldehyde}.
        \begin{itemize}
            \item The leftmost molecule is \textbf{glyceraldehyde}. It has one chiral carbon (the central one) and thus two enantiomers.
            \item The (S) enantiomer of glyceraldehyde is drawn in line-angle, in a top-view, and as a Fischer projection  along the top row. The same is true of the (R) enantiomer along the bottom row.
            \item The line angle drawings are fairly self-explanatory.
            \item The top view is simply a redrawing of the line-angle but from the perspective of the eyes, with right being out of the page and left being into the page.
            \item The Fischer projection then takes this view and simplifies all of the wedges and dashes to straight lines.
        \end{itemize}
        \item We canonically place the carbonyl group at the top of a Fischer projection.
        \item The backbone of the hydrocarbon always curves into the page in a Fischer projection.
    \end{itemize}
    \item \textbf{Absolute configuration} (of a molecule): The stereochemistry as denoted by R/S nomenclature.
    \item \textbf{D/L nomenclature}: An empirical way of describing the chirality at a carbon in a sugar. \emph{Procedure}
    \begin{enumerate}
        \item Identify the (bottommost) chiral carbon in the Fischer projection of a sugar.
        \item If the hydroxyl group here points to the \underline{l}eft, we insert "L-" before the name of the sugar.
        \item If the hydroxyl group here points to the right, we insert "D-" before the name of the sugar.
    \end{enumerate}
    \item D/L nomenclature in Figure \ref{fig:FischerGlyceraldehyde}.
    \begin{itemize}
        \item The (S) enantiomer is \textbf{L-glyceraldehyde}.
        \item The (R) enantiomer is \textbf{D-glyceraldehyde}.
    \end{itemize}
    \item All naturally occurring sugars have the D configuration.
    \item \textbf{Aldose}: A sugar in which the one degree of unsaturation comes from an aldehyde.
    \item \textbf{Ketose}: A sugar in which the one degree of unsaturation comes from a ketone.
    \item \textbf{Triose}: A sugar with three carbons.
    \item \textbf{Tetrose}: A sugar with four carbons.
    \item \textbf{Pentose}: A sugar with five carbons.
    \item \textbf{Hexose}: A sugar with six carbons.
    \item Examples.
    \begin{itemize}
        \item D-glyceraldehyde is an aldose triose.
        \item \textbf{D-threose} is an aldose tetrose.
        \item \textbf{D-fructose} is a ketose hexose.
    \end{itemize}
    \item \textbf{D-threose}: The following sugar. \emph{Structure}
    \begin{figure}[h!]
        \centering
        \footnotesize
        \chemfig{CHO-[6](-H)(-[4]HO)-[6](-OH)(-[4]H)-[6]-OH}
        \caption{D-threose.}
        \label{fig:dThreose}
    \end{figure}
    \item D-threose has two chiral carbons.
    \begin{itemize}
        \item As we've previously discussed, the "D" tells us the chirality at the bottom carbon.
        \item The name "threose" differentiates the the chirality at the top carbon from that of this molecule's diastereomer, \textbf{D-erythrose} (see Figure \ref{fig:mechanismKilianiFischer}).
        \item Note that there also exist L-threose and L-erythrose (diastereomers of the respective D-versions with inverted chirality at the bottom carbon).
    \end{itemize}
    \item Conformations of D-threose.
    \begin{itemize}
        \item If we draw the conformation of D-threose indicated by the Fischer projection, we will notice that a lot of groups are eclipsed and that this is actually a very high energy conformation of the molecule.
        \item This is why Fischer projections are not used beyond sugars: because normal molecules would never assume such a conformation.
        \item A normal compounds gets drawn in the typical zig-zag/line-angle form.
    \end{itemize}
    \item \textbf{D-fructose}: The following sugar. \emph{Structure}
    \begin{figure}[H]
        \centering
        \footnotesize
        \chemfig{CH_2OH-[6](=O)-[6](-H)(-[4]HO)-[6](-OH)(-[4]H)-[6](-OH)(-[4]H)-[6](-OH)}
        \caption{D-fructose.}
        \label{fig:dFructose}
    \end{figure}
    \item \textbf{D-glucose}: The following sugar. \emph{Also known as} \textbf{dextrose}. \emph{Structure}
    \begin{figure}[h!]
        \centering
        \footnotesize
        \chemfig{CHO-[6](-OH)(-[4]H)-[6](-H)(-[4]HO)-[6](-OH)(-[4]H)-[6](-OH)(-[4]H)-[6](-OH)}
        \caption{D-glucose.}
        \label{fig:dGlucose}
    \end{figure}
    \begin{itemize}
        \item This is the only structure Tang expects us to know by heart; every other structure will be given.
    \end{itemize}
    \item Open and closed structures.
    \item General form.
    \begin{center}
        \footnotesize
        \setchemfig{atom sep=1.4em}
        \schemestart
            \chemfig{CHO-[6](-OH)(-[4]H)-[6](-H)(-[4]HO)-[6](-OH)(-[4]H)-[6](-OH)(-[4]H)-[6](-OH)}
            \arrow{<=>}
            \chemfig{-O-[7](-[,,,,wvbond]OH)<[5,1.3](-[6,0.8]OH)-[4,1.3,,,line width=3.4pt,line cap=round](-[2,0.8]OH)>[3,1.3](-[6,0.8,,2]HO)-[1,1.32](-[2,0.8]CH_2OH)}
        \schemestop
    \end{center}
    \begin{itemize}
        \item This is how we transfer the one degree of unsaturation from an aldehyde or ketone to a ring.
        \item The open and (multiple) closed forms are always in equilibrium.
        \begin{itemize}
            \item For instance, D-glucose prefers to exist as a six-membered ring, but we can find trace amounts ($\approx 0.02\%$) of the open form and even smaller amounts of a five-membered ring form.
            \item We will still never observe four-membered rings (or smaller) or seven-membered rings (or bigger).
        \end{itemize}
        \item The wavy line represents indeterminate chirality.
        \item This is very testable material!
    \end{itemize}
    \item Mechanism.
    \begin{figure}[h!]
        \centering
        \vspace{2em}
        \footnotesize
        \schemestart
            \chemfig{HO-[:-30]-[:30](<[2]OH)-[:-30](<:[6]OH)-[:30](<:[2]OH)-[:-30](<:[6]OH)-[:30](=[2]@{O1}\charge{90=\:}{O})-[:-30]H}
            \arrow{<=>[\chemfig[atom sep=1.4em]{@{H2}H-[@{sb2}]@{O2}\charge{90:3pt=$\oplus$}{O}H_2}][\ce{H2O}]}[,1.4]
            \chemleft{[}
                \subscheme{
                    \chemfig{HO-[:-30]-[:30](<[2]@{O3a}\charge{90=\:}{O}H)-[:-30](<:[6]OH)-[:30](<:[2]OH)-[:-30](<:[6]OH)-[:30]@{C3}(=[@{db3}2]@{O3b}\charge{90:3pt=$\oplus$}{O}H)-[:-30]H}
                    \arrow{<->}
                    \chemfig{-[:20]@{O4a}\charge{90=\:}{O}H-[:-50,,1,,white]@{C4}(-[:20,0.8]H)(=[@{db4}6]@{O4b}\charge{-90:3pt=$\oplus$}{O}H)-[:170,1.207](-[:-70,0.8]OH)-[:-160,1.207](-[:160,0.8]HO)-[:130,1.207](-[:-160,0.8]HO)-[:-10,1.207](-[:110,0.8]-[:40,0.8]OH)}
                }
            \chemright{]}
            \arrow{<=>}[-90]
            \chemfig{-[:20]@{O5}\charge{90:3pt=$\oplus$}{O}(-[@{sb5}:10,0.8]@{H5}H)-[:-50,,1](-[,0.8,,,wvbond]OH)-[:170,1.207](-[:-70,0.8]OH)-[:-160,1.207](-[:160,0.8]HO)-[:130,1.207](-[:-160,0.8]HO)-[:-10,1.207](-[:110,0.8]-[:40,0.8]OH)}
            \arrow{<=>[\ce{H3O+}][*{0.90} {\chemfig[atom sep=1.4em]{H-@{O6}\charge{-90=\:}{O}H}}]}[180,1.3]
            \chemfig{-[:20]O-[:-50,,1](-[,0.8,,,wvbond]OH)-[:170,1.207](-[:-70,0.8]OH)-[:-160,1.207](-[:160,0.8]HO)-[:130,1.207](-[:-160,0.8]HO)-[:-10,1.207](-[:110,0.8]-[:40,0.8]OH)}
        \schemestop
        \chemmove{
            \draw [curved arrow={6pt}{2pt}] (O1) to[out=90,in=90,looseness=2] (H2);
            \draw [curved arrow={2pt}{2pt}] (sb2) to[out=110,in=130,looseness=3] (O2);
            \draw [curved arrow={6pt}{3pt}] (O3a) to[out=90,in=30,out looseness=2,in looseness=3] (C3);
            \draw [curved arrow={3pt}{2pt}] (db3) to[bend left=90,looseness=3] (O3b);
            \draw [curved arrow={6pt}{2pt}] (O4a) to[out=90,in=90,out looseness=1.5,in looseness=4] (C4);
            \draw [curved arrow={3pt}{2pt}] (db4) to[out=20,in=55,looseness=2] (O4b);
            \draw [curved arrow={6pt}{2pt}] (O6) to[out=-90,in=-20,out looseness=1,in looseness=2.3] (H5);
            \draw [curved arrow={2pt}{2pt}] (sb5) to[out=70,in=50,looseness=3] (O5);
        }
        \vspace{1em}
        \caption{Glucose ring-closing mechanism.}
        \label{fig:mechanismRingCloseGlucose}
    \end{figure}
    \begin{itemize}
        \item This process is just hemiacetal/hemiketal formation, as we can see by observing the similarities between Figure \ref{fig:mechanismRingCloseGlucose} and the first three steps of Figure \ref{fig:mechanismKetalFormation}.
        \item We now know that the indeterminacy in the chirality at the one carbon comes from the fact that chirality is set by the intramolecular attack, not the prior stereochemistry.
        \item Note that the six-membered ring of glucose is particularly favored because every substituent is in the equatorial position.
    \end{itemize}
    \item \textbf{Anometic} (carbon): The carbon whose chirality is decided by the attack.
    \item \textbf{$\bm{\beta}$-D-glucose}: The six-membered ring form of D-glucose wherein the hydroxyl group on the anomeric carbon is equatorial. \emph{Also known as} \textbf{$\bm{\beta}$-D-glucopyranose}.
    \item \textbf{$\bm{\alpha}$-D-glucose}: The six-membered ring form of D-glucose wherein the hydroxyl group on the anomeric carbon is axial. \emph{Also known as} \textbf{$\bm{\alpha}$-D-glucopyranose}.
    \item \textbf{Pyran}: The following compound. \emph{Structure}
    \begin{figure}[h!]
        \centering
        \footnotesize
        \chemfig{*6(-O-=--=)}
        \caption{Pyran.}
        \label{fig:pyran}
    \end{figure}
    \begin{itemize}
        \item Pyran is useful in describing the cyclized form of sugars.
        \item Indeed, we call six membered rings with five carbons and one oxygen \textbf{pyranose}.
    \end{itemize}
    \item \textbf{Furan}: The following compound. \emph{Structure}
    \begin{figure}[h!]
        \centering
        \footnotesize
        \chemfig{*5([:-18]-O-=-=)}
        \caption{Furan.}
        \label{fig:furan}
    \end{figure}
    \begin{itemize}
        \item Similarly, we call five membered rings with four carbons and one oxygen \textbf{furanose}.
        \item This is why we have tetrahydrofuran!
    \end{itemize}
    \item The ratio of $\beta$-D-glucose to $\alpha$-D-glucose in water is $64:36$, owing to the former's greater stability.
    \item $\beta$- vs. $\alpha$-linkages play a key role in determining how easy it is to hydrolyze polysaccharaides. For example, starch monomers are connected via $\alpha$-linkages and thus can be easily hydrolyzed; on the other hand, cellulose monomers are connected via $\beta$-linkages and thus cannot be easily hydrolyzed.
    \item Reducing carbohydrates.
    \begin{itemize}
        \item Note that the six-membered ring form of D-glucose (see the end product in Figure \ref{fig:mechanismRingCloseGlucose}) contains a hemiacetal.
        \item Thus, since hemiacetal formation is readily reversible, we will have some of the open form in solution with which we can perform aldehyde chemistry.
        \item In particular, if we mix a solution of glucose with \ce{NaBH4}, we can reduce it.
        \item Note that if we have a hemiketal instead, we can observe muted reactivity.
    \end{itemize}
    \item Fun facts.
    \begin{itemize}
        \item This is not testable content.
        \item Draws the structure of \textbf{sucrose}.
        \begin{itemize}
            \item Sucrose contains an acetal in its ring system, and fructose contains a ketal in its ring system.
            \item Thus, the formation is not readily reversible, so sucrose cannot be reduced by \ce{NaBH4}.
            \item A \textbf{glucometer} (tests blood sugar) determines whether or not the blood is oxidative.
        \end{itemize}
        \item Glucose is 0.75 times as sweet as glucose.
        \item Fructose is 1.75 times as sweet as fructose.
        \item High-fructose corn syrup (industrial synthesis).
        \begin{itemize}
            \item Start from corn starch (cheap compared to cane sugar; mostly made of glucose).
            \item Do hydrolysis to yield glucose.
            \item Add an enzyme to convert some of the glucose into fructose, yielding a $55:45$ ratio of fructose and glucose. We will also have about 20\% water in solution.
        \end{itemize}
    \end{itemize}
    \item \textbf{Sucrose}: A disaccharide of glucose and fructose connected by an $\alpha$-linkage. \emph{Also known as} \textbf{table sugar}. \emph{Structure}
    \begin{figure}[h!]
        \centering
        \footnotesize
        \chemfig{-[:20]O-[:-50,,1](-[6]O-[:10](
            -[:20]O-[:-20](-[:-45,0.8]-[:10,0.8]OH)-[:-130](-[2,0.5,,2]HO)-[4](-[6,0.5]OH)-[:130](-[:100,0.8]-[:35,0.8]OH)
        ))-[:170,1.207](-[:-70,0.8]OH)-[:-160,1.207](-[:160,0.8]HO)-[:130,1.207](-[:-160,0.8]HO)-[:-10,1.207](-[:110,0.8]-[:40,0.8]OH)}
        \caption{Sucrose.}
        \label{fig:sucrose}
    \end{figure}
    \item \textbf{Saccharin}: An artificial sweetener that is 350 times sweeter than table sugar. \emph{Structure}
    \begin{figure}[h!]
        \centering
        \footnotesize
        \chemfig{*6(=-(*5(-S(=[:-92]O)(=[:-52]O)-NH-(=O)-))=-=-)}
        \caption{Saccharin.}
        \label{fig:saccharin}
    \end{figure}
    \item The discovery of saccharin (according to legend).
    \begin{itemize}
        \item The chemist who first synthesized it didn't wash his hand very well after lab, went home, touched the dough of a cake that his wife was baking, and commented that it was really sweet. He later figured out that the additional sweetness was coming from saccharin.
    \end{itemize}
    \item \textbf{Aspartame}: An artificial sweetener that is 180 times sweeter than table sugar. \emph{Structure}
    \begin{figure}[H]
        \centering
        \footnotesize
        \chemfig{HO-[:30](=[2]O)-[:-30]-[:30](<[2]NH_2)-[:-30](=[6]O)-[:30]\chemabove{N}{H}-[:-30](<:[6]-[:-30]*6(-=-=-=))-[:30](=[2]O)-[:-30]OMe}
        \caption{Aspartame.}
        \label{fig:aspartame}
    \end{figure}
    \item \textbf{Sucralose}: An artificial sweetener that is 600 times sweeter than table sugar. \emph{Also known as} \textbf{Splenda}. \emph{Structure}
    \begin{figure}[h!]
        \centering
        \footnotesize
        \chemfig{-[:20]O-[:-50,,1](-[6]O-[:10](
            -[:20]O-[:-20](-[:-45,0.8]-[:10,0.8]Cl)-[:-130](-[2,0.5,,2]HO)-[4](-[6,0.5]OH)-[:130](-[:100,0.8]-[:35,0.8]Cl)
        ))-[:170,1.207](-[:-70,0.8]OH)-[:-160,1.207](-[:160,0.8]HO)-[:130,1.207](-[2,0.8]Cl)-[:-10,1.207](-[:110,0.8]-[:40,0.8]OH)}
        \caption{Sucralose.}
        \label{fig:sucralose}
    \end{figure}
    \item The discovery of sucralose (according to legend).
    \begin{itemize}
        \item A PI asks his (Indian) post-doc to synthesize this chlorinated sucrose analogue. The post-doc reports the complete synthesis and the PI ask him to "test it," as in characterize spectroscopically. However, the post-doc hears "taste it," is confused but goes and does so, and reports, "sir, it's very sweet."
    \end{itemize}
    \item All artificial sweeteners were discovered by accident --- there's no reason you'd think they're sweet just by looking at the structure.
    \item Artificial sweeteners are dangerous in extreme excess, but not in any ordinary amount.
    \begin{itemize}
        \item Tang goes over an experiment on mice and their kidneys to support this claim.
        \item If we use a sugar alcohol, however, (e.g., sorbitol), we can get diarrhea for eating too much.
        \begin{itemize}
            \item This is because these are not as sweet, so we use more; but since they are not digestible, consuming too much does not bode well for the digestive system.
        \end{itemize}
    \end{itemize}
    \item Kiliani-Fischer synthesis.
    \item General form.
    \begin{center}
        \footnotesize
        \setchemfig{atom sep=1.4em}
        \schemestart
            \chemfig{CHO-[6](-OH)(-[4]H)-[6]-OH}
            \arrow{->[\begin{tabular}{l}
                1. \ce{HCN}\\
                2. \ce{Ba(OH)2}\\
                3. \ce{H3O+}\\
                4. \ce{Na(Hg), H2O, $\pH=3\text{-}5$}\\
            \end{tabular}]}[,2.8]
            \chemfig{CHO-[6](-OH)(-[4]H)-[6](-OH)(-[4]H)-[6]-OH}
            \+{,,-2.3em}
            \chemfig{CHO-[6](-H)(-[4]HO)-[6](-OH)(-[4]H)-[6]-OH}
        \schemestop
    \end{center}
    \begin{itemize}
        \item This is a \textbf{chain-elongation reaction}.
    \end{itemize}
    \item \textbf{Chain-elongation reaction}: A reaction that takes a simple sugar and extends it by one carbon.
    \item Mechanism.
    \begin{figure}[H]
        \centering
        \footnotesize
        \begin{tikzpicture}
            \node (1) {
                \chemfig{CHO-[6](-OH)(-[4]H)-[6]-OH}
            };
    
            \node (2t) at (8.8,2) {
                \schemestart
                    \chemfig{CN-[6](-OH)(-[4]H)-[6](-OH)(-[4]H)-[6]-OH}
                    \arrow{->[1. \ce{Ba(OH)2}][2. \ce{H3O+}\rule{3.5mm}{0pt}]}[,1.5]
                    \chemfig{COOH-[6](-OH)(-[4]H)-[6](-OH)(-[4]H)-[6]-OH}
                    \arrow{->[\ce{Na(Hg), H2O}][$\pH=3\text{-}5$]}[,1.6]
                    \chemname{\chemfig{CHO-[6](-OH)(-[4]H)-[6](-OH)(-[4]H)-[6]-OH}}{D-erythrose}
                \schemestop
            };
            \node (2b) at (8.8,-2) {
                \schemestart
                    \chemfig{CN-[6](-H)(-[4]HO)-[6](-OH)(-[4]H)-[6]-OH}
                    \arrow{->[1. \ce{Ba(OH)2}][2. \ce{H3O+}\rule{3.5mm}{0pt}]}[,1.5]
                    \chemfig{COOH-[6](-H)(-[4]HO)-[6](-OH)(-[4]H)-[6]-OH}
                    \arrow{->[\ce{Na(Hg), H2O}][$\pH=3\text{-}5$]}[,1.6]
                    \chemname{\chemfig{CHO-[6](-H)(-[4]HO)-[6](-OH)(-[4]H)-[6]-OH}}{D-threose}
                \schemestop
            };
            \draw ([xshift=2mm]1.east) -- node[above]{\ce{HCN}} ++(1,0);
            \draw [CF-CF] ([xshift=2.2cm,yshift=2cm]1.east) -- ++(-1,0) -- ++(0,-4) -- ++(1,0);
        \end{tikzpicture}
        \chemnameinit{}
        \caption{Kiliani-Fischer synthesis mechanism.}
        \label{fig:mechanismKilianiFischer}
    \end{figure}
    \begin{itemize}
        \item In the first step (cyanohydrin formation), the cyanide ion can attack either face. Thus, the first step leads to the formation of two diastereomers.
        \item In the second and third steps, it is an empirical finding that barium hydroxide is a better base source than \ce{NaOH} or something else of the sort, but it is hard to say that any of these alternatives flat-out would not work. What we are basically accomplishing here, though, is nitrile hydrolysis (see Figure \ref{fig:mechanismNitrileHydrolysis}).
        \begin{itemize}
            \item Tang indicates that the \ce{Ba(OH)2} generates a carboxylate from our nitrile and then the acid protonates it to a carboxylic acid.
        \end{itemize}
        \item The last step is a reduction.
    \end{itemize}
\end{itemize}




\end{document}