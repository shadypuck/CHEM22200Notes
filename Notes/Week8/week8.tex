\documentclass[../notes.tex]{subfiles}

\pagestyle{main}
\renewcommand{\chaptermark}[1]{\markboth{\chaptername\ \thechapter\ (#1)}{}}
\setcounter{chapter}{7}

\begin{document}




\chapter{Amine Reactions and Carbohydrate Structure}
\section{Amines 2}
\begin{itemize}
    \item \marginnote{5/17:}Midterm 2.
    \begin{itemize}
        \item Scores back after class.
        \item Request a regrade (of your whole exam) ASAP if needed.
        \item Raw score: $56\pm 24$ (median 59).
        \item Range: 0-99.
        \item Adjusted: $70\pm 10$.
    \end{itemize}
    \item Today's lecture content in \textcite{bib:SolomonsEtAl}.
    \begin{itemize}
        \item Today: Sections 20.4, 20.12, and 20.6-20.7.
        \item Next time: Sections 22.1-22.2, 22.9A.
        \item Practice problems: 20.19-20.24, 20.26, 20.34-20.36.
    \end{itemize}
    \item Review of last lecture.
    \begin{itemize}
        \item Basicity of amines.
        \item Higher $\pKa(\ce{RNH3+})$ means more basic \ce{RNH2}.
        \item Key: How willing is \ce{N} to share its lone pair.
    \end{itemize}
    \item Today, we will cover the following.
    \begin{enumerate}[label={\Roman*.}]
        \item Properties of amines.
        \begin{enumerate}[label={\Alph*.}]
            \stepcounter{enumii}
            \item Acid-base properties (cotd.).
        \end{enumerate}
        \item Preparation of amines.
        \begin{enumerate}[label={\Alph*.}]
            \item Alkylation.
            \item Reduction.
            \item Hofmann rearrangement.
            \item Curtius rearrangement (review).
        \end{enumerate}
        \item Reactions of amines.
        \begin{enumerate}[label={\Alph*.}]
            \item Hofmann elimination.
            \item Cope elimination.
        \end{enumerate}
    \end{enumerate}
    \item Acid-base properties (cotd.).
    \item Additional example amine $\pKa$'s.
    \begin{itemize}
        \item \ce{Py} has $\pKa(\ce{PyH+})=5.3$.
        \begin{itemize}
            \item Its basicity is intermediate between \ce{NH3} and \ce{PhNH2} due to its $sp^2$ hybridization.
        \end{itemize}
        \item Pyrrole has $\pKa(\ce{RH+})=0.4$.
        \begin{itemize}
            \item Since nitrogen's lone pair here is fully incorporated into the aromatic system, it is not basic.
            \item In fact, pyrrole has $\pKa(\ce{R})=16.5$.
            \item This means that its amine hydrogen is actually mildly acidic (about equivalent to ethanol's hydroxyl hydrogen).
        \end{itemize}
        \item Indole (left to us).
        \begin{itemize}
            \item Indole is like pyrrole: To have $4n+2$ aromatic electrons, it needs nitrogen's lone pair.
        \end{itemize}
        \item See the Aromaticity 2 lecture from \textcite{bib:CHEM22100Notes} for more on aromatic $\pKa$'s.
        \item Amides.
        \begin{itemize}
            \item An amide will coordinate a proton at its oxygen, not its nitrogen.
            \item This protonated species will have $\pKa=0$.
            \item The reason for coordination at oxygen is that the resonance structure with a negative charge on oxygen makes a significant contribution to the overall molecule (oxygen is more electronegative than nitrogen). In fact, this resonance structure implies that the \ce{C-N} bond in an amide is not rotatable, and thus the six atoms \ce{C-C(=O)-NH2} are coplanar.
            \item Additionally, the nitrogen protons are slightly acidic with $\pKa(\ce{RNH2})=18$.
            \item Hence, if we react an amide with a Grignard, we will deprotonate the \ce{NH2} portion.
        \end{itemize}
    \end{itemize}
    \item A note on how protonation can be used to isolate amines (and other basic species) when synthesizing them in the lab.
    \begin{itemize}
        \item Begin by protonating the amines and performing an extraction.
        \item The protonated amines will be attracted to the polar aqueous layer and all other organic compounds can be separated out with the organic layer.
        \item Then we can deprotonate to recover our desired amines.
    \end{itemize}
    \item Preparation of amines.
    \item Alkylation (direct).
    \item General form.
    \begin{equation*}
        \ce{NH3 ->[1. MeI][2. NaOH] MeNH2}
    \end{equation*}
    \item Mechanism.
    \begin{itemize}
        \item The first step proceeds via an S\textsubscript{N}2 mechanism to yield a quaternary ammonium salt.
        \item The second step (a basic workup) removes one of the three nitrogen protons, yielding \ce{H2O + NaI} as side products.
    \end{itemize}
    \item Problems with direct alkylation:
    \begin{itemize}
        \item Even before the basic workup, we have base in solution (\ce{NH3}). This base can accomplish the second-step deprotonation, introducing \ce{MeNH2} into our initial reaction mixture.
        \item But adding alkyl groups (EDGs) creates more reactive amines, so \ce{MeNH2} will preferentially attack \ce{CH3I} compared with \ce{NH3}.
        \item Thus, with direct alkylation, we cannot stop at one particular stage; we will always get a mixture of \ce{NH3}, \ce{MeNH2}, \ce{Me2NH}, \ce{Me3N}, and \ce{Me4NI}.
    \end{itemize}
    \item One potential solution.
    \begin{itemize}
        \item In some cases, we can use excess amine and a bulky alkyl halide.
        \item For example, mixing approx. 20 equivalents of \ce{MeNH2} with \ce{BnCl} yields fairly pure \ce{BnNMeH}.
    \end{itemize}
    \item Alkylation (Gabriel synthesis).
    \item General form.
    \begin{center}
        \footnotesize
        \setchemfig{atom sep=1.4em}
        \schemestart
            \chemfig{*6(=-(*5(-(=O)-NH-(=O)-))=-=-)}
            \arrow{->[1. reagents][2. \ce{MeI}\rule{5.5mm}{0pt}]}[,1.5]
            \chemfig{*6(=-(*5(-(=O)-N(-)-(=O)-))=-=-)}
        \schemestop
    \end{center}
    \begin{itemize}
        \item The Gabriel synthesis prepares primary amines.
        \item The starting material is called \textbf{phthalimide}.
        \item Reagents is either \ce{NaH} (nice because it liberates \ce{H2_{(g)}} as an additional driving force) or \ce{K2CO3} (nice because it's not as strong as \ce{NaH}).
    \end{itemize}
    \item \textbf{Phthalimide}: A $2^\circ$ amine, the lone hydrogen of which has has $\pKa=8.3$ since it is subject to \emph{two} EWG carbonyls and additional resonance with the aromatic ring. \emph{Structure} see above left.
    \item Mechanism.
    \begin{itemize}
        \item The first step is a deprotonation.
        \item The second step proceeds via an S\textsubscript{N}2 mechanism.
        \begin{itemize}
            \item Thus, we preferentially use it in conjunction with primary alkyl halides.
            \item Secondary, allylic, and benzylic alkyl halides will work.
            \item An attempt to run this reaction with a tertiary alkyl halide will lead to elimination.
        \end{itemize}
        \item Notice that the product is a $3^\circ$ amide and thus cannot react any further.
    \end{itemize}
    \item There are three ways to recover the primary amine from the product above.
    \begin{enumerate}
        \item Use \ce{H2SO4}, \ce{H2O}, and heat.
        \begin{itemize}
            \item This amide hydrolysis proceeds analogously to the last several steps of Figure \ref{fig:mechanismNitrileHydrolysis}.
            \item A subsequent deprotonation of \ce{MeNH3+} will be required.
        \end{itemize}
        \item Use \ce{NaOH}, \ce{H2O}, and heat.
        \begin{itemize}
            \item This amide hydrolysis proceeds analogous to the saponification mechanism.
        \end{itemize}
        \item Use \ce{H2NNH2} and reflux.
        \begin{itemize}
            \item See \textcite{bib:SolomonsEtAl} for the mechanism.
        \end{itemize}
    \end{enumerate}
    \item Reduction.
    \begin{itemize}
        \item This method of preparation can proceed from a number of starting materials.
    \end{itemize}
    \item From azides.
    \begin{center}
        \footnotesize
        \setchemfig{atom sep=1.4em}
        \schemestart
            \chemfig{*5([:18]--(--[::-60]-Br)---)}
            \arrow{->[1. \ce{NaN3}\rule{3.3mm}{0pt}][2. reagents]}[,1.5]
            \chemfig{*5([:18]--(--[::-60]-NH_2)---)}
        \schemestop
    \end{center}
    \begin{itemize}
        \item Begin with the desired alkyl group as an alkyl halide.
        \item React it with an azide nucleophile via an S\textsubscript{N}2 mechanism.
        \begin{itemize}
            \item Azide is one of the few nucleophiles that is a very poor base, so it is very good for S\textsubscript{N}2.
        \end{itemize}
        \item Reagents is either \ce{LiAlH4} followed by an acidic workup or hydrogenation (\ce{H2 + Pd/C}).
    \end{itemize}
    \item From nitriles.
    \begin{center}
        \footnotesize
        \setchemfig{atom sep=1.4em}
        \schemestart
            \chemfig{*6(--(--[:30]Br)----)}
            \arrow{->[\begin{tabular}{l}
                1. \ce{NaCN}\\
                2. \ce{LiAlH4}\\
                3. \ce{H3O+}\\
            \end{tabular}]}[,1.4]
            \chemfig{*6(--(--[:30]-NH_2)----)}
        \schemestop
    \end{center}
    \begin{itemize}
        \item Take the desired alkyl group, S\textsubscript{N}2 it with a cyanide nucleophile, and then reduce with \ce{LiAlH4} as in Chapter 17.
        \item Notice that this reaction adds an extra carbon before the amide, unlike with azides.
    \end{itemize}
    \item From amides.
    \begin{center}
        \footnotesize
        \setchemfig{atom sep=1.4em}
        \schemestart
            \chemfig{R-[:30](=[2]O)-[:-30]NR'R''}
            \arrow{->[1. \ce{LiAlH4}][2. \ce{H3O+}\rule{1.4mm}{0pt}]}[,1.4]
            \chemfig{R-[:30]-[:-30]NR'R''}
        \schemestop
    \end{center}
    \begin{itemize}
        \item This is a review reaction; see the discussion associated with Figure \ref{fig:mechanismAmideReduction}.
    \end{itemize}
    \item From iminium ions (reductive amination).
    \begin{center}
        \footnotesize
        \setchemfig{atom sep=1.4em}
        \schemestart
            \chemfig{*6(----(=O)--)}
            \arrow{0}[,0.1]\+
            \chemfig{\chembelow{N}{H}(-[:30])(-[:150])}
            \arrow{->[reagents]}[,1.2]
            \chemfig{*6(----(-N(-[:30])(-[:150]))--)}
        \schemestop
    \end{center}
    \begin{itemize}
        \item This is a very useful reaction in the pharmaceutical industry.
        \item Depending on the reagents, we can accomplish this reaction in a stepwise fashion or all at once.
        \item Stepwise reagents.
        \begin{itemize}
            \item Use mild \ce{H+} followed by a mild hydride source, such as \ce{NaBH4}.
            \item In the first step, we create an enamine in equilibrium with the corresponding iminium ion.
            \item In the second step, hydride attacks the iminium ion's carbon, leading to the final product.
        \end{itemize}
        \item This set of reagents explains the name of the reaction: It is \emph{amination} because we are replacing an oxygen with a nitrogen and \emph{reductive} because we are reducing the iminium ion's double bond.
        \item If we use these reagents, we must (in theory) perform the reaction stepwise because \ce{NaBH4} can reduce any unreacted ketone.
        \begin{itemize}
            \item In reality, there is a trick we can use to do this reaction all at once with these reagents.
        \end{itemize}
        \item One-step reagents.
        \begin{itemize}
            \item Use sodium cyanoborohydride (\ce{NaBH3CN}) in alcoholic solvent (\ce{EtOH} or \ce{MeOH}).
        \end{itemize}
        \item \ce{NaBH3CN} is a weaker hydride source (cyano groups are EWGs), so it can't react with the ketone because it's not electrophilic enough (the charged iminium ion is much more electrophilic).
    \end{itemize}
    \item Reductive amination describes the above reaction of a relatively complicated ketone with a relatively simple amine. If we use, instead, a relatively simple ketone and a relatively complicated amine, the reaction is called\dots
    \item Reductive alkylation.
    \begin{center}
        \footnotesize
        \setchemfig{atom sep=1.4em}
        \schemestart
            \chemfig{*5([:18]--(-\chemabove{N}{H}-[::-60]-Ph)---)}
            \arrow{->[\ce{HCHO}, \ce{MeOH}, \ce{NaBH3CN}]}[,2.8]
            \chemfig{*5([:18]--(-N(-[2])-[::-60]-Ph)---)}
        \schemestop
    \end{center}
    \begin{itemize}
        \item Remember that \ce{HCHO} is formaldehyde, which is our carbon source here.
    \end{itemize}
    \item Reductive amination/alkylation can be more controlled than alkylation.
    \begin{itemize}
        \item This is because with alkylation, our final $3^\circ$ amine could still form a quaternary ammonium salt in the presence of excess \ce{MeI}.
        \item However, a $3^\circ$ amine can never form another iminium ion.
    \end{itemize}
    \item From nitro groups.
    \begin{center}
        \footnotesize
        \setchemfig{atom sep=1.4em}
        \schemestart
            \chemfig{*6(=-=(-NO_2)-=-)}
            \arrow{->[reagents]}[,1.2]
            \chemfig{*6(=-=(-NH_2)-=-)}
        \schemestop
    \end{center}
    \begin{itemize}
        \item Reagents is \ce{H2 + Pd/C}, \ce{Fe + HCl}, or \ce{Zn(Hg) + HCl}.
    \end{itemize}
    \item Hofmann rearrangement.
    \item General form.
    \begin{center}
        \footnotesize
        \setchemfig{atom sep=1.4em}
        \schemestart
            \chemfig{R-[:30](=[2]O)-[:-30]NH_2}
            \arrow{->[\ce{NaOH}, \ce{Br2}][\ce{H2O}]}[,1.5]
            \chemfig{R-NH_2}
        \schemestop
    \end{center}
    \begin{itemize}
        \item Whereas with azides and amides kept the number of carbons constant and nitriles added a carbon, here we lose a carbon.
        \item This reaction is similar to the Curtius rearrangement.
        \item The conditions are identical to those used in the haloform reaction, and we will see that there are homologies in the mechanisms, too.
    \end{itemize}
    \item Mechanism.
    \begin{figure}[h!]
        \centering
        \footnotesize
        \schemestart
            \chemfig{R-[:30](=[2]O)-[:-30]@{N1}N(-[6]H)-[@{sb1}:30]@{H1}H}
            \arrow{->[\chemfig{@{O2}\charge{90:3pt=$\ominus$}{O}H}][-\ce{H2O}]}
            \chemleft{[}
                \subscheme{
                    \chemfig{R-[:30](=[2]O)-[:-30]@{N3}\charge{90:3pt=$\ominus$}{N}-[6]H}
                    \arrow{<->}[-90]
                    \chemfig{R-[:30](-[2]\charge{135:1pt=$\ominus$}{O})=^[:-30]N-[6]H}
                }
            \chemright{]}
            \arrow{->[\chemfig[atom sep=1.4em]{@{Br5a}Br-[@{sb5}]@{Br5b}Br}][-\ce{Br-}]}[,1.2]
            \chemfig{R-[:30](=[2]O)-[:-30]@{N6}N(-[6]Br)-[@{sb6}:30]@{H6}H}
            \arrow{->[\chemfig{@{O7}\charge{90:3pt=$\ominus$}{O}H}][-\ce{H2O}]}
            \chemleft{[}
                \subscheme{
                    \chemfig{R-[@{sb8a}:30](=[2]O)-[@{sb8b}:-30]@{N8}\charge{90:3pt=$\ominus$}{N}-[@{sb8c}6]@{Br8}Br}
                    \arrow{<->}[-90]
                    \chemfig{R-[@{sb9a}:30](-[@{sb9b}2]@{O9}\charge{135:1pt=$\ominus$}{O})=^[:-30]@{N9}N-[@{sb9c}6]@{Br9}Br}
                }
            \chemright{]}
            \arrow{->[][-\ce{Br-}]}
            \chemfig{R-[:30]N=[:-30]C=[:-30]O}
        \schemestop
        \chemmove{
            \draw [curved arrow={11pt}{2pt}] (O2) to[out=90,in=90,looseness=3] (H1);
            \draw [curved arrow={2pt}{2pt}] (sb1) to[out=120,in=90,looseness=2.5] (N1);
            \draw [curved arrow={11pt}{2pt}] (N3) to[out=90,in=90,out looseness=1.5,in looseness=3] (Br5a);
            \draw [curved arrow={2pt}{2pt}] (sb5) to[bend left=90,looseness=3] (Br5b);
            \draw [curved arrow={11pt}{2pt}] (O7) to[out=90,in=90,looseness=3] (H6);
            \draw [curved arrow={2pt}{2pt}] (sb6) to[out=120,in=90,looseness=2.5] (N6);
            \draw [curved arrow={11pt}{2pt},densely dashed] (N8) to[out=90,in=60,looseness=3] (sb8b);
            \draw [curved arrow={2pt}{2pt},densely dashed] (sb8a) to[out=-60,in=-150,looseness=1.3] (N8);
            \draw [curved arrow={2pt}{2pt},densely dashed] (sb8c) to[bend right=90,looseness=3] (Br8);
            \draw [curved arrow={11pt}{2pt}] (O9) to[out=135,in=180,looseness=4] (sb9b);
            \draw [curved arrow={2pt}{2pt}] (sb9a) to[out=-60,in=-150,looseness=1.3] (N9);
            \draw [curved arrow={2pt}{2pt}] (sb9c) to[bend right=90,looseness=3] (Br9);
        }
        \caption{Hofmann rearrangement mechanism (isocyanate formation).}
        \label{fig:mechanismHofmannRearr}
    \end{figure}
    \begin{itemize}
        \item Many of these reactions are reversible, but the equilibria are not that important here.
        \item The first two brominations proceed analogously to those in Figure \ref{fig:mechanismHaloformRxn}.
        \begin{itemize}
            \item Recall that the second bromination happens more readily because having bromine (an EWG) on the nitrogen makes the remaining hydrogen more acidic.
        \end{itemize}
        \item There are two possible rearrangement mechanisms after this for forming the isocyanate.
        \begin{itemize}
            \item The two proceed from different resonance structures.
            \item The one drawn in dashed lines is advocated for by \textcite{bib:SolomonsEtAl}. In it, the \emph{nitrogen} lone pair kicks in, the alkyl group migrates to the nitrogen, and bromine leaves.
            \item The one drawn in solid lines is advocated for by Tang. In it, the \emph{oxygen} lone pair kicks in, the alkyl group migrates to the nitrogen, and bromine leaves.
            \item Tang will accept either on a test despite her preference for the latter.
        \end{itemize}
        \item Once we have an isocyanate, we remove it exactly as in Figure \ref{fig:mechanismCurtiusb}.
        \begin{itemize}
            \item The \ce{NHCOOH} intermediate (intermediate 2 in Figure \ref{fig:mechanismCurtiusb}) is a \textbf{carbamic acid}.
            \item A possible intermediate between intermediate 1 and the carbamic acid is a resonance form of the former wherein we have kicked the oxygen lone pair in and used the double bond to create a lone pair on nitrogen, negatively charging it.
            \item Note that \textcite{bib:SolomonsEtAl} uses a simplified mechanism for these first two steps (isocyanate to carbamic acid). Therein the hydroxide attacks the isocyanate carbon and kicks the \ce{N=C} electrons back onto nitrogen, forming the negatively charged nitrogen intermediate described above in one go. From here, the negative nitrogen can attack water to form the carbamic acid.
            \item The mechanism of \textcite{bib:SolomonsEtAl} is inaccurate, though, because when displaced the electrons will preferentially move toward the more electronegative oxygen.
            \item Regardless, both mechanisms will be accepted as correct in this course.
        \end{itemize}
        \item Other comments.
        \begin{itemize}
            \item Whereas we can isolate the isocyanate intermediate in the Curtius rearrangement, the conditions of the Hofmann rearrangement are such that it will continue reacting immediately upon being formed.
            \item Even though \ce{CO2} is released by this mechanism, we will not observe bubbling in the reaction mixture because the gas is absorbed by the basic media.
            \item Overall, we form isocyanate and then perform two consecutive types of nucleophilic acyl substitution.
        \end{itemize}
    \end{itemize}
    \item An advantage of the Hofmann rearrangement is that it maintains the chirality in the \ce{R} group.
    \begin{itemize}
        \item In particular, we preserve the chirality at the carbon that ends up being $\alpha$ to the amine.
        \item This differs from any of the reductive pathways that use S\textsubscript{N}2, for instance.
    \end{itemize}
    \item Comments on the Curtius rearrangement.
    \begin{itemize}
        \item In the first step, heat is used to transform the (relatively stable) acyl azide into the isocyanate and liberate \ce{N2} gas.
        \item This detail was not mentioned in Lecture 6 and is not shown in Figure \ref{fig:mechanismCurtiusa}.
        \item You can hydrolyze the isocyanate with alcohol instead of water, leading to different products. We will explore this in PSet 5.
    \end{itemize}
    \item Reactions of amines.
    \item Hofmann elimination.
    \item General form.
    \begin{center}
        \footnotesize
        \setchemfig{atom sep=1.4em}
        \schemestart
            \chemfig{-[:-30]-[:30]-[:-30]NH_2}
            \arrow{->[1. \ce{MeI} (excess), base][2. \ce{Ag2O}, \ce{H2O}, $\Delta$\rule{4mm}{0pt}]}[,2.3]
            \chemfig{-[:-30]=^[:30]}
        \schemestop
    \end{center}
    \begin{itemize}
        \item This reaction solves the problem of how to turn \ce{NH2} into a good leaving group so that we can eliminate it.
        \item Example bases are \ce{NEt3} or a \ce{NaOH} pellet (it doesn't even have to be dissolved).
        \item Yields the non-Zaitsev product\footnote{This is why Mrs. Meer introduced the Zaitsev v. Hofmann product!} (less substituted alkene).
    \end{itemize}
    \item Mechanism.
    \begin{itemize}
        \item The first step makes the amide \ce{NH2-} into a good leaving group by transforming it into a quaternary ammonium salt.
        \item The second step causes the elimination. How it works centers around the dual role \ce{Ag2O} serves.
        \begin{itemize}
            \item First, it relinquishes a silver cation to precipitate the iodide anion of the ammonium salt\footnote{Silver and iodide ions preferentially bond because of the HSAB principle from \textcite{bib:CHEM20100Notes}.}.
            \item Second, the remaining \ce{AgO-} species acts as a strong bulky base.
        \end{itemize}
    \end{itemize}
    \item If we use a non-Hofmann elimination base (e.g., \ce{NaOEt}) after forming the quaternary ammonium salt, then we get a mix of products with the Zaitsev product as the major product.
    \item Cope elimination.
    \item General form.
    \begin{center}
        \footnotesize
        \setchemfig{atom sep=1.4em}
        \schemestart
            \chemfig{R-[:-30]-[:30]-[:-30]NMe_2}
            \arrow{->[1. reagents][2. $\Delta$\rule{8.3mm}{0pt}]}[,1.5]
            \chemfig{R-[:-30]=^[:30]}
        \schemestop
    \end{center}
    \begin{itemize}
        \item Reagents is mCPBA or \ce{H2O2}.
        \item We need heat around \SI{150}{\celsius} in the second step.
    \end{itemize}
    \item Mechanism.
    \begin{figure}[h!]
        \centering
        \footnotesize
        \schemestart
            \chemfig{R-[:-30]-[:30]-[:-30]NMe_2}
            \arrow{->[mCPBA]}[,1.2]
            \chemfig{*6([:120,1.25]@{O2}\charge{180=\:,45:1pt=$\ominus$}{O}-@{N2}\charge{90:3pt=$\oplus$}{N}Me_2-[@{sb2a}]-[@{sb2b}](-[,1]R)-[@{sb2c}]@{H2}H)}
            \arrow{->[$\Delta$]}
            \chemfig{R-[:-30]=^[:30]}
        \schemestop
        \chemmove{
            \draw [curved arrow={6pt}{2pt}] (O2) to[bend right=20] (H2);
            \draw [curved arrow={2pt}{2pt}] (sb2c) to[bend right=60,looseness=1.8] (sb2b);
            \draw [curved arrow={2pt}{2pt}] (sb2a) to[bend right=80,looseness=2.5] (N2);
        }
        \caption{Cope elimination mechanism.}
        \label{fig:mechanismCope}
    \end{figure}
    \begin{itemize}
        \item A concerted second step; hence, this is syn elimination.
    \end{itemize}
    \item The Cope elimination is regioselective.
    \begin{figure}[h!]
        \centering
        \footnotesize
        \schemestart
            \chemfig{*6(-(<(-[::60])(-[::-60]))-(<N(-[::60])(-[::-60]))--(<:)--)}
            \arrow{->[1. mCPBA][2. $\Delta$\rule{8.5mm}{0pt}]}[,1.5]
            \chemfig{*6(-(<(-[::60])(-[::-60]))-=-(<:)--)}
        \schemestop
        \caption{Cope elimination regioselectivity.}
        \label{fig:copeRegioselectivity}
    \end{figure}
    \begin{itemize}
        \item The hydrogen and oxygen need to be able to align (i.e., in the transition state). Thus, if they cannot, we will not get elimination there.
        \item Guiding principle: The proton that you pull off has to point in the same direction as the nitrogen.
    \end{itemize}
\end{itemize}




\end{document}