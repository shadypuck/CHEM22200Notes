\documentclass[../notes.tex]{subfiles}

\pagestyle{main}
\renewcommand{\chaptermark}[1]{\markboth{\chaptername\ \thechapter\ (#1)}{}}
\setcounter{chapter}{16}

\begin{document}




\chapter{Carboxylic Acids and Derivatives}
\section{Carboxylic Acids and Derivatives 1}
\begin{itemize}
    \item \marginnote{4/7:}We now consider compounds that have heteroatoms where the $\alpha$ carbon of the carbonyl used to be.
    \begin{itemize}
        \item The heteroatoms can be oxygen (esters), nitrogen, etc.
    \end{itemize}
    \item Today, we will do oxygen and nitrogen nucleophiles but in this context.
    \begin{itemize}
        \item Next Tuesday, we will do carbon and hydrogen nucleophiles in this context.
    \end{itemize}
    \item Carboxylic acid derivatives.
    \begin{figure}[h!]
        \centering
        \footnotesize
        \begin{subfigure}[b]{0.19\linewidth}
            \centering
            \chemfig{R-[:30](=[2]O)-[:-30]OH}
            \caption{Carboxylic acid.}
            \label{fig:carboxylicAcidDerivativesa}
        \end{subfigure}
        \begin{subfigure}[b]{0.19\linewidth}
            \centering
            \chemfig{R-[:30](=[2]O)-[:-30]O-[:30]R'}
            \caption{Ester.}
            \label{fig:carboxylicAcidDerivativesb}
        \end{subfigure}
        \begin{subfigure}[b]{0.19\linewidth}
            \centering
            \chemfig{R-[:30](=[2]O)-[:-30]X}
            \caption{Acid halide.}
            \label{fig:carboxylicAcidDerivativesc}
        \end{subfigure}
        \begin{subfigure}[b]{0.19\linewidth}
            \centering
            \chemfig{R-[:30](=[2]O)-[:-30]O-[:30](=[2]O)-[:-30]R'}
            \caption{Acid anhydride.}
            \label{fig:carboxylicAcidDerivativesd}
        \end{subfigure}
        \begin{subfigure}[b]{0.19\linewidth}
            \centering
            \chemfig{R-[:30](=[2]O)-[:-30]N(-[6]R'')-[:30]R'}
            \caption{Amide.}
            \label{fig:carboxylicAcidDerivativese}
        \end{subfigure}\\[2em]
        \begin{subfigure}[b]{0.22\linewidth}
            \centering
            \chemfig{R-C~N}
            \caption{Nitrile.}
            \label{fig:carboxylicAcidDerivativesf}
        \end{subfigure}
        \begin{subfigure}[b]{0.22\linewidth}
            \centering
            \chemfig{R-[:-30]O-[:30](=[2]O)-[:-30]O-[:30]R'}
            \caption{Carbonate.}
            \label{fig:carboxylicAcidDerivativesg}
        \end{subfigure}
        \begin{subfigure}[b]{0.22\linewidth}
            \centering
            \chemfig{R-[:-30]O-[:30](=[2]O)-[:-30]N(-[6]R'')-[:30]R'}
            \caption{Carbamate.}
            \label{fig:carboxylicAcidDerivativesh}
        \end{subfigure}
        \begin{subfigure}[b]{0.22\linewidth}
            \centering
            \chemfig{R-[:-30]N(-[6]R')-[:30](=[2]O)-[:-30]N(-[6]R''')-[:30]R''}
            \caption{Urea.}
            \label{fig:carboxylicAcidDerivativesi}
        \end{subfigure}
        \caption{Carboxylic acid derivatives.}
        \label{fig:carboxylicAcidDerivatives}
    \end{figure}
    \begin{itemize}
        \item Once again, we will not be tested on nomenclature, but it's good to know.
        \item Acid anhydrides are so named because it is two carboxylic acids, minus a water molecule.
        \item Nitriles are still a carbon bonded to three heteroatoms; it's just the same heteroatom.
    \end{itemize}
    \item A key property of carboxylic acids is that they're\dots acidic.
    \item Acidity.
    \begin{itemize}
        \item Gives the $\pKa$'s of benzoic acid, benzyl alcohol, and phenol to demonstrate that resonance is king in determining acidity.
        \begin{itemize}
            \item Benzoic acid is more acidic than phenol, which is more acidic than benzyl alcohol.
        \end{itemize}
        \item Inductive effects (changes to the $\alpha$ carbon) play a smaller role.
        \item EWGs on arene rings when present play an even smaller role.
        \item These latter two effects allow us to fine-tune acidity.
    \end{itemize}
    \item Methods of carboxylic acid synthesis.
    \begin{enumerate}
        \item Overoxidation.
        \item Carboxylation of Grignards or lithiates.
        \item Nitrile hydrolysis.
    \end{enumerate}
    \item Overoxidation.
    \item General form.
    \begin{equation*}
        \ce{CRH(OH) ->[CrO3, H2SO4][H2O] RCOOH}
    \end{equation*}
    \begin{itemize}
        \item Note that the reagents constitute Jones reagent.
    \end{itemize}
    \item Mechanism.
    \begin{itemize}
        \item Virtually identical to that from \textcite{bib:CHEM22100Notes}.
    \end{itemize}
    \item Carboxyliation of Grignards and lithiates.
    \item General form.
    \begin{equation*}
        \ce{RLi ->[1. CO2][2. H3O+] RCOOH}
    \end{equation*}
    \begin{itemize}
        \item Note that we may use either lithiates (\ce{RLi}) or Grignards (\ce{RMgBr}), even though only an organolithium compound is shown above.
    \end{itemize}
    \item Mechanism.
    \begin{figure}[h!]
        \centering
        \footnotesize
        \schemestart
            \chemfig{R-[@{sb1}]Li}
            \arrow{->[\chemfig[atom sep=1.4em]{O=@{C2}C=[@{db2}]@{O2}O}]}[,1.5]
            \chemname[-3em]{\chemfig{R-[:30](=[2]O)-[:-30]\charge{45:1pt=$\ominus$}{O}-[,0.6,,,white]\charge{45:1pt=$\oplus$}{Li}}}{Carboxylate salt}
            \arrow{->[\ce{H3O+}]}
            \chemfig{R-[:30](=[2]O)-[:-30]OH}
        \schemestop
        \chemmove{
            \draw [curved arrow={2pt}{2pt}] (sb1) to[bend left=90,looseness=1.5] (C2);
            \draw [curved arrow={3pt}{2pt}] (db2) to[bend left=90,looseness=3] (O2);
        }
        \caption{Carboxylation of lithiates mechanism.}
        \label{fig:mechanismLithiateCarboxylation}
    \end{figure}
    \item Mechanistic interlude: Nucleophilic acyl substitution.
    \begin{figure}[H]
        \centering
        \footnotesize
        \begin{subfigure}[b]{\linewidth}
            \centering
            \schemestart
                \chemfig{R-[:30](=[2]O)-[:-30]LG}
                \arrow{0}[,0.1]\+
                \chemfig{Nu-H}
                \arrow{->[cat. \ce{HX}]}[,1.2]
                \chemfig{R-[:30](-[:110]HO)(-[:70]Nu)-[:-30]LG}
                \arrow{->[cat. \ce{HX}]}[,1.2]
                \chemfig{R-[:30](=[2]O)-[:-30]Nu}
                \arrow{0}[,0.1]\+
                \chemfig{LG-H}
            \schemestop
            \caption{Acid-catalyzed reactivity.}
            \label{fig:acidCarboxylica}
        \end{subfigure}\\[2em]
        \begin{subfigure}[b]{\linewidth}
            \centering
            \schemestart
                \chemfig{@{Nu1}Nu-[@{sb1}]@{H1}H}
                \arrow{->[\chemfig{@{B2}\charge{90=\:}{B}}]}
                \chemfig{@{Nu3}\charge{90=\:,45:1pt=$\ominus$}{Nu}}
                \+
                \chemfig{H\charge{90:3pt=$\oplus$}{B}}
                \arrow{-U>[\chemfig[atom sep=1.4em]{R-[:30]@{C5}(=[@{db5}2]@{O5}O)-[:-30]LG}][][][][80]}[,1.5]
                \chemfig{R-[:30](-[@{sb6a}:110]@{O6}\charge{180=\:,90:3pt=$\ominus$}{O})(-[:70]Nu)-[@{sb6b}:-30]@{LG6}LG}
                \arrow{0}[,0.1]\+
                \chemfig{H\charge{90:3pt=$\oplus$}{B}}
                \arrow{-U>[][\chemfig[atom sep=1.4em]{R-[:30](=[2]O)-[:-30]Nu}][][][80]}[,1.3]
                \chemfig{@{LG9}\charge{45:1pt=$\ominus$}{LG}}
                \arrow{0}[,0.1]\+
                \chemfig{@{H10}H-[@{sb10}]@{B10}\charge{90:3pt=$\oplus$}{B}}
                \arrow{->[][-\ce{B}]}
                \chemfig{LG-H}
            \schemestop
            \chemmove{
                \draw [curved arrow={6pt}{2pt}] (B2) to[out=90,in=90,looseness=2] (H1);
                \draw [curved arrow={2pt}{2pt}] (sb1) to[bend right=90,looseness=3] (Nu1);
                \draw [curved arrow={6pt}{3pt}] (Nu3) to[out=90,in=150] (C5);
                \draw [curved arrow={3pt}{2pt}] (db5) to[bend right=90,looseness=3] (O5);
                \draw [curved arrow={6pt}{2pt}] (O6) to[out=180,in=-150,in looseness=4,out looseness=3] (sb6a);
                \draw [curved arrow={2pt}{2pt}] (sb6b) to[bend left=90,looseness=3] (LG6);
                \draw [curved arrow={10pt}{2pt}] (LG9) to[bend left=50,looseness=1.5] (H10);
                \draw [curved arrow={2pt}{2pt}] (sb10) to[bend right=90,looseness=3] (B10);
            }
            \caption{Base-catalyzed reactivity.}
            \label{fig:acidCarboxylicb}
        \end{subfigure}
        \caption{The typical reactivity of carboxylic acid derivatives.}
        \label{fig:acidCarboxylic}
    \end{figure}
    \begin{itemize}
        \item This mode of reactivity is the one that is most typical of carboxylic acid derivatives.
        \begin{itemize}
            \item It is so-named because the portion of a carboxylic acid derivative that is not the leaving group is called an acyl group, and we are substituting one group on the acyl for another.
        \end{itemize}
        \item Think of all of the carboxylic acid derivatives (see Figure \ref{fig:carboxylicAcidDerivatives}) as containing a leaving group on one of their sides.
        \begin{itemize}
            \item When these compounds react nucleophiles, the nucleophile replaces the leaving group.
        \end{itemize}
        \item These reactions are either acid- or base-catalyzed.
        \begin{itemize}
            \item In the acid-catalyzed version (Figure \ref{fig:acidCarboxylica}), the first step proceeds exactly as in Figure \ref{fig:acidPromotedNua}, except that $\ce{R$'$}=\ce{LG}$. The second step proceeds exactly as in Figure \ref{fig:acidPromotedNub}, except that it is the leaving group that is protonated and kicked out instead of the nucleophile we just added in.
            \item The basic mechanism is related to Figure \ref{fig:basePromotedNu}, but rather than being a straight replication, the alkoxide species produced in Figure \ref{fig:basePromotedNua} proceeds straight to the reactivity of the alkoxide in Figure \ref{fig:basePromotedNub} (see Figure \ref{fig:acidCarboxylicb}).
        \end{itemize}
    \end{itemize}
    \item \textbf{Tetrahedral intermediates}: The nucleophilic acyl substitution intermediates (of both the acidic and basic pathways) that have four groups attached to the central carbon.
    \begin{figure}[h!]
        \centering
        \footnotesize
        \begin{subfigure}[b]{0.3\linewidth}
            \centering
            \chemfig{R-[:30](-[:110]HO)(-[:70]Nu)-[:-30]LG}
            \caption{Acidic intermediate.}
            \label{fig:tetrahedralIntermediatesa}
        \end{subfigure}
        \begin{subfigure}[b]{0.3\linewidth}
            \centering
            \chemfig{R-[:30](-[:110]\charge{135:1pt=$\ominus$}{O})(-[:70]Nu)-[:-30]LG}
            \caption{Basic intermediate.}
            \label{fig:tetrahedralIntermediatesb}
        \end{subfigure}
        \caption{The tetrahedral intermediates.}
        \label{fig:tetrahedralIntermediates}
    \end{figure}
    \begin{itemize}
        \item Historically, the name arose when scientists were arguing about whether or not an $sp^3$ carbon could be in this reaction. Some scientists supported the theory that these tetrahedral intermediates existed, while others disagreed.
    \end{itemize}
    \item Nitrile hydrolysis.
    \item General form.
    \begin{equation*}
        \ce{RCN + H3O+ -> RCOOH + NH4+}
    \end{equation*}
    \begin{itemize}
        \item Note that here we're using a stoichiometric full equivalent of acid, not just catalytic acid, because we are liberating ammonia which mops up our acid, forming \ce{NH4+} as a byproduct.
        \item The existence of this reaction is the reason we consider nitriles to be carboxylic acid derivatives (i.e., because we can interconvert them with carboxylic acids). 
    \end{itemize}
    \item Mechanism.
    \begin{figure}[h!]
        \centering
        \footnotesize
        \schemestart
            \chemfig{R-C~@{N1}\charge{90=\:}{N}}
            \arrow{->[\chemfig[atom sep=1.4em]{@{H2}H-[@{sb2}]@{O2}\charge{90:3pt=$\oplus$}{O}H_2}][-\ce{H2O}]}[,1.3]
            \chemfig{R-@{C3}C~[@{tb3}]@{N3}\charge{90:3pt=$\oplus$}{N}-H}
            \arrow{->[\chemfig{H_2@{O4}\charge{90=\:}{O}}]}
            \chemfig{R-[:30](-[2]@{O5}\charge{90:3pt=$\oplus$}{O}(-[@{sb5}:30]@{H5}H)(-[:150]H))=[:-30]N-[:30]H}
            \arrow{->[\chemfig{H_2@{O6}\charge{90=\:}{O}}][-\ce{H3O+}]}
            \chemleft{[}
                \subscheme{
                    \chemfig{R-[:30](-[@{sb7}2]@{O7}\charge{0=\:}{O}-[:150]H)=[@{db7}:-30]@{N7}N-[:30]H}
                    \arrow{<->}[-90]
                    \chemfig{R-[:30](=[2]\charge{45:1pt=$\oplus$}{O}-[:150]H)-[:-30]@{N8}\charge{-90:3pt=$\ominus$}{N}-[:30]H}
                }
            \chemright{]}
            \arrow{->[*{0}\setchemfig{atom sep=1.4em}\chemfig{@{H9}H-[@{sb9}]@{O9}\charge{90:3pt=$\oplus$}{O}H_2}]}[-90]
            \chemfig{R-[:30]@{C10}(=[@{db10}2]@{O10}\charge{45:1pt=$\oplus$}{O}-[:150]H)-[:-30]N(-[6]H)-[:30]H}
            \arrow{->[*{0.-90}\chemfig{H_2@{O11}\charge{90=\:}{O}}]}[180]
            \chemfig{R-[:30](-[:110]HO)(-[:70]@{O12}\charge{-70:2pt=$\oplus$}{O}H-[@{sb12}2]@{H12}H)-[:-30]NH_2}
            \arrow{->[*{0.-90}\chemfig{H_2@{O13}\charge{90=\:}{O}}][-\ce{H3O+}]}[180]
            \chemfig{R-[:30](-[:110]HO)(-[:70]OH)-[:-30]@{N14}\charge{90=\:}{N}H_2}
            \arrow{->[\chemfig[atom sep=1.4em]{@{H15}H-[@{sb15}]@{O15}\charge{90:3pt=$\oplus$}{O}H_2}][-\ce{H2O}]}[180,1.3]
            \chemfig{R-[:30](-[@{sb16a}:110]H@{O16}\charge{90=\:}{O})(-[:70]OH)-[@{sb16b}:-30]@{N16}\charge{-90:3pt=$\oplus$}{N}H_3}
            \arrow[-90]
            \subscheme{
                \chemfig{R-[:30](=[2]@{O17}\charge{90:3pt=$\oplus$}{O}-[@{sb17}:30]@{H17}H)-[:-30]OH}
                \arrow{0}[,0.1]\+
                \chemfig{@{N18}\charge{90=\:}{N}H_3}
            }
            \arrow{->[][-\ce{NH4+}]}
            \chemfig{R-[:30](=[2]O)-[:-30]OH}
        \schemestop
        \chemmove{
            \draw [curved arrow={6pt}{2pt}] (N1) to[out=90,in=90,looseness=3] (H2);
            \draw [curved arrow={2pt}{2pt}] (sb2) to[out=110,in=130,looseness=3] (O2);
            \draw [curved arrow={6pt}{2pt}] (O4) to[out=90,in=90,looseness=1.2] (C3);
            \draw [curved arrow={4pt}{2pt}] (tb3) to[bend right=90,looseness=3] (N3);
            \draw [curved arrow={6pt}{2pt}] (O6) to[out=90,in=0] (H5);
            \draw [curved arrow={2pt}{2pt}] (sb5) to[bend left=90,looseness=3] (O5);
            \draw [curved arrow={6pt}{2pt},blx] (O7) to[bend left=90,looseness=3] (sb7);
            \draw [curved arrow={3pt}{2pt},blx] (db7) to[bend right=90,looseness=3] (N7);
            \draw [curved arrow={0pt}{2pt}] ([yshift=-10pt]N8.south) to[out=-90,in=75] (H9);
            \draw [curved arrow={2pt}{2pt}] (sb9) to[bend right=90,looseness=4] (O9);
            \draw [curved arrow={6pt}{3pt}] (O11) to[out=90,in=150,looseness=1.5] (C10);
            \draw [curved arrow={3pt}{2pt}] (db10) to[bend right=90,looseness=3] (O10);
            \draw [curved arrow={6pt}{2pt}] (O13) to[out=90,in=180,looseness=1.1] (H12);
            \draw [curved arrow={2pt}{2pt}] (sb12) to[bend right=70,looseness=2.5] (O12);
            \draw [curved arrow={6pt}{3pt}] (N14) to[out=75,in=90,out looseness=2] (H15);
            \draw [curved arrow={2pt}{2pt}] (sb15) to[out=110,in=130,looseness=3] (O15);
            \draw [curved arrow={6pt}{2pt}] (O16) to[out=90,in=-150,looseness=7.5] (sb16a);
            \draw [curved arrow={2pt}{2pt}] (sb16b) to[bend left=90,looseness=3] (N16);
            \draw [curved arrow={6pt}{2pt}] (N18) to[out=90,in=0,looseness=1.1] (H17);
            \draw [curved arrow={2pt}{2pt}] (sb17) to[bend left=90,looseness=3] (O17);
        }
        \caption{Nitrile hydrolysis mechanism.}
        \label{fig:mechanismNitrileHydrolysis}
    \end{figure}
    \begin{itemize}
        \item Note that the fourth intermediate is one deprotonation away from being an amide.
        \begin{itemize}
            \item However, the reaction conditions do not produce an amide but continue as drawn to a carboxylic acid.
            \item This is because in general, the amide oxygen is more basic than the nitrile nitrogen, so if the conditions are such that the nitrile will begin the reaction, the amide will certainly finish it.
        \end{itemize}
        \item Note that there are some enzymes that can stop at the amide through various mechanisms that recognize one species as substrate but not another.
        \item Every once in a while, people will claim that they've isolated the amide in this mechanism, but these results are hard to reproduce because of the above facts.
        \item If we do add up all of the equivalents of water and acid added, we can see that only one equivalent of acid is added, overall (and two equivalents of water).
    \end{itemize}
    \item Dehydration of amides.
    \item General form.
    \begin{equation*}
        \ce{RCONH2 ->[reagents][\Delta] RCN}
    \end{equation*}
    \begin{itemize}
        \item This is the reverse reaction to nitrile hydrolysis.
        \item Reagents is either \ce{SOCl2} or \ce{POCl3}.
        \item \ce{SOCl2} and \ce{POCl3} are \textbf{dehydrating agents}.
    \end{itemize}
    \item \textbf{Dehydrating agent}: A chemical that drives conversions in which water is lost from a molecule.
    \begin{itemize}
        \item Notice how the amide overall loses two hydrogens and an oxygen (i.e., a water molecule overall) in Figure \ref{fig:mechanismAmideDehydration}.
    \end{itemize}
    \item Mechanism.
    \begin{figure}[h!]
        \centering
        \footnotesize
        \schemestart
            \chemfig{R-[:30](=[@{db1}2]O)-[@{sb1}:-30]@{N1}\charge{90=\:}{N}H_2}
            \arrow{->[\chemfig[atom sep=1.4em]{@{S2}S(=[2]O)(-[@{sb2}:-30]@{Cl2}Cl)(-[:-150]Cl)}]}[,1.5]
            \chemfig{R-[:30](-[2]O-[:30]S(=[2]O)-[:-30]Cl)=[:-30]@{N3}\charge{90:3pt=$\oplus$}{N}(-[@{sb3}6]@{H3}H)(-[:30]H)}
            \arrow{0}[,0.1]\+
            \chemfig{@{Cl4}\charge{-90=\:,45:1pt=$\ominus$}{Cl}}
            \arrow{->[][-\ce{HCl}]}
            \chemfig{R-[:30](-[@{sb5a}2]O-[@{sb5b}:30]S(=[2]O)-[@{sb5c}:-30]@{Cl5}Cl)=[@{db5}:-30]@{N5}\charge{-90=\:}{N}H}
            \arrow{->[][*{0}-\ce{SO2}]}[-90,1.5,shorten <=5mm,shorten >=3mm]
            \subscheme{
                \chemfig{R-C~@{N6}\charge{90:3pt=$\oplus$}{N}-[@{sb6}]@{H6}H}
                \arrow{0}[,0.1]\+
                \chemfig{@{Cl7}\charge{90=\:,45:1pt=$\ominus$}{Cl}}
            }
            \arrow{->[][*{0.90}-\ce{HCl}]}[180]
            \chemfig{R-C~N}
        \schemestop
        \chemmove{
            \draw [curved arrow={6pt}{2pt}] (N1) to[bend right=70,looseness=2.5] (sb1);
            \draw [curved arrow={3pt}{2pt}] (db1) to[out=10,in=150] (S2);
            \draw [curved arrow={2pt}{2pt}] (sb2) to[bend left=90,looseness=3] (Cl2);
            \draw [curved arrow={6pt}{2pt}] (Cl4) to[out=-90,in=0,looseness=1.1] (H3);
            \draw [curved arrow={2pt}{2pt}] (sb3) to[bend left=90,looseness=3] (N3);
            \draw [curved arrow={6pt}{3pt}] (N5) to[out=-90,in=-120,looseness=4] (db5);
            \draw [curved arrow={2pt}{2pt}] (sb5a) to[bend right=60,looseness=1.5] (sb5b);
            \draw [curved arrow={2pt}{2pt}] (sb5c) to[bend left=90,looseness=3] (Cl5);
            \draw [curved arrow={6pt}{2pt}] (Cl7) to[out=90,in=90,looseness=2] (H6);
            \draw [curved arrow={2pt}{2pt}] (sb6) to[bend left=90,looseness=3] (N6);
        }
        \caption{Dehydration of amides mechanism.}
        \label{fig:mechanismAmideDehydration}
    \end{figure}
    \begin{itemize}
        \item Part of the reason the amide oxygen is such a good nucleophile is because the nitrogen can participate, as in step 1 above.
        \item Driving force: Kicking out a gas (\ce{SO2}) and chloride.
        \item Note that the mechanism implies that we must have an amide with two \ce{H}'s (esp., we cannot have one or two \ce{R} groups in their place).
        \item Although only the mechanism for \ce{SOCl2} is illustrated, the mechanism is virtually identical for \ce{POCl3}.
    \end{itemize}
    \item Comparing methods 2 and 3 of synthesizing carboxylic acids.
    \begin{figure}[H]
        \centering
        \footnotesize
        \begin{tikzpicture}
            \node{\chemfig{*6(---(-Br)---)}};
            \draw (1.5,0) -- ++(1,0);
            \draw [CF-CF] (3.5,1.5) -- node[above]{\ce{KCN}} ++(-1,0) -- ++(0,-3) -- node[above]{\ce{Mg${}^\circ$}} ++(1,0);
            
            \node at (5,1.5)  {\chemfig{*6(---(-[,,,,white]\phantom{MgBr})(-CN)---)}};
            \node at (5,-1.5) {\chemfig{*6(---(-MgBr)---)}};
            \draw (6.5,1.5) -- node[above]{\ce{H3O+}} ++(1.5,0) -- ++(0,-3) -- node[above]{1. \ce{CO2}\rule{2mm}{0pt}} node[below]{2. \ce{H3O+}} ++(-1.5,0);
            \draw [-CF] (8,0) -- ++(1,0);
    
            \node at (11,0) {\chemfig{*6(---(-(=[2]O)-[:-30]OH)---)}};
        \end{tikzpicture}
        \caption{Two ways to synthesize a carboxylic acid from an alkyl halide.}
        \label{fig:2and3}
    \end{figure}
    \begin{itemize}
        \item Both carboxylation and nitrile hydrolysis achieve the same end result from the same starting material, begging the question of why both are necessary.
        \item The answer lies in the fact that both suit different types of reaction conditions.
        \item Carboxylation is strongly basic, so we can't use molecules with free \ce{H}'s.
        \item Nitrile hydrolysis proceeds through S\textsubscript{N}2 to start, so we can't use tertiary bromides.
        \begin{itemize}
            \item This is important on part of PSet 1!
        \end{itemize}
    \end{itemize}
    \item Methods of ester synthesis.
    \begin{enumerate}
        \item Nucleophilic.
        \item Fischer esterification.
    \end{enumerate}
    \item Nucleophilic.
    \item General form.
    \begin{center}
        \footnotesize
        \setchemfig{atom sep=1.4em}
        \schemestart
            \chemfig{R-[:30](=[2]O)-[:-30]OH}
            \arrow{->[\ce{K2CO3}][-\ce{KHCO3}]}[,1.3]
            \chemfig{R-[:30](=[2]\textcolor{grx}{O})-[:-30]\charge{45:1pt=$\ominus$}{\textcolor{grx}{O}}-[,0.6,,,white]\charge{45:1pt=$\oplus$}{K}}
            \arrow{->[\ce{R$'$I}]}
            \chemfig{R-[:30](=[2]\textcolor{grx}{O})-[:-30]\textcolor{grx}{O}-[:30]R'}
        \schemestop
    \end{center}
    \begin{itemize}
        \item We deprotonate the carboxylic acid using a relatively weak base.
        \begin{itemize}
            \item \ce{K2CO3} is often the weak base of choice because it's insoluble in most solvents but will react in a biphasic mixture.
            \item Additionally, since \ce{KHCO3} is usually insoluble and the carboxylate is typically soluble in the organic solvent in which the reaction is being carried out, it's really easy to separate the two.
        \end{itemize}
        \item The second step proceeds via an S\textsubscript{N}2 mechanism, so methyl or primary alkyl halides are best.
        \item Note that the two initial oxygens (green) proceed through the whole of the process and end up in the product.
    \end{itemize}
    \item Fischer esterification.
    \item General form.
    \begin{center}
        \footnotesize
        \setchemfig{atom sep=1.4em}
        \schemestart
            \chemfig{R-[:30](=[2]\textcolor{grx}{O})-[:-30]\textcolor{grx}{O}H}
            \arrow{->[\ce{H+}][\ce{R$'${\color{blx}O}H}]}
            \chemfig{R-[:30](=[2]\textcolor{grx}{O})-[:-30]\textcolor{blx}{O}-[:30]R'}
        \schemestop
    \end{center}
    \begin{itemize}
        \item The acid is a catalyst, and we need an excess of the alcohol, which we typically just use as our solvent.
        \item Reasons we need an excess of the alcohol.
        \begin{itemize}
            \item This is essentially a thermoneutral reaction; there's not a great thermodynamic driving force between the carboxylic acid and ester.
            \item Thus, the only way to get the reaction to go forward is to overwhelm it with an excess of the alcohol so that Le Ch\^{a}telier's principle comes into play.
        \end{itemize}
        \item Removing water can also help drive the reaction.
        \item \ce{H3O+} (i.e., excess water) reverses the reaction.
        \item Note that the mechanism here is a nucleophilic attack, and it is the \emph{methanol} oxygen (blue) that gets incorporated into the final ester (whose initial oxygens are colored green).
    \end{itemize}
    \item \textbf{Saponification}: Subjecting an ester to a single equivalent of \ce{KOH} (or any other hydroxide base) to form the carboxylate and the alcohol.
    \begin{itemize}
        \item This is very old chemistry.
        \item Sapon- is the Latin prefix for soap.
        \item Ancient peoples discovered that combining and heating animal fat, wood ash, and a bit of water creates soap.
        \item Combining triglycerides with pot ash yields glycerol soap and long-chain fatty acid carboxylates.
        \begin{itemize}
            \item Pot ash is where we get the name for potassium, because the ashes from a wood stove are rich in potassium hydroxide.
            \item Fatty acid carboxylates serve to solublize grease in water because the lipid end interacts with the grease and the carboxylate end interacts with the water. This is how all soaps work!
        \end{itemize}
    \end{itemize}
    \item General form.
    \begin{equation*}
        \ce{RCOOR$'$ ->[KOH] RCOOK + R$'$OH}
    \end{equation*}
    \begin{itemize}
        \item The carboxylate is an end-stage product. Resonance delocalizes the negative charge over the carbon atom, significantly decreasing its electrophilicity and hence its capacity to participate in future reactions.
        \item The presence of basic conditions make it so that this reaction is not reversible. Indeed, if we mix a base with \ce{RCOOH}, we will just deprotonate the acid and return to the carboxylate form.
    \end{itemize}
    \item Mechanism.
    \begin{itemize}
        \item Hydroxide attacks the ester as a nucleophile, and \ce{OR-} leaves to form a carboxylic acid. But \ce{OR-} (a strong base) will then deprotonate \ce{RCOOH} (a strong acid) to form the carboxylate and alcohol.
    \end{itemize}
    \item Acid chloride synthesis.
    \item General form.
    \begin{equation*}
        \ce{RCOOH ->[SOCl2][Py] RCOCl + [PyH]Cl + SO2}
    \end{equation*}
    \begin{itemize}
        \item Pyridine is not strictly necessary, but it greatly increases the reaction rate.
        \item Driven in a similar way to the dehydration of amides; we release \ce{SO2} gas, expel a water molecule, and mop up the extra \ce{Cl-} with pyridine.
    \end{itemize}
    \item Mechanism.
    \begin{figure}[h!]
        \centering
        \footnotesize
        \schemestart
            \chemfig{R-[:30](=[2]O)-[:-30]@{O1}O-[@{sb1}:30]@{H1}H}
            \arrow{->[*{0}\chemfig{@{Py2}\charge{0=\:}{Py}}][*{0}-\ce{PyH+}]}[-90]
            \chemfig{R-[:30](=[2]O)-[:-30]@{O3}\charge{90=\:,45:1pt=$\ominus$}{O}}
            \arrow{->[\chemfig[atom sep=1.4em]{@{S4}S(=[2]O)(-[@{sb4}:-30]@{Cl4}Cl)(-[:-150]Cl)}]}[,1.5]
            \chemfig{R-[:30]@{C5}(=[@{db5}2]@{O5}O)-[:-30]O-[:30]S(=[2]O)-[:-30]Cl}
            \arrow{0}[,0.1]\+
            \chemfig{@{Cl6}\charge{90=\:,45:1pt=$\ominus$}{Cl}}
            \arrow
            \chemfig{R-[:30](-[@{sb7a}:110]@{O7}\charge{180=\:,135:1pt=$\ominus$}{O})(-[:70]Cl)-[@{sb7b}:-30]O-[@{sb7c}:30]S(=[2]O)-[@{sb7d}:-30]@{Cl7}Cl}
            \arrow{->[][-\ce{SO2, Cl-}]}[,1.3]
            \chemfig{R-[:30](=[2]O)-[:-30]Cl}
        \schemestop
        \chemmove{
            \draw [curved arrow={6pt}{2pt}] (Py2) to[out=0,in=-90] (H1);
            \draw [curved arrow={2pt}{2pt}] (sb1) to[bend right=90,looseness=3] (O1);
            \draw [curved arrow={6pt}{2pt}] (O3) to[out=90,in=150,looseness=1.5] (S4);
            \draw [curved arrow={2pt}{2pt}] (sb4) to[bend left=90,looseness=3] (Cl4);
            \draw [curved arrow={6pt}{3pt}] (Cl6) to[out=100,in=50,out looseness=2] (C5);
            \draw [curved arrow={3pt}{2pt}] (db5) to[bend left=90,looseness=3] (O5);
            \draw [curved arrow={6pt}{3pt}] (O7) to[out=180,in=-150,looseness=4] (sb7a);
            \draw [curved arrow={2pt}{2pt}] (sb7b) to[bend left=60,looseness=1.5] (sb7c);
            \draw [curved arrow={2pt}{2pt}] (sb7d) to[bend left=90,looseness=3] (Cl7);
        }
        \caption{Acid chloride synthesis mechanism.}
        \label{fig:mechanismAcidChloride}
    \end{figure}
    \begin{itemize}
        \item Since chloride is a fairly week nucleophile, it's addition in step 3 takes a while and is reversible.
        \begin{itemize}
            \item However, this step is driven in the forward direction by releasing \ce{SO2} gas from the resulting tetrahedral intermediate (Le Ch\^{a}telier's principle).
        \end{itemize}
    \end{itemize}
    \item Anhydride synthesis.
    \item General form (standard).
    \begin{equation*}
        \ce{2RCOOH ->[\Delta][{[-H2O]}] RCOOCOR}
    \end{equation*}
    \begin{itemize}
        \item High heat is required.
        \item If you use two different carboxylic acids, you will get a statistical mixture (no real selectivity).
    \end{itemize}
    \item You can selectively create 5-6 membered rings containing anhydrides because this reaction proceeds intramolecularly as well as intramolecularly.
    \item General form (intramolecular).
    \begin{center}
        \footnotesize
        \setchemfig{atom sep=1.4em}
        \schemestart
            \chemfig{[4]*6(OH-(=O)--(-[:-170])(-[:-130])-(=O)-OH)}
            \arrow{->[$\Delta$][$[-\ce{H2O}]$]}[,1.2]
            \chemfig{*5((-[:-164])(-[:-124])-(=O)-O-(=O)--)}
        \schemestop
    \end{center}
    \begin{itemize}
        \item In particular, if you have a single molecule with two different carboxylic acid groups 2-3 carbons apart, then heating a sample of said molecule while removing water will result in a ring-closing anhydridization.
        \item If we want to make a ring with another number of carbons, we should go through acid chlorides (see below).
    \end{itemize}
    \item A way to selectively create anhydrides is via acid chlorides and sodium carboxylates.
    \item Mixed anhydride synthesis.
    \item General form.
    \begin{center}
        \footnotesize
        \setchemfig{atom sep=1.4em}
        \schemestart
            \chemfig{R-[:30](=[2]O)-[:-30]Cl}
            \arrow{0}[,0.1]\+{,,1.5em}
            \chemfig{R'-[:30](=[2]O)-[:-30]\charge{45:1pt=$\ominus$}{O}-[,0.6,,,white]\charge{45:1pt=$\oplus$}{Na}}
            \arrow
            \chemfig{R-[:30](=[2]O)-[:-30]O-[:30](=[2]O)-[:-30]R'}
            \arrow{0}[,0.1]\+
            \chemfig{NaCl}
        \schemestop
    \end{center}
    \begin{itemize}
        \item This reaction proceeds via nucleophilic substitution.
    \end{itemize}
    \item Amide synthesis.
    \item General form.
    \begin{equation*}
        \ce{RCOOH + NHR$'$R$''$ ->[DCC][Py] RCONR$'$R$''$}
    \end{equation*}
    \item Mechanism.
    \begin{figure}[h!]
        \centering
        \vspace{1em}
        \footnotesize
        \schemestart
            \chemfig{R-[:30](=[2]O)-[:-30]@{O1}O-[@{sb1}:30]@{H1}H}
            \arrow{->[\chemfig{@{Py2}\charge{90=\:}{Py}}]}
            \chemfig{R-[:30](=[2]O)-[:-30]@{O3}\charge{90=\:,45:1pt=$\ominus$}{O}}
            \arrow{0}[,0.1]\+
            \chemfig{\charge{90:3pt=$\oplus$}{Py}H}
            \arrow{->[\chemfig[atom sep=1.4em]{CyN=@{C5}C=[@{db5}]@{N5}NCy}]}[,2]
            \subscheme{
                \chemfig{R-[:30](=[2]O)-[:-30]O-[:30](-[2]@{N6}\charge{90=\:,135:3pt=$\ominus$}{N}Cy)(=[:-30]NCy)}
                \arrow{0}[,0.1]\+
                \chemfig{@{H7}H-[@{sb7}]@{Py7}\charge{90:3pt=$\oplus$}{Py}}
            }
            \arrow{->[][*{0}-\ce{Py}]}[-90]
            \chemfig{R-[:30]@{C8}(=[@{db8}2]@{O8}O)-[:-30]O-[:30](-[2]NHCy)(=[:-30]NCy)}
            \arrow{->[*{0.-90}\setchemfig{atom sep=1.4em}\chemfig{@{N9}\charge{90=\:}{N}HR'R''}]}[180,1.3]
            \chemfig[atom sep=2.5em]{[:120]*6(@{N10a}\charge{[extra sep=1.5pt]-135=\:}{N}Cy=(-[,0.8]NHCy)-O-(-[:130,0.8]\charge{135:1pt=$\ominus$}{O})(-[:170,0.8]R)-@{N10b}\charge{45:1pt=$\oplus$}{N}(-[:-170,0.8]R')(-[:-130,0.8]R'')-[@{sb10}]@{H10}H)}
            \arrow[180]
            \chemfig[atom sep=2.5em]{[:120]*6(@{N11}\charge{-90:3pt=$\oplus$}{N}HCy=[@{db11}](-[,0.8]NHCy)-[@{sb11a}]O-[@{sb11b}](-[@{sb11c}:130,0.8]@{O11}\charge{90=\:,135:1pt=$\ominus$}{O})(-[:170,0.8]R)-N(-[,0.8]R'')-R')}
            \arrow{->[][*{0}-\ce{DCU}]}[-90]
            \chemfig{R-[:30](=[2]O)-[:-30]N(-[6]R'')-[:30]R'}
        \schemestop
        \chemmove{
            \draw [curved arrow={6pt}{2pt}] (Py2) to[out=90,in=90,looseness=2] (H1);
            \draw [curved arrow={2pt}{2pt}] (sb1) to[bend right=90,looseness=3] (O1);
            \draw [curved arrow={6pt}{2pt}] (O3) to[out=90,in=90,looseness=1.3] (C5);
            \draw [curved arrow={3pt}{2pt}] (db5) to[bend left=90,looseness=4] (N5);
            \draw [curved arrow={6pt}{2pt}] (N6) to[out=90,in=90,looseness=1.3] (H7);
            \draw [curved arrow={2pt}{2pt}] (sb7) to[bend right=90,looseness=3] (Py7);
            \draw [curved arrow={6pt}{3pt}] (N9) to[out=90,in=150] (C8);
            \draw [curved arrow={3pt}{2pt}] (db8) to[bend right=90,looseness=3] (O8);
            \draw [curved arrow={6pt}{2pt}] (N10a) to[bend left=20,looseness=1] (H10);
            \draw [curved arrow={2pt}{2pt}] (sb10) to[bend left=70,looseness=2.5] (N10b);
            \draw [curved arrow={6pt}{2pt}] (O11) to[out=90,in=40,looseness=4] (sb11c);
            \draw [curved arrow={2pt}{2pt}] (sb11b) to[bend right=60,looseness=1.3] (sb11a);
            \draw [curved arrow={3pt}{2pt}] (db11) to[bend right=90,looseness=3] (N11);
        }
        \caption{Amide synthesis mechanism.}
        \label{fig:mechanismAmide}
    \end{figure}
    \begin{itemize}
        \item Note that as in other mechanisms, DCC eventually transforms into a type of leaving group.
        \item Normally, we use external reagents for proton transfers because doing an internal one would in most cases involve a transition state with a 4-membered ring, which is highly strained.
        \begin{itemize}
            \item However, in step 5 here, we can do an internal proton transfer because the transition state's conformation is that of a 6-membered ring.
        \end{itemize}
    \end{itemize}
    \item \textbf{DCC}: Dicyclohexylcarbodiimide, a dehydrating reagent key to amide synthesis. \emph{Structure}
    \begin{figure}[H]
        \centering
        \footnotesize
        \chemfig{C(=[:30]N-[::-60]*6(------))(=[:-150]N-[::-60]*6(------))}
        \caption{Dicyclohexylcarbodiimide (DCC).}
        \label{fig:DCC}
    \end{figure}
    \item DCC reacts with water as follows.
    \begin{figure}[H]
        \centering
        \footnotesize
        \schemestart
            \chemfig{C(=[:30]N-[::-60]*6(------))(=[:-150]N-[::-60]*6(------))}
            \arrow{->[\ce{H2O}]}
            \chemname{\chemfig{(-[:-30]N(-[6]H)-[:30]*6(------))(-[:-150]N(-[6]H)-[:150]*6(------))(=[2]O)}}{Dicyclohexylurea}
        \schemestop
        \caption{DCC and water.}
        \label{fig:DCCH2O}
    \end{figure}
    \item \textbf{DCU}: Dicyclohexylurea, the product of the reaction of DCC and water.
    \item Reactivity scale.
    \begin{equation*}
        \text{acid chloride} > \text{anhydride}
        > \text{ester}
        > \text{amide}
        > \text{carboxylate}
    \end{equation*}
    \begin{itemize}
        \item It should make intuitive sense that acid chlorides are the most reactive carboxylic acid derivatives and carboxylates are the least.
        \begin{itemize}
            \item Acid chlorides have an electronegative group on the already electrophilic carbon, exacerbating the molecular dipole.
            \item Carboxylates delocalize their negative charge over the carbon (as discussed earlier), greatly reducing or eliminating the molecular dipole.
            \item A good rule of thumb is that the compound with the best leaving group and worst nucleophile (an acid chloride) is the most reactive, and vice versa in that the compound with the worst leaving group and the best nucleophile (a carboxylate) is the most reactive.
        \end{itemize}
        \item What we mean by "reactivity" is that compounds higher on the reactive scale can react with an appropriate nucleophile to become compounds lower on the scale.
        \begin{itemize}
            \item For instance, we can take an acid chloride to an anhydride, ester, amide, or carboxylate (and we have reactions to do that), but we cannot take all (or any) of these molecules back to an acid chloride without forcing conditions.
            \item Some things that qualify as forcing conditions are the use of acidic conditions and dehydrating reagents.
            \item In other words, this reactivity scale is for the compounds in basic media with no dehydrating reagents present.
        \end{itemize}
    \end{itemize}
    \item MCAT comments.
    \item Trialkyl amines and pyridines.
    \begin{itemize}
        \item According to our reactivity scale, we should be able to react \ce{NEt3} with \ce{RCOCl} to yield an amine, for example.
        \begin{itemize}
            \item However, this leads to a positively charged nitrogen in the amine that cannot be quenched (e.g., by deprotonation). Thus, this is a highly reversible reaction that favors the reactants.
        \end{itemize}
        \item Similarly, we should be able to react an anhydride with pyridine.
        \begin{itemize}
            \item But since pyridine cannot be deprotonated either, the reactants are favored in this reversible reaction once again.
        \end{itemize}
    \end{itemize}
    \item However, this implies that pyridines can be used to catalyze nucleophilic acyl substitutions.
    \item \textbf{DMAP}: Dimethylaminopyridine, which is one of the best catalysts for nucleophilic acyl substitutions. \emph{Structure}
    \begin{figure}[h!]
        \centering
        \footnotesize
        \chemfig{[:30]**6(--(-N(-[:60])(-[:-60]))---N-)}
        \caption{Dimethylaminopyridine (DMAP).}
        \label{fig:DMAP}
    \end{figure}
    \begin{itemize}
        \item Levin gives an example synthesis using DMAP, namely nucleophilic addition to an anhydride.
        \begin{itemize}
            \item In essence, DMAP adds to the carbonyl, kicks out the leaving group, and then the nucleophile adds to the carbonyl and kicks out DMAP.
        \end{itemize}
        \item Adding DMAP can accelerate a reaction that would take overnight to taking only a few minutes.
    \end{itemize}
    \item Acid chlorides, anhydrides, and esters all create the same product (an amide) when reacting with an amine.
    \begin{itemize}
        \item But, you need only one equivalent of the amine for esters while you need two equivalents for the first two.
        \item This is because of the $\pKa$'s.
        \begin{itemize}
            \item In order of increasing $\pKa$, we have $\ce{HCl}<\ce{RCOOH}<\ce{NR2H2+}<\ce{ROH}$.
            \item Thus, the first two byproducts (\ce{HCl} and \ce{RCOOH}) protonate amines in solution, whereas \ce{ROH} does not.
        \end{itemize}
    \end{itemize}
\end{itemize}



\section{Discussion Section}
\begin{itemize}
    \item \marginnote{4/8:}We will be working with hot sand baths in the next lab, so just leave them to cool and do not dispose of the contents unless you're sure they're cool.
    \item Practice problems.
    \begin{enumerate}
        \item ${\color{white}hi}$
        \begin{center}
            \footnotesize
            \setchemfig{atom sep=1.4em}
            \schemestart
                \chemfig{*4(--(-(=[::60]O)-[::-60]OEt)--)}
                \arrow{->[\ce{NaOH}]}[,1.1]
                \color{rex}
                \chemfig{*4(--(-(=[::60]O)-[::-60]\charge{45:1pt=$\ominus$}{O})--)}
            \schemestop
        \end{center}
        \begin{itemize}
            \item We form a \ce{COO-} ion instead of the carboxylic acid because we are in basic solution.
            \item The mechanism is a nucleophilic attack on the carbonyl, the oxygen electrons swinging back down and kicking out \ce{EtO-}, and then deprotonation of the acid.
        \end{itemize}
        \item ${\color{white}hi}$
        \begin{center}
            \footnotesize
            \setchemfig{atom sep=1.4em}
            \schemestart
                \chemfig{-[:-30]-[:30](=[2]O)-[:-30]H}
                \arrow{->[\color{rex}1. \ce{CrO3, H2SO4, H2O}][\color{rex}2. \ce{NH3, DCC}\rule{1.1cm}{0pt}]}[,2.5]
                \chemfig{-[:-30]-[:30](=[2]O)-[:-30]NH_2}
            \schemestop
        \end{center}
        \begin{itemize}
            \item The intermediate after step 1 is the carboxylic acid, as we have used aqueous Jones reagent.
        \end{itemize}
        \item ${\color{white}hi}$
        \begin{center}
            \footnotesize
            \setchemfig{atom sep=1.4em}
            \schemestart
                \chemfig{EtO-[:30](=[2]O)-[:-30]-[:30](=[2]O)-[:-30]OH}
                \arrow{->[\ce{MeOH}][\ce{H2SO4}]}[,1.2]
                \color{rex}
                \chemfig{MeO-[:30](-[2]OH)-[:-30]-[:30](=[2]O)-[:-30]OMe}
            \schemestop
        \end{center}
        \begin{itemize}
            \item The reaction of the ester (left) is called \textbf{transesterification}; the reaction of the carboxylic acid (right) is called ether formation.
            \item It's important to know that you can get ester formation in both of these cases.
            \item This is a common problematic side reaction in synthetic chemistry.
            \item Mechanism: Methanol attacks each carbonyl, the other group leaves, and then deprotonation.
        \end{itemize}
        \item ${\color{white}hi}$
        \begin{center}
            \footnotesize
            \setchemfig{atom sep=1.4em}
            \schemestart
                \chemfig{-[:-30](-[6])-[:30](=[2]O)-[:-30]}
                \arrow{0}[,0.1]\+{,,1em}
                \chemfig{[:18]*5(---\chemabove{N}{H}--)}
                \arrow{->[\ce{H3O+}]}
                \color{rex}
                \chemfig{-[:-30](-[6])=_[:30](-[2]N*5(-----))-[:-30]}
            \schemestop
        \end{center}
        \begin{itemize}
            \item We choose this enamine as the major product by Zaitsev's rule.
        \end{itemize}
        \item ${\color{white}hi}$
        \begin{center}
            \footnotesize
            \setchemfig{atom sep=1.4em}
            \schemestart
                \chemfig{H-[:30](=[2]O)-[:-30]-[:30]-[:-30](=[6]O)-[:30]}
                \arrow{->[\chemfig[atom sep=1.4em]{HO-[:30]-[:-30]-[:30]OH}][\ce{H3O+}]}[,1.8]
                \color{rex}
                \chemfig{H-[:30](-[2,0.01]*5(-O---O-))-[:-30]-[:30]-[:-30](=[6]O)-[:30]}
                \color{black}
                \arrow{->[1. \ce{MeLi}\rule{1.5mm}{0pt}][2. \ce{H3O+}]}[,1.3]
                \color{rex}
                \chemfig{H-[:30](=[2]O)-[:-30]-[:30]-[:-30](-[:-70]OH)(-[:-110])-[:30]}
            \schemestop
        \end{center}
        \item ${\color{white}hi}$
        \begin{center}
            \footnotesize
            \setchemfig{atom sep=1.4em}
            \schemestart
                \chemfig{-[:30]*6(----(-OH)-(-)-)}
                \arrow(R--P1){->[1. PCC\rule{8.5mm}{0pt}][2. \chemfig[atom sep=1.4em]{-[:30]=_[:-30]PPh_3}]}[,1.7]
                \color{rex}
                \chemfig{-[:30]*6(----(=_-[:30])-(-)-)}
                \color{black}
                \arrow(@R--P2){->[*{0.-90}\ce{H2SO4}]}[180,1.2]
                \color{rex}
                \chemfig{-[:30]*6(-----(-)=)}
            \schemestop
        \end{center}
        \item ${\color{white}hi}$
        \begin{center}
            \footnotesize
            \setchemfig{atom sep=1.4em}
            \schemestart
                \chemfig{-[:-30](=[6])-[:30](=[2]O)-[:-30]}
                \arrow{->[\ce{Me2CuLi}]}[,1.3]
                \color{rex}
                \chemfig{-[:-30](-[6]-[:-150])-[:30](=[2]O)-[:-30]}
                \color{black}
                \arrow{->[1. \ce{MeLi}\rule{1.5mm}{0pt}][2. \ce{H3O+}]}[,1.3]
                \color{rex}
                \chemfig{-[:-30](-[6]-[:-150])-[:30](-[:110])(-[:70]OH)-[:-30]}
            \schemestop
        \end{center}
    \end{enumerate}
\end{itemize}



\section{Office Hours (Levin)}
\begin{itemize}
    \item \marginnote{4/11:}$\alpha,\beta$-unsaturated carbonyls?
    \begin{figure}[h!]
        \centering
        \footnotesize
        \begin{subfigure}[b]{\linewidth}
            \centering
            \schemestart
                \chemfig{-[:30](=[@{db1a}2]O)-[@{sb1}:-30]=_[@{db1b}6]@{C1}}
                \arrow{->[\chemfig[atom sep=1.4em]{@{H2}H-[@{sb2}]@{O2}OMe}][\chemfig[atom sep=1.4em]{\charge{45:1pt=$\oplus$}{Na}-[,0.6,,,white]H-[@{sb3}]\charge{90:3pt=$\ominus$}{B}H_3-[2,0.6,,,opacity=0]}]}[,1.7]
                \chemfig{-[:30](-[2]OH)=_[:-30]-[6]}
                \arrow
                \chemfig{-[:30](=[2]O)-[:-30]-[6]}
                \arrow{->[\ce{NaBH4}][\ce{MeOH}]}[,1.2]
                \chemname{\chemfig{-[:30](-[2]OH)-[:-30]=_[6]}}{50\%}
                \arrow{0}[,0.1]\+
                \chemname{\chemfig{-[:30](-[2]OH)-[:-30]-[6]}}{50\%}
            \schemestop
            \chemmove{
                \draw [curved arrow={2pt}{2pt}] (sb3) to[out=-90,in=-30] (C1);
                \draw [curved arrow={4pt}{2pt}] (db1b) to[bend left=60,looseness=2] (sb1);
                \draw [curved arrow={3pt}{2pt}] (db1a) to[bend left=30] (H2);
                \draw [curved arrow={2pt}{2pt}] (sb2) to[bend left=90,looseness=4] (O2);
            }
            \caption{\ce{NaBH4}.}
            \label{fig:alphaBetaReductiona}
        \end{subfigure}\\[2em]
        \begin{subfigure}[b]{\linewidth}
            \centering
            \schemestart
                \chemfig{-[:30](=[@{db1a}2]@{O1}O)-[@{sb1}:-30]=_[@{db1b}6]@{C1}}
                \arrow{->[][\chemfig[atom sep=1.4em]{\charge{45:1pt=$\oplus$}{Li}-[,0.6,,,white]H-[@{sb2}]\charge{90:3pt=$\ominus$}{Al}H_3-[2,0.6,,,opacity=0]}]}[,1.7]
                \chemfig{-[:30](-[2]O-[:30]\charge{90:3pt=$\ominus$}{Al}H_3)=_[:-30]-[6]}
                \arrow{->[\ce{H3O+}]}
                \chemname{\chemfig{-[:30](-[2]OH)-[:-30]=_[6]}}{Major}
                \arrow{0}[,0.1]\+
                \chemname{\chemfig{-[:30](=[2]O)-[:-30]-[6]}}{Minor}
            \schemestop
            \chemmove{
                \draw [curved arrow={2pt}{2pt}] (sb2) to[out=-90,in=-30] (C1);
                \draw [curved arrow={4pt}{2pt}] (db1b) to[bend left=60,looseness=2] (sb1);
                \draw [curved arrow={3pt}{2pt}] (db1a) to[bend right=90,looseness=3] (O1);
            }
            \caption{\ce{LiAlH4}.}
            \label{fig:alphaBetaReductionb}
        \end{subfigure}
        \caption{Reduction of $\alpha,\beta$ unsaturated compounds.}
        \label{fig:alphaBetaReduction}
    \end{figure}
    \begin{itemize}
        \item Levin's predictions basically line up with those from \textcite{bib:CHEM22100Notes}, although he has a different way of deriving them.
        \item The 1,2-reduction product is the same in both. But for the \ce{NaBH4}, you get full reduction as the other major byproduct.
        \item We will never be asked to use this reaction synthetically because it is not selective.
        \item We're most likely to encounter alkyllithiums or cuprates. The thing to keep in mind with the messy ones is that they're messy. We're more just interested in introducing enolate chemistry with these.
    \end{itemize}
    \item Problem Set 1, Question 3a: We form one bond with the best stereochemistry and then do an S\textsubscript{N}2 to simultaneously form the epoxide and kick out \ce{SPh2}.
    \item Problem Set 1, Question 2f: Cyclic systems are one of the only places you see hemi-acetals.
    \item Problem Set 1, Question 3b: The transition state has too much ring strain, so show proton transfers as being mediated by solvent molecules.
    \item n-butyl lithium stands for "normal"-butyl lithium; s-butyl lithium is sec-butyl lithium.
\end{itemize}



\section{Carboxylic Acids and Derivatives 2}
\begin{itemize}
    \item \marginnote{4/12:}Last time:
    \begin{itemize}
        \item We discussed the reactivity of compounds of the form \ce{RCOOXR$'$} where \ce{X} is a heteroatom.
        \item We looked at nucleophilic addition to such compounds under acidic and basic conditions, which more often than not proceeds through a nucleophilic acyl substitution mechanism.
        \item Certain classes can be taken to others by the addition of a nucleophile.
        \item Reviews adding amines to acid chlorides, anhydrides, and esters, and the amount of amine needed for each.
    \end{itemize}
    \item Today: How carboxylic acid derivatives interact with hydrides and carbides.
    \begin{itemize}
        \item Most of the early lecture content is straight outta CHEM 221. Highlights will follow.
    \end{itemize}
    \item Carbide addition to\dots
    \begin{enumerate}
        \item Ketones and aldehydes.
        \item Carboxylic acids.
        \item Esters.
    \end{enumerate}
    \item Ketones and aldehydes.
    \begin{equation*}
        \ce{RCOR$'$ ->[1. R$''$Li][2. H3O+] CRR$'$R$''$(OH)}
    \end{equation*}
    \begin{itemize}
        \item We can use lithiates or Grignards.
    \end{itemize}
    \item Carboxylic acids.
    \begin{equation*}
        \ce{RCOOH ->[R$'$Li] RCOOLi + R$'$H}
    \end{equation*}
    \begin{itemize}
        \item We protonate the lithiate, yielding a carboxylate with a lithium countercation and an aliphatic species.
    \end{itemize}
    \item Esters.
    \begin{equation*}
        \ce{RCOOR$'$ ->[1. R$''$Li][2. H3O+] CR(R$''$)2(OH)}
    \end{equation*}
    \begin{itemize}
        \item Two equivalents of the lithiate add in, the \ce{OR$'$} group leaves, and the alcohol is reduced.
        \item See Figure 9.2 of \textcite{bib:CHEM22100Notes} for the mechanism.
        \item The fact that we observe double addition means that the \textbf{overaddition product} is the major product. 
        \item If you only add one equivalent of lithiate, the major products will be the overaddition product and unreacted ester; the ketone will only be a very minor product.
        \begin{itemize}
            \item This is because esters are less electrophilic due to donation from the ether oxygen, so the lithiate will selectively go for the ketone as soon as it becomes available.
            \item Ester resonance essentially partially protects it from nucleophilic addition.
        \end{itemize}
    \end{itemize}
    \item \textbf{Overaddition product}: A nucleophilic addition product in which the nucleophile adds more than once.
    \begin{itemize}
        \item So named because we typically only want monoaddition.
    \end{itemize}
    \item Hydride addition to\dots
    \begin{enumerate}
        \item Esters (\ce{NaBH4}, \ce{LiAlH4}, and DIBAL-H).
        \item Amides (\ce{LiAlH4} and DIBAL-H).
    \end{enumerate}
    \item Esters (\ce{NaBH4}).
    \begin{itemize}
        \item \ce{NaBH4 + MeOH} does not react with esters (for the purposes of this class).
    \end{itemize}
    \item Esters (\ce{LiAlH4}).
    \begin{equation*}
        \ce{RCOOR$'$ ->[1. LiAlH4][2. H3O+] RCH2OH + R$'$OH}
    \end{equation*}
    \begin{itemize}
        \item See Figure 9.2 of \textcite{bib:CHEM22100Notes} for the mechanism.
        \item Mechanistically, the aldehyde intermediate is much more reactive than the ester, once again.
        \item Is it the lithium cation that bonds to the alkoxide or the \ce{AlH3} species?
    \end{itemize}
    \item Selecting for addition to the ester instead of addition to the aldehyde intermediate.
    \begin{itemize}
        \item We are going to change the structure of our reducing agent.
        \item We want to continue using aluminum since \ce{NaBH4} is not strong enough, but we can play with the ligands.
        \item Thus, we change from the tetracoordinate \ce{AlH4-} to \textbf{DIBAL-H}.
    \end{itemize}
    \item \textbf{DIBAL-H}: Diisobutylaluminum hydride, a neutral, tricoordinate aluminum species with an empty $p$ orbital that is useful for selecting the mono-hydride addition product in cases where overaddition is common. \emph{Also known as} \textbf{DIBAL}. \emph{Structure}
    \begin{figure}[h!]
        \centering
        \footnotesize
        \chemfig{Al(-[:60]-(-[:60])(-[:-60]))(-[:-60]-(-[:60])(-[:-60]))(-[4]H)}
        \caption{Diisobutylaluminum hydride (DIBAL-H).}
        \label{fig:DIBAL}
    \end{figure}
    \item Esters (DIBAL-H).
    \item General form.
    \begin{equation*}
        \ce{RCOOR$'$ ->[1. DIBAL-H][2. H3O+] RCOH + R$'$OH}
    \end{equation*}
    \item Mechanism.
    \begin{figure}[h!]
        \centering
        \footnotesize
        \schemestart
            \chemfig{R-[:30](=[2]@{O1}\charge{90=\:}{O})-[:-30]OR'}
            \arrow{->[\chemfig{H@{Al2}AlBu^\emph{i}{}_2}]}[,1.2]
            \chemfig{R-[:30]@{C3}(=[@{db3}2]@{O3}\charge{135:1pt=$\oplus$}{O}-[:30]\charge{90:3pt=$\ominus$}{Al}Bu^\emph{i}{}_2-[@{sb3}:-60,,1]H)-[:-30]OR'}
            \arrow
            \chemfig{R-[:30](-[2]O-[:30]@{Al4}AlBu^\emph{i}{}_2)-[:-30]@{O4}\charge{90=\:}{O}R'}
            \arrow
            \chemname[-3em]{\chemfig{R-[:30](-[2]O-[@{sb5a}:5,1.1]\charge{90:3pt=$\ominus$}{Al}Bu^\emph{i}{}_2(-[@{sb5b}:-106]))-[:-10]@{O5}\charge{-90:3pt=$\oplus$}{O}R'}}{Chelate}
            \arrow{->[*{0} {\chemfig[atom sep=1.4em]{@{H6}H-[@{sb6}]@{O6}\charge{90:3pt=$\oplus$}{O}H_2}}][*{0}-\ce{AlBu^{$i$}{}_2-}]}[-90]
            \chemfig{R-[:30](-[2]OH)-[:-30]@{O7}\charge{90=\:}{O}R'}
            \arrow{->[*{0.-90} {\chemfig[atom sep=1.4em]{@{H8}H-[@{sb8}]@{O8}\charge{90:3pt=$\oplus$}{O}H_2}}]}[180,1.3]
            \chemfig{R-[:30](-[@{sb9a}2]@{O9a}\charge{180=\:}{O}H)-[@{sb9b}:-30]@{O9b}\charge{90:3pt=$\oplus$}{O}HR'}
            \arrow{->[][*{0.90}-\ce{R$'$OH}]}[180,1.1]
            \chemfig{R-[:30](=[2]@{O10}\charge{135:1pt=$\oplus$}{O}-[@{sb10}:30]@{H10}H)-[:-30]H}
            \arrow{->[*{0.-90}\chemfig{H_2@{O11}\charge{90=\:}{O}}]}[180]
            \chemfig{R-[:30](=[2]O)-[:-30]H}
        \schemestop
        \chemmove{
            \draw [curved arrow={6pt}{2pt}] (O1) to[out=90,in=90,looseness=1.5] (Al2);
            \draw [curved arrow={2pt}{3pt}] (sb3) to[out=-150,in=30] (C3);
            \draw [curved arrow={3pt}{2pt}] (db3) to[bend left=90,looseness=3] (O3);
            \filldraw [-,thick,draw=orx,fill=ory]
                ([yshift=1mm]Al4.89) to[bend right=110,looseness=600] ([yshift=1mm]Al4.91)
                ([yshift=-1mm]Al4.-89) to[bend left=110,looseness=600] ([yshift=-1mm]Al4.-91)
            ;
            \draw [curved arrow={6pt}{6pt}] (O4) to[out=90,in=-120,in looseness=1.5] ([yshift=-5mm]Al4.south);
            \draw [curved arrow={2pt}{2pt}] (sb5a) to[out=95,in=90,looseness=1.5] ++(1.5,0) to[out=-90,in=65] (H6);
            \draw [curved arrow={2pt}{2pt}] (sb6) to[bend right=90,looseness=4] (O6);
            \draw [curved arrow={2pt}{2pt}] (sb5b) to[bend right=70,looseness=2.5] (O5);
            \draw [curved arrow={6pt}{2pt}] (O7) to[out=90,in=-25,looseness=1] ++(-0.4,1.3) to[out=155,in=90] (H8);
            \draw [curved arrow={2pt}{2pt}] (sb8) to[out=110,in=130,looseness=3] (O8);
            \draw [curved arrow={6pt}{2pt}] (O9a) to[bend right=90,looseness=3] (sb9a);
            \draw [curved arrow={2pt}{2pt}] (sb9b) to[bend right=90,looseness=3] (O9b);
            \draw [curved arrow={6pt}{2pt}] (O11) to[out=90,in=90,looseness=1.3] (H10);
            \draw [curved arrow={2pt}{2pt}] (sb10) to[bend left=70,looseness=2.5] (O10);
        }
        \caption{Monoreducton of esters mechanism.}
        \label{fig:mechanismEsterDIBAL}
    \end{figure}
    \begin{itemize}
        \item We might commonly expect to see the second intermediate (the zwitterion) decompose back into the initial reactants. However, it reacts to form a charge-neutral species that will not dissociate, as doing so would create an aluminum cation (highly unstable) in addition to the alkoxide.
        \item Aluminum's empty $p$ orbital plays a key role in the third step as a Lewis acid/electron acceptor for the electrons of the ether oxygen.
        \item The chelate is extra stable.
        \begin{itemize}
            \item Even though there are only four atoms in its ring (as opposed to five or six), aluminum is a \emph{third}-row main group element, meaning that it forms longer, more flexible bonds. Thus, aluminum-containing rings can tolerate smaller number of atoms than normal organic ring systems.
            \item The implication is that it will not break down to kick out the alkoxide \ce{OR$'$-}. This stability is what most directly favors the monoaddition product.
        \end{itemize}
        \item The last several steps (after the addition of the acid) constitute the decomposition of a hemiacetal under acidic conditions.
        \item In practice, this reaction is really difficult to pull off.
        \begin{itemize}
            \item The chelate is only stable at \SI{-78}{\celsius}. If it warms up much beyond that, it will decompose into the aldehyde.
            \item The reaction of DIBAL-H with the ester is exothermic, so you have to keep it really cold and do the addition really slowly. Otherwise, the internal exotherm will raise the temperature and ruin the reaction.
            \item Thus, you will often see in the literature chemists circumventing this reaction via a reduction (\ce{LiAlH4 + H3O+}) followed by PCC/Swern.
            \item However, for the purposes of this class, we can treat the DIBAL-H method as if it works perfectly in every case, i.e., as if we're just laying out a synthetic plan and the person performing the reactions will do everything perfectly. In other words, we should definitely feel free to use this method (as written from a na\"{i}ve perspective) in any synthesis questions we encounter.
        \end{itemize}
    \end{itemize}
    \item Amides (\ce{LiAlH4}).
    \item General form.
    \begin{equation*}
        \ce{RCONR$'$R$''$ ->[LiAlH4] RCH2NR$'$R$''$}
    \end{equation*}
    \begin{itemize}
        \item We don't \emph{need} an aqueous workup, but it's often performed anyway to remove excess alumina.
    \end{itemize}
    \item Mechanism.
    \begin{figure}[H]
        \centering
        \footnotesize
        \schemestart
            \chemfig{R-[:30]@{C1}(=[@{db1}2]O)-[:-30]NR'R''}
            \arrow{0}[,0.6]
            \chemfig{H-[@{sb2}2]@{Al2}\charge{90:3pt=$\ominus$}{Al}H_3}
            \arrow
            \chemfig{R-[:30](-[@{sb3a}2]@{O3}O-[:30]\charge{90:3pt=$\ominus$}{Al}H_3)-[@{sb3b}:-30]@{N3}\charge{90=\:}{N}R'R''}
            \arrow{->[][-\ce{AlH3O^2-}]}[,1.4]
            \chemfig{R-[:30]@{C4}=_[@{db4}:-30]@{N4}\charge{90:3pt=$\oplus$}{N}R'R''}
            \arrow{->[\chemfig[atom sep=1.4em]{H-[@{sb5}]\charge{90:3pt=$\ominus$}{Al}H_3}]}[,1.3]
            \chemfig{R-[:30]-[:-30]NR'R''}
        \schemestop
        \chemmove{
            \draw [curved arrow={2pt}{3pt}] (sb2) to[out=180,in=30] (C1);
            \draw [curved arrow={3pt}{2pt}] (db1) to[bend left=20] (Al2);
            \draw [curved arrow={6pt}{2pt}] (N3) to[out=90,in=60,looseness=3] (sb3b);
            \draw [curved arrow={2pt}{2pt}] (sb3a) to[bend left=90,looseness=3] (O3);
            \draw [curved arrow={2pt}{2pt}] (sb5) to[out=90,in=90,looseness=1.3] (C4);
            \draw [curved arrow={4pt}{2pt}] (db4) to[bend right=90,looseness=3] (N4);
        }
        \caption{Reduction of amides mechanism.}
        \label{fig:mechanismAmideReduction}
    \end{figure}
    \begin{itemize}
        \item Unlike with esters, nitrogen is a stronger donor than the oxygen atom, so it will kick it out in the second step.
    \end{itemize}
    \item Amides (DIBAL-H).
    \item General form.
    \begin{equation*}
        \ce{RCONR$'$R$''$ ->[1. DIBAL-H][2. H3O+] RCOH + NHR$'$R$''$}
    \end{equation*}
    \item Mechanism.
    \begin{figure}[h!]
        \centering
        \footnotesize
        \schemestart
            \chemfig{R-[:30](=[2]O)-[:-30]NR'R''}
            \arrow{->[*{0}DIBAL-H]}[90]
            \chemfig{R-[:30](-[2]O-[:30]@{Al2}AlBu^\emph{i}{}_2)-[:-30]@{N2}\charge{90=\:}{N}R'R''}
            \arrow
            \chemfig{R-[:30](-[2]O-[:5,1.1]\charge{90:3pt=$\ominus$}{Al}Bu^\emph{i}{}_2(-[:-106]))-[:-10]\charge{-90:3pt=$\oplus$}{N}R'R''}
            \arrow{->[\ce{H3O+}][-\ce{AlH3O^2-}]}[,1.4]
            \chemfig{R-[:30]=_[:-30]\charge{90:3pt=$\oplus$}{N}R'R''}
            \arrow{->[*{0}\ce{H2O}]}[-90]
            \subscheme{
                \chemfig{R-[:30](=[2]O)-[:-30]H}
                \arrow{0}[,0.1]\+{,,2em}
                \chemfig{R'-[:30]N(-[2]H)-[:-30]R''}
            }
        \schemestop
        \chemmove{
            \draw [curved arrow={6pt}{2pt}] (N2) to[out=90,in=-90] (Al2);
        }
        \caption{Monoreducton of amides mechanism.}
        \label{fig:mechanismAmideDIBAL}
    \end{figure}
    \begin{itemize}
        \item Amides coordinate with DIBAL much more easily than esters.
        \item Note that in the last step, the acid destroys any remaining DIBAL-H and then reduces the final species.
        \begin{itemize}
            \item This likely proceeds analogously to the steps in the latter parts of Figure \ref{fig:mechanismNitrileHydrolysis}.
        \end{itemize}
    \end{itemize}
    \item Note that the role, stability, and structure of the tetrahedral intermediates are what determines the reactivity of amines with both sets of reagents.
    \item Reactions of nitriles.
    \item Nitriles (\ce{R$'$Li}).
    \item General form.
    \begin{equation*}
        \ce{RCN ->[1. R$'$Li][2. H3O+] RCOR$'$}
    \end{equation*}
    \begin{itemize}
        \item Useful for generating a ketone from a carboxylic acid derivative.
        \item No overaddition.
    \end{itemize}
    \item Mechanism.
    \begin{figure}[h!]
        \centering
        \footnotesize
        \schemestart
            \chemfig{R-@{C1}C~[@{tb1}]@{N1}N}
            \arrow{->[\chemfig[atom sep=1.4em]{R'-[@{sb2}]Li}][-\ce{Li+}]}[,1.2]
            \chemfig{R-[:30](=[2]@{N3}\charge{90=\:,45:1pt=$\ominus$}{N})-[:-30]R'}
            \arrow{->[\chemfig[atom sep=1.4em]{@{H4}H-[@{sb4}]@{O4}\charge{90:3pt=$\oplus$}{O}H_2}][-\ce{H2O}]}[,1.3]
            \chemfig{R-[:30](=[2]NH)-[:-30]R'}
            \arrow{->[\ce{H3O+}]}
            \chemfig{R-[:30](=[2]O)-[:-30]R'}
        \schemestop
        \chemmove{
            \draw [curved arrow={2pt}{2pt}] (sb2) to[out=90,in=90,looseness=1.3] (C1);
            \draw [curved arrow={4pt}{2pt}] (tb1) to[bend right=90,looseness=3] (N1);
            \draw [curved arrow={6pt}{2pt}] (N3) to[out=90,in=90,looseness=1.5] (H4);
            \draw [curved arrow={2pt}{2pt}] (sb4) to[out=120,in=130,looseness=3] (O4);
        }
        \caption{Nitrile alkylation mechanism.}
        \label{fig:mechanismNitrileAlkylation}
    \end{figure}
    \begin{itemize}
        \item Explaining the lack of overaddition.
        \begin{itemize}
            \item Unlike with esters, there is no good leaving group in the first intermediate.
            \item Indeed, adding another lithiate would kick out an \ce{N^2-} species (highly unstable), but this would never happen.
            \item Additionally, since the acid destroys the \ce{LiAlH4}, even though we end up producing a ketone (an electrophilic carbonyl), there is no further reactivity.
        \end{itemize}
        \item The last step is imine hydrolysis, which Levin mentioned in Aldehydes and Ketones 1 is reactivity to which imines are prone.
    \end{itemize}
    \item Nitriles (DIBAL-H).
    \item General form.
    \begin{equation*}
        \ce{RCN ->[1. DIBAL-H][2. H3O+] RCOH}
    \end{equation*}
    \item Mechanism.
    \begin{itemize}
        \item As in Figure \ref{fig:mechanismEsterDIBAL}, the heteroatom (nitrogen) attacks the aluminum of DIBAL-H to start. We then undergo the same proton rearrangement to get to a stable species. However, instead of forming a chelate, the acid takes us to the same imine as in Figure \ref{fig:mechanismNitrileAlkylation}, and then further to the aldehyde (also as in Figure \ref{fig:mechanismNitrileAlkylation}).
    \end{itemize}
    \item Nitriles (\ce{LiAlH4}).
    \item General form.
    \begin{equation*}
        \ce{RCN ->[1. LiAlH4][2. H3O+] CH2RNH2}
    \end{equation*}
    \item Mechanism.
    \begin{figure}[h!]
        \centering
        \footnotesize
        \schemestart
            \chemfig{R-@{C1}C~[@{tb1}]@{N1}N}
            \arrow{->[\chemfig[atom sep=1.4em]{H-[@{sb2}]\charge{90:3pt=$\ominus$}{Al}H_3}]}[,1.3]
            \chemfig{R-[:30]@{C3}(=[@{db3}2]@{N3}N-[:30]\charge{90:3pt=$\ominus$}{Al}H_3)-[:-30]R'}
            \arrow{->[\chemfig[atom sep=1.4em]{H-[@{sb4}]\charge{90:3pt=$\ominus$}{Al}H_3}]}[,1.3]
            \chemfig{R-[:30](-[2]N(-[:30]\charge{90:3pt=$\ominus$}{Al}H_3)(-[:150]H_3\charge{90:3pt=$\ominus$}{Al}))(-[:-70]H)-[:-30]H}
            \arrow{->[\ce{H3O+}]}
            \chemfig{R-[:30]-[:-30]NH_2}
        \schemestop
        \chemmove{
            \draw [curved arrow={2pt}{2pt}] (sb2) to[out=90,in=90,looseness=1.3] (C1);
            \draw [curved arrow={4pt}{2pt}] (tb1) to[bend right=90,looseness=3] (N1);
            \draw [curved arrow={2pt}{3pt}] (sb4) to[out=90,in=30,out looseness=1.2] (C3);
            \draw [curved arrow={3pt}{2pt}] (db3) to[bend left=90,looseness=3] (N3);
        }
        \caption{Nitrile reduction mechanism.}
        \label{fig:mechanismNitrileReduction}
    \end{figure}
    \begin{itemize}
        \item Why does this work here but not with \ce{R$'$Li}?
    \end{itemize}
    \item This nitrile reactivity allows two important types of transformations.
    \begin{itemize}
        \item From an alkyl halide precursor, use \ce{KCN} to take it to a nitrile, and then transform it to your carboxylic acid derivative of choice.
        \item From a ketone, use \ce{HCN} to take it to a cyanohydrin, and then move to a carboxylic acid derivative.
        \begin{itemize}
            \item Watch out for acidic protons on the alcohol here, though!
            \item Because of it, we can reduce to an amine with \ce{LiAlH4} with ease, but we have to play with the concentrations to get the others to work (for example, by using a huge excess of the reagent in comparison to a lithiate).
        \end{itemize}
    \end{itemize}
    \item Transforming carboxylates to ketones.
    \item General form.
    \begin{equation*}
        \ce{RCOOH ->[1. R$'$Li, \Delta][2. H3O+, time] CORR$'$}
    \end{equation*}
    \begin{itemize}
        \item Grignards won't work here; we do need the stronger lithiates.
        \item We need an excess of \ce{R$'$Li} and high heat ($\sim\SI{100}{\celsius}$).
    \end{itemize}
    \item Mechanism.
    \begin{figure}[H]
        \centering
        \footnotesize
        \schemestart
            \chemfig{R-[:30](=[2]O)-[:-30]OH}
            \arrow{->[\ce{R$'$Li}][-\ce{Li+, R$'$H}]}[,1.3]
            \chemfig{R-[:30](=[2]O)-[:-30]\charge{45:1pt=$\ominus$}{O}}
            \arrow{->[\ce{R$'$Li}, $\Delta$][-\ce{Li+}]}[,1.2]
            \chemname[-3em]{\chemfig{R-[:30](-[:110]\charge{135:1pt=$\ominus$}{O})(-[:70]\charge{45:1pt=$\ominus$}{O})-[:-30]R'}}{Ketone diolate}
            \arrow{->[\ce{H3O+}]}
            \chemfig{R-[:30](-[:110]HO)(-[:70]OH)-[:-30]R'}
            \arrow
            \chemfig{R-[:30](=[2]O)-[:-30]R'}
        \schemestop
        \caption{Carboxylic acid to ketone mechanism.}
        \label{fig:mechanismCarboxylicKetone}
    \end{figure}
    \begin{itemize}
        \item The excess lithiate is used to both deprotonate the carboxylic acid and alkylate the carboxylate that gets formed.
        \item The heat is used to overcome the low electrophilicity of the carboxylate.
    \end{itemize}
\end{itemize}



\section{Problem Session}
\begin{itemize}
    \item Practice problems.
    \begin{enumerate}
        \item ${\color{white}hi}$
        \begin{center}
            \footnotesize
            \setchemfig{atom sep=1.4em}
            \schemestart
                \chemfig{-[:30](=[2]O)-[:-30]-[:30](=[2]O)-[:-30]H}
                \arrow{->[\color{rex}\begin{tabular}{l}
                    1. \ce{NaBH4}, \ce{MeOH}\\
                    2. acetone, \ce{H+}, $[-\ce{H2O}]$
                \end{tabular}][\color{rex}]}[,2.6]
                \chemfig{-[:30]*6(---O-(-[:70])(-[:110])-O-)}
            \schemestop
        \end{center}
        \begin{itemize}
            \item Sulfuric acid is a dehydrating acid.
        \end{itemize}
        \item ${\color{white}hi}$
        \begin{center}
            \footnotesize
            \setchemfig{atom sep=1.4em}
            \schemestart
                \chemfig{[:18]*5(--(-~)-(=O)--)}
                \arrow{->[\color{rex}\begin{tabular}{l}
                    1. 9-BBN-H\\
                    2. \ce{H2O2, OH-}
                \end{tabular}]}[,1.8]
                \chemfig{[:18]*5(--(--[:-30](=[6]O)-[:30]H)-(=O)--)}
            \schemestop
        \end{center}
        \item ${\color{white}hi}$
        \begin{center}
            \footnotesize
            \setchemfig{atom sep=1.4em}
            \schemestart
                \chemfig{**6(--(*5(-(=O)-O-(=O)-))---(-)-)}
                \arrow{->[\begin{tabular}{l}
                    1. \ce{NaOMe + MeOH}\\
                    2. \ce{H3O+}
                \end{tabular}]}[,2.2]
                \color{rex}
                \chemfig{**6(--(*6(-(=O)-OH-[,,,,white]OMe-(=O)-))---(-)-)}
                \arrow{0}[,0.1]\+
                \color{rex}
                \chemfig{**6(--(*6(-(=O)-OMe-[,,,,white]OH-(=O)-))---(-)-)}
            \schemestop
        \end{center}
        \begin{itemize}
            \item Notice that this is an asymmetric anhydride, so there are multiple possible products.
            \item Regioselectivity goes out the window a bit due to the high temperatures, so don't worry about major and minor products
        \end{itemize}
        \item ${\color{white}hi}$
        \begin{center}
            \footnotesize
            \setchemfig{atom sep=1.4em}
            \schemestart
                \chemfig{-[:-30](-[6])-[:30](=[2]O)-[:-30]OH}
                \arrow{0}[,0.1]\+
                \chemfig{HO-[:30](=[2]O)-[:-30]}
                \arrow{->[$\Delta$]}
                \color{rex}
                \text{products}
            \schemestop
        \end{center}
        \begin{itemize}
            \item The products are a whole variety of coupled anhydrides.
            \item We can do this selectively by transforming one of the carboxylic acids into an acid chloride with \ce{SOCl2}.
            \begin{itemize}
                \item Note that we don't \emph{have} to turn the other carboxylic acid into a carboxylate, but we can catalyze/accelerate the reaction by doing so with the addition of catalytic pyridine.
            \end{itemize}
        \end{itemize}
        \item ${\color{white}hi}$
        \begin{center}
            \footnotesize
            \setchemfig{atom sep=1.4em}
            \schemestart
                \chemfig{*6(=-=(-C~N)-=-)}
                \arrow{->[\begin{tabular}{l}
                    1. DIBAL-H\\
                    2. \ce{H3O+}
                \end{tabular}]}[,1.5]
                \color{rex}
                \chemfig{*6(=-=(-(=[2]O)-[:-30]H)-=-)}
            \schemestop
        \end{center}
        \begin{itemize}
            \item We can also buy DIBAL-D.
            \item We're assuming that we're running this for only 15 mins, and thus stopping at the aldehyde. Running for longer will eventually take us down to the alcohol.
            \item To protonate a nitrile, we need a very strong acid (e.g., concentrated sulfuric acid).
            \item Goes over the mechanism, but in less depth than lecture.
        \end{itemize}
        \item ${\color{white}hi}$
        \begin{center}
            \footnotesize
            \setchemfig{atom sep=1.4em}
            \schemestart
                \chemfig{HO-[:30](=[2]O)-[:-30]-[:30]-[:-30]-[:30]OH}
                \arrow{->[\begin{tabular}{l}
                    1. \ce{H+}, $[-\ce{H2O}]$\\
                    2. DIBAL-H
                \end{tabular}]}[,1.8]
                \color{rex}
                \chemfig{[:18]*5(--O-(-OH)--)}
            \schemestop
        \end{center}
        \begin{itemize}
            \item We first protonate the carboxylic acid oxygen, and then the alcohol at the end attacks the carbonyl.
            \item Water leaves, yielding a 5-membered cyclic lactone.
            \begin{itemize}
                \item Cyclic lactones are more stable as 5-membered rings than 6-membered rings.
            \end{itemize}
            \item DIBAL-H reduces the carbonyl to an alcohol.
        \end{itemize}
        \item Rank the following in order of rate of nucleophilic acyl substitution with an alkoxide nucleophile.
        \begin{center}
            \footnotesize
            \setchemfig{atom sep=1.4em}
            \begin{tikzpicture}
                \node at (0,0) {\chemfig{**6(---(-(=[2]O)-[:-30]Cl)---)}};
                \node at (3,0) {\chemfig{**6(---(-(=[2]O)-[:-30]I)---)}};
                \node at (6,0) {\chemfig{**6(---(-(=[2]O)-[:-30]OCF_3)---)}};
                \node at (9,0) {\chemfig{**6(---(-(=[2]O)-[:-30]OMe)---)}};
                \node at (12,0) {\chemfig{**6(---(-(=[2]O)-[:-30]OH)---)}};

                \small
                \node [draw,inner sep=5pt] at (0,-1.5) {\textcolor{rex}{2}};
                \node [draw,inner sep=5pt] at (3,-1.5) {\textcolor{rex}{1}};
                \node [draw,inner sep=5pt] at (6,-1.5) {\textcolor{rex}{3}};
                \node [draw,inner sep=5pt] at (9,-1.5) {\textcolor{rex}{4}};
                \node [draw,inner sep=5pt] at (12,-1.5) {\textcolor{rex}{5}};
            \end{tikzpicture}
        \end{center}
        \begin{itemize}
            \item The determining factor is the stability of the leaving group.
        \end{itemize}
        \item A long "propose a synthesis" question.
        \begin{center}
            \footnotesize
            \setchemfig{atom sep=1.4em}
            \color{rex}
            \schemestart
                \chemfig{[:18]*5(--(-=_[:-30])---)}
                \arrow{->[\begin{tabular}{l}
                    1. \ce{O3}\\
                    2. \ce{Me2S}
                \end{tabular}]}[,1.2]
                \chemfig{[:18]*5(--(-(=[2]O)-[:-30]H)---)}
                \arrow{->[\begin{tabular}{l}
                    1. \ce{CrO3}\\
                    2. \ce{H2SO4 + H2O}
                \end{tabular}]}[,2]
                \chemfig{[:18]*5(--(-(=[2]O)-[:-30]OH)---)}
                \arrow{->[\begin{tabular}{l}
                    1. \ce{K2CO3}\\
                    2. \ce{EtI}
                \end{tabular}]}[,1.4]
                \chemfig{[:18]*5(--(-(=[2]O)-[:-30]OEt)---)}
                \arrow{->[*{0}DIBAL-D]}[-90]
                \chemfig{[:18]*5(--(-(=[2]O)-[:-30]D)---)}
                \arrow{->[*{0.-90}\setchemfig{atom sep=1.3em}\chemfig{Ph_3P=}]}[180,1.3]
                \chemfig{[:18]*5(--(-(-[2]D)=_[:-30])---)}
                \arrow{->[*{0.-90}\begin{tabular}{l}
                    1. \ce{OsO4}\\
                    2. \ce{NaHSO3}
                \end{tabular}]}[180,1.5]
                \chemfig{[:18]*5(--(-(-[:110]D)(-[:70]OH)-[:-30]-[:30]OH)---)}
            \schemestop
        \end{center}
        \begin{itemize}
            \item First thought: Dihydroxylation. But this doesn't provide a good way to incorporate deuterium. So we want our next-to-last intermediate to be like our reactant except with a deuterium in the right place.
            \item Knowing that DIBAL-D is a good way to incorporate a single equivalent of deuterium, we can backtrack through a Wittig to a deuterated aldehyde.
            \item If we want to follow the DIBAL-D route, we backtrack even further to an ester.
            \item Then to a carboxylic acid, which we can create from the initial alkene via ozonolysis.
            \begin{itemize}
                \item Note that we can get directly from an alkene to a carboxylic acid with 1. \ce{O3}, 2. \ce{Me2S, H2O2}, where the peroxide attacks either the molozonide or the ozonide.
            \end{itemize}
            \item This is a greater than exam strength question.
        \end{itemize}
    \end{enumerate}
    \item We will not get "no reaction" questions on the exam.
    \item For any mechanism questions, we will get a complete acid (i.e., one with a defined conjugate base and not just \ce{H+}).
\end{itemize}



\section{Carboxylic Acids and Derivatives 3}
\begin{itemize}
    \item \marginnote{4/14:}Announcements:
    \begin{itemize}
        \item Lecture 5 has now been posted on Canvas $>$ Panopto.
        \item CHEM 23500, Fridays at 12:30 PM, Kent 107.
        \begin{itemize}
            \item A new pilot course consisting of chem professors giving a single lecture on their research.
            \item Levin goes tomorrow.
        \end{itemize}
        \item PSet 2 due Tuesday.
        \item Midterm next Thursday.
        \begin{itemize}
            \item Both PSet 2 and the midterm only cover through today's lecture.
            \item How to study for the exam: For each reaction we've learned, we need to know the products, conditions, and mechanism.
            \item The best way to master the information is to take the above information and connect it from one reaction to the next.
            \item Start from a generic carbonyl compound and make a web of everywhere you can convert and what gets you where.
            \item Still make a study sheet even if you don't use it because it's great preparation.
        \end{itemize}
    \end{itemize}
    \item Last time: Levin introduced a number of reactions to convert from carboxylic acid derivatives to aldehydes/ketones.
    \item Today: Reactions that convert from aldehydes/ketones to carboxylic acid derivatives.
    \begin{itemize}
        \item Currently, we only know how to get a carboxylic acid, and the only way we know how to do that is using Jones reagent.
        \item What we want to develop are insertion reactions, i.e., reactions that can stick a heteroatom into a \ce{C-H} or \ce{C-R$'$} bond.
        \item This is Levin's favorite lecture of the course because it's very similar to what he works on; the reactions we talk about are what inspired his research.
    \end{itemize}
    \item Four insertion reactions.
    \begin{enumerate}
        \item Baeyer-Villiger oxidation.
        \item Schmidt reaction.
        \item Curtius rearrangement.
        \item Beckmann rearrangement.
    \end{enumerate}
    \item The Baeyer-Villiger oxidation.
    \item General form.
    \begin{center}
        \footnotesize
        \setchemfig{atom sep=1.4em}
        \schemestart
            \chemfig{*6(----(=O)--)}
            \arrow{->[mCPBA]}[,1.2]
            \chemfig{[:-12.86]*7(----O-(=O)--)}
        \schemestop
    \end{center}
    \begin{itemize}
        \item Transforms a ketone into an ester; the general form above transforms a ketone into a \textbf{lactone}.
        \item This is one of the most intuitive reactions to reverse engineer in a synthesis problem.
        \item Important acidity properties of mCPBA.
        \begin{itemize}
            \item The $\pKa$ of benzoic acid is -4; benzyl alcohol is 15; mCPBA is 8. mCPBA is of intermediate acidity because there's no conjugation but the ketone is a strong EWG.
            \item It's acidity means we don't need to add an external acid catalyst.
        \end{itemize}
        \item Other reasons to use mCPBA.
        \begin{itemize}
            \item In layman's terms, the active part of the molecule is the peracid functional group, but we use a chlorinated benzene ring to make the molecule both more reactive and less explosive.
            \item More specifically, peracids are explosive. However, chlorine burns endothermically, by which we mean that making \ce{HCl} from water requires heat. Thus, if the peracid were to begin combusting, a lot of the energy would go toward making \ce{HCl} and not toward the explosive chain reaction. Additionally, chlorine is electron withdrawing from the meta position, meaning that the initial deprotonation is favored by having a more stable conjugate base.
            \item Note that adding chlorine atoms to compounds is actually an oft-used trick to reduce their explosivity.
        \end{itemize}
        \item In sum, other peracids can work, but mCPBA is the most practical.
    \end{itemize}
    \item \textbf{Lactone}: A cyclic ester.
    \item Mechanism.
    \begin{figure}[h!]
        \centering
        \footnotesize
        \schemestart
            \chemfig{*6(----(=@{O1}\charge{90=\:}{O})--)}
            % \arrow{->[\chemfig[atom sep=1.4em]{*6(=-=(-(=[2]O)-[:-30]O-[:30]O-[:-30]H)-=(-Cl)-)}]}[,2.9]
            \arrow{->[\chemfig[atom sep=1.4em]{@{H2}H-[@{sb2}:120]@{O2}O-[:60]O-[:120](=[4]O)-[:60]*6(-=(-Cl)-=-=)}]}[,2]
            \chemfig{*6(----@{C3}(=[@{db3}]@{O3}\charge{90:3pt=$\oplus$}{O}-[:30]H)--)}
            % \arrow{0}[,0.1]\+{,1em,2em}
            \arrow{0}[,0.1]\+{,1em,4.5em}
            \chemfig{@{O4}\charge{180=\:,-135:1pt=$\oplus$}{O}-[:60]O-[:120](=[4]O)-[:60]*6(-=(-Cl)-=-=)}
            % \chemfig{*6(=-=(-(=[2]O)-[:-30]O-[:30]\charge{90=\:,45:1pt=$\ominus$}{O})-=(-Cl)-)}
            \arrow
            \chemname[-3em]{\chemfig{*6(----[@{sb5a}]([:93.21]*7(-@{O5}O-[@{sb5b}]O-[@{sb5c}](-*6(-=(-Cl)-=-=))=[@{db5}]O-[,,,,white]@{H5}H-[@{sb5d}]O-[@{sb5e}]))--)}}{Criegee intermediate}
            \arrow{->[][-\ce{mCBA}]}[,1.2]
            \chemfig{[:-12.86]*7(----O-(=O)--)}
        \schemestop
        \chemmove{
            \draw [curved arrow={6pt}{2pt}] (O1) to[out=90,in=180,out looseness=2] (H2);
            \draw [curved arrow={2pt}{2pt}] (sb2) to[bend right=80,looseness=3] (O2);
            \draw [curved arrow={6pt}{3pt}] (O4) to[out=180,in=30,out looseness=1] (C3);
            \draw [curved arrow={3pt}{2pt}] (db3) to[bend left=90,looseness=3] (O3);
            \draw [curved arrow={2pt}{2pt}] (sb5a) to[out=60,in=-127] (O5);
            \draw [curved arrow={2pt}{2pt}] (sb5b) to[bend left=60,looseness=1.5] (sb5c);
            \draw [curved arrow={4pt}{2pt}] (db5) to[bend left=40,looseness=1.3] (H5);
            \draw [curved arrow={2pt}{2pt}] (sb5d) to[bend left=60,looseness=1.5] (sb5e);
        }
        \caption{Baeyer-Villiger mechanism.}
        \label{fig:mechanismBaeyerVilliger}
    \end{figure}
    \begin{itemize}
        \item Criegee made his fame for studying this reaction. He was the one who actually first proposed the existence of the intermediate that now bears his name.
        \item In the Criegee intermediate, one of the neighboring \ce{C-C} bonds can slide over in a \textbf{migration}.
        \begin{itemize}
            \item Think about the parallel to hydroboration/oxidation (Figure \ref{fig:mechanismAlkyneHydroboration}) and the formation of the enol boronate.
        \end{itemize}
        \item Additionally, the \ce{O-O} bond is pretty weak and can be displaced.
        \begin{itemize}
            \item However, because this is mCPBA (with its electron withdrawing carbonyl), the \ce{O-O} electrons can swing around and facilitate the attack of the carbonyl electrons on the substrate's acidic proton.
        \end{itemize}
        \item Last step arrow pushing chronology: The \ce{O-H} electrons swinging down. Reforming the carbonyl provides the oomph that breaks the \ce{C-C} bond. The \ce{C-C} electrons migrate. This makes everything else just swing around.
        \begin{itemize}
            \item Note that this chronology is not technically accurate; curved arrows are a human invention we assert overtop a concerted step. However, this is a good trick to think of for memorization purposes.
        \end{itemize}
    \end{itemize}
    \item \textbf{Migratory aptitude}: How likely a group is to shift, or migrate.
    \begin{itemize}
        \item Discussing the migratory aptitude of different \ce{R} groups we might see on either side of the ketone (in a Baeyer-Villiger, for instance) allows us to predict the products of the reaction in ambiguous cases, such as with asymmetric ketones.
    \end{itemize}
    \item Asymmetric ketones in the Baeyer-Villiger.
    \begin{figure}[H]
        \centering
        \footnotesize
        \begin{subfigure}[b]{\linewidth}
            \centering
            \schemestart
                \chemfig{-[:-30](-[:-70])(-[:-110])-[:30](=[2]O)-[:-30]}
                \arrow{->[mCPBA]}[,1.2]
                \chemfig{-[:-30](-[:-70])(-[:-110])-[:30]O-[:-30](=[6]O)-[:30]}
            \schemestop
            \caption{A selective case.}
            \label{fig:baeyerVilligerAsymmetrica}
        \end{subfigure}\\[2em]
        \begin{subfigure}[b]{\linewidth}
            \centering
            \schemestart
                \chemfig{(-[:-150]*6(-=-=-=))(-[:-30](-[::60])(-[::-60]))(=[2]O)}
                \arrow{->[mCPBA]}[,1.2]
                \chemname{\chemfig{(-[:-150]O-[:150]*6(-=-=-=))(-[:-30](-[::60])(-[::-60]))(=[2]O)}}{50\%}
                \+{,,-1.5em}
                \chemname{\chemfig{(-[:-150]*6(-=-=-=))(-[:-30]O-[:30](-[::60])(-[::-60]))(=[2]O)}}{50\%}
            \schemestop
            \caption{An unselective case.}
            \label{fig:baeyerVilligerAsymmetricb}
        \end{subfigure}
        \caption{Asymmetric ketones in the Baeyer-Villiger.}
        \label{fig:baeyerVilligerAsymmetric}
    \end{figure}
    \begin{itemize}
        \item How likely a \ce{C-C} bond is to move depends on what's attached to the $\alpha$-carbon.
        \item Selectivity for this reaction (not the same for all reactions):
        \begin{equation*}
            3^\circ\text{ alkyls} > 2^\circ\text{ alkyls}
            \approx \text{aromatics}
            > 1^\circ\text{ alkyls}
            > \text{methyl}
        \end{equation*}
        where the $\alpha$-carbon being $3^\circ$ promotes the reaction the most and it being a methyl group promotes it the least.
        \begin{itemize}
            \item This does work with aldehydes; hydrogen will migrate faster than anything else (i.e., forming carboxylic acids).
            \item Jones is a cheat to do the same thing, though.
        \end{itemize}
        \item Because of differing migratory aptitudes, the Baeyer-Villiger is not always useful synthetically.
        \item Always think about a precursor being asymmetric when doing a retrosynthetic analysis!
    \end{itemize}
    \item Note that epoxidation is usually faster than the Baeyer-Villiger. Thus, compounds with both an alkene and a ketone that react with mCPBA will form epoxides and the carbonyls will be untouched.
    \item Schmidt reaction.
    \item General form.
    \begin{center}
        \footnotesize
        \setchemfig{atom sep=1.4em}
        \schemestart
            \chemfig{[:18]*5(---(=O)--)}
            \arrow{->[\ce{HN3}]}
            \chemfig{*6(---NH-(=O)--)}
        \schemestop
    \end{center}
    \begin{itemize}
        \item You can use catalytic acid, but you don't need it.
    \end{itemize}
    \item \textbf{Hydrazoic acid}: A toxic, volatile, and explosive substance. \emph{Structure} \ce{HN=N+=N-}
    \begin{itemize}
        \item This is useful industrially, but less useful in the lab (because of all the associated hazards).
    \end{itemize}
    \item Mechanism.
    \begin{figure}[h!]
        \centering
        \footnotesize
        \schemestart
            \chemfig{[:18]*5(---(=@{O1}\charge{90=\:}{O})--)}
            \arrow{->[\chemfig[atom sep=1.4em]{@{H2}H-[@{sb2}:-60]@{N2}N=\charge{90:3pt=$\oplus$}{N}=\charge{90:3pt=$\ominus$}{N}}]}[,1.8]
            \chemfig{[:18]*5(---@{C3}(=[@{db3}]@{O3}\charge{90:3pt=$\oplus$}{O}-[:30]H)--)}
            \arrow{0}[,0.5]
            \chemfig{@{N4}\charge{180=\:,90:3pt=$\ominus$}{N}=\charge{90:3pt=$\oplus$}{N}=\charge{90:3pt=$\ominus$}{N}}
            \arrow
            \chemfig{[:18]*5(---(-[:70]N=[@{db5a}:10]\charge{90:3pt=$\oplus$}{N}=[@{db5b}:10]@{N5}\charge{-90=\:,90:3pt=$\ominus$}{N})(-[:110]O-[:170]H)--)}
            \arrow{->[*{0} {\chemfig[atom sep=1.4em]{@{H6}H-[@{sb6}]@{N6}N_3}}]}[-90]
            \subscheme{
                \chemfig{[:18]*5(---[@{sb7a}](-[:70]@{N7a}N(-[2]H)-[@{sb7b}0]@{N7b}\charge{90:3pt=$\oplus$}{N}~[0]N)(-[@{sb7c}:110]H@{O7}\charge{90=\:}{O})--)}
                \arrow{0}[,0.1]\+
                \chemfig{\charge{45:1pt=$\ominus$}{N}_3}
            }
            \arrow{->[][*{0.90}-\ce{N2}]}[180]
            \subscheme{
                \chemfig{*6(---NH-(=@{O9}\charge{90:3pt=$\oplus$}{O}-[@{sb9}:30]@{H9}H)--)}
                \arrow{0}[,0.1]\+
                \chemfig{@{N10}\charge{90=\:,45:1pt=$\ominus$}{N}_3}
            }
            \arrow{->[][*{0.90}-\ce{HN3}]}[180]
            \chemfig{*6(---NH-(=O)--)}
        \schemestop
        \chemmove{
            \draw [curved arrow={6pt}{2pt}] (O1) to[out=90,in=90,looseness=1.8] (H2);
            \draw [curved arrow={2pt}{2pt}] (sb2) to[bend right=80,looseness=3] (N2);
            \draw [curved arrow={6pt}{3pt}] (N4) to[out=180,in=30] (C3);
            \draw [curved arrow={3pt}{2pt}] (db3) to[bend left=90,looseness=3] (O3);
            \draw [curved arrow={3pt}{2pt}] (db5a) to[out=-80,in=90] (H6);
            \draw [curved arrow={2pt}{2pt}] (sb6) to[bend right=90,looseness=3] (N6);
            \draw [curved arrow={6pt}{3pt}] (N5) to[bend left=90,looseness=3] (db5b);
            \draw [curved arrow={6pt}{2pt}] (O7) to[out=90,in=-150,looseness=7] (sb7c);
            \draw [curved arrow={2pt}{2pt}] (sb7a) to[out=54,in=-60,looseness=1.1] (N7a);
            \draw [curved arrow={2pt}{2pt}] (sb7b) to[bend right=90,looseness=3] (N7b);
            \draw [curved arrow={6pt}{2pt}] (N10) to[out=90,in=0,looseness=1.2] (H9);
            \draw [curved arrow={2pt}{2pt}] (sb9) to[bend left=80,looseness=2.5] (O9);
        }
        \caption{Schmidt reaction mechanism.}
        \label{fig:mechanismSchmidt}
    \end{figure}
    \begin{itemize}
        \item Getting rid of nitrogen is a massive thermodynamic sink/driving force.
        \item One of the molecules of hydrazoic acid is being incorporated, and the other is a catalyst (which we can supplement with external acid catalyst). It will go faster with an acid catalyst if the acid used is stronger than hydrazoic acid, but the acid is not necessary.
        \item Again, the arrow pushing chronology starts at the alcohol oxygen for the final step.
    \end{itemize}
    \item Migratory aptitude is the same for Schmidt as for the Baeyer-Villiger.
    \item The Schmidt does work with aldehydes; hydrogen will migrate faster than anything else.
    \begin{itemize}
        \item You would form an amide in this case.
        \item It's rare to see this in the literature, though.
    \end{itemize}
    \item Using an alkyl group in place of the hydrogen on the hydrazoic acid \emph{requires} catalytic acid (the new acid isn't strong enough to catalyze its own chemistry). The alkyl group just gets added to the nitrogen in the product.
    \item The Schmidt reaction also works intramolecularly.
    \begin{figure}[h!]
        \centering
        \footnotesize
        \schemestart
            \chemfig{*6(---(--[2]--[2]N=[0]\charge{90:3pt=$\oplus$}{N}=[0]\charge{90:3pt=$\ominus$}{N})-(=O)--)}
            \arrow{->[\ce{H+}]}
            \chemname{\chemfig{[:{-360/28}]*7(---(*5(---?))-N?-(=O)--)}}{Major product}
            \+{,,1.5em}
            \chemname{\chemfig{*6(---([:72,1.7]--[::72,1.7]-[::72,1.7]?)-(=O)-N?-)}}{Minor product}
        \schemestop
        \caption{Intramolecular Schmidt reaction.}
        \label{fig:intramolecularSchmidt}
    \end{figure}
    \begin{itemize}
        \item The intramolecular Schmidt builds complexity really quickly.
        \item When you're building natural molecules, it allows you to get up from simple cheap starting materials to complex polycycles quite quickly, which you want.
    \end{itemize}
    \item The Curtius rearrangement.
    \item General form.
    \begin{equation*}
        \ce{RCOCl ->[NaN3][\Delta] RNCO ->[NaOH][H2O] RNH2}
    \end{equation*}
    \begin{itemize}
        \item The product of the first step is an \textbf{isocyanate}.
    \end{itemize}
    \item Mechanism.
    \begin{figure}[H]
        \centering
        \footnotesize
        \begin{subfigure}[b]{\linewidth}
            \centering
            \schemestart
                \chemfig{R-[:30]@{C1}(=[@{db1}2]@{O1}O)-[:-30]Cl}
                \arrow{->[*{0} {\chemfig[atom sep=1.4em]{\charge{90:3pt=$\oplus$}{Na}-[,0.3,,,white]@{N2}\charge{90:3pt=$\ominus$}{N}=\charge{90:3pt=$\oplus$}{N}=\charge{90:3pt=$\ominus$}{N}}}]}[-90]
                \chemfig{R-[:30](-[@{sb3a}:110]@{O3}\charge{180=\:,90:3pt=$\ominus$}{O})(-[:70]N=\charge{90:3pt=$\oplus$}{N}=\charge{90:3pt=$\ominus$}{N})-[@{sb3b}:-30]@{Cl3}Cl}
                \arrow{->[][-\ce{Cl-}]}
                \chemleft{[}
                    \subscheme{
                        \chemfig{R-[:30](=[@{db4a}2]@{O4}O)-[@{sb4}:-30]N=[@{db4b}:30]\charge{90:3pt=$\oplus$}{N}=[@{db4c}:30]@{N4}\charge{90:3pt=$\ominus$}{N}}
                        \arrow{<->}
                        \chemfig{R-[@{sb5a}:30](-[@{sb5b}2]@{O5}\charge{180=\:,45:1pt=$\ominus$}{O})=^[:-30]@{N5a}N-[@{sb5c}:30]@{N5b}\charge{90:3pt=$\oplus$}{N}~[:30]N}
                    }
                \chemright{]}
                \arrow{->[][-\ce{N2}]}
                \chemfig{R-[:30]N=[:-30]C=[:-30]O}
            \schemestop
            \chemmove{
                \draw [curved arrow={11pt}{2pt}] (N2) to[out=90,in=-90] (C1);
                \draw [curved arrow={3pt}{2pt}] (db1) to[bend left=90,looseness=3] (O1);
                \draw [curved arrow={6pt}{2pt}] (O3) to[out=180,in=-150,looseness=4] (sb3a);
                \draw [curved arrow={2pt}{2pt}] (sb3b) to[bend right=90,looseness=3] (Cl3);
                \draw [curved arrow={5pt}{3pt},blx] ([yshift=2mm]N4.north) to[bend right=80,looseness=2.5] (db4c);
                \draw [curved arrow={3pt}{2pt},blx] (db4b) to[bend right=60,looseness=1.5] (sb4);
                \draw [curved arrow={3pt}{2pt},blx] (db4a) to[bend left=90,looseness=3] (O4);
                \draw [curved arrow={2pt}{2pt}] (sb5a) to[bend right=40,looseness=1.2] (N5a);
                \draw [curved arrow={6pt}{2pt}] (O5) to[bend right=90,looseness=3] (sb5b);
                \draw [curved arrow={2pt}{2pt}] (sb5c) to[bend right=90,looseness=3] (N5b);
            }
            \caption{Step 1.}
            \label{fig:mechanismCurtiusa}
        \end{subfigure}\\[2em]
        \begin{subfigure}[b]{\linewidth}
            \centering
            \schemestart
                \chemfig{R-[:30]N=[:-30]@{C1}C=[@{db1}:-30]@{O1}O}
                \arrow{->[*{0} {\chemfig[atom sep=1.4em]{-[,0.5,,,white]@{O2}\charge{90=\:,135:1pt=$\ominus$}{O}H}}]}[-90]
                \chemfig{R-[:30]N(-[2,0.8,,,white])=[@{db3}:-30](-[6]OH)-[@{sb3}:30]@{O3}\charge{90=\:,45:1pt=$\ominus$}{O}}
                \arrow{->[\chemfig[atom sep=1.4em]{@{H4}H-[@{sb4}]@{O4}OH}][-\ce{OH-}]}[,1.2]
                \chemfig{R-[:30]N(-[2]H)-[:-30](-[6]@{O5}O-[@{sb5}:-30]@{H5}H)=[:30]O}
                \arrow{->[\chemfig[atom sep=1.4em]{@{O6}\charge{180=\:,135:1pt=$\ominus$}{O}H}][-\ce{H2O}]}
                \chemfig{R-[:30]@{N7}N(-[2]H)-[@{sb7a}:-30](-[@{sb7b}6]@{O7}\charge{180=\:,0:3pt=$\ominus$}{O})=[:30]O}
                \arrow{->[][-\ce{CO2}]}
                \chemfig{R-[:30]@{N8}\charge{90:3pt=$\ominus$}{N}-[:-30]H}
                \arrow{->[\chemfig[atom sep=1.4em]{@{H9}H-[@{sb9}]@{O9}OH}][-\ce{OH-}]}[,1.2]
                \chemfig{R-NH_2}
            \schemestop
            \chemmove{
                \draw [curved arrow={6pt}{2pt}] (O2) to[out=90,in=-120,in looseness=1.2] (C1);
                \draw [curved arrow={3pt}{2pt}] (db1) to[bend left=90,looseness=3] (O1);
                \draw [curved arrow={6pt}{2pt}] (O3) to[out=90,in=120,looseness=4] (sb3);
                \draw [curved arrow={3pt}{2pt}] (db3) to[out=80,in=90,out looseness=3,in looseness=1.7] (H4);
                \draw [curved arrow={2pt}{2pt}] (sb4) to[bend left=90,looseness=3] (O4);
                \draw [curved arrow={6pt}{2pt}] (O6) to[out=180,in=90] (H5);
                \draw [curved arrow={2pt}{2pt}] (sb5) to[bend left=90,looseness=3] (O5);
                \draw [curved arrow={6pt}{2pt}] (O7) to[bend left=90,looseness=3] (sb7b);
                \draw [curved arrow={2pt}{2pt}] (sb7a) to[bend right=90,looseness=3] (N7);
                \draw [curved arrow={11pt}{2pt}] (N8) to[out=90,in=90,looseness=2] (H9);
                \draw [curved arrow={2pt}{2pt}] (sb9) to[bend left=90,looseness=3] (O9);
            }
            \caption{Step 2.}
            \label{fig:mechanismCurtiusb}
        \end{subfigure}
        \caption{Curtius rearrangement mechanism.}
        \label{fig:mechanismCurtius}
    \end{figure}
    \begin{itemize}
        \item Acyl azides are sometimes isolable. Heating one up will always cause it to convert, though.
    \end{itemize}
    \item Isocyanates can also be trapped to form carbamates.
    \begin{figure}[h!]
        \centering
        \footnotesize
        \schemestart
            \chemfig{RN=C=O}
            \arrow{->[\ce{R$'$OH}][cat. base]}[,1.3]
            \chemfig{HRN-[:30](=[2]O)-[:-30]OR'}
        \schemestop
        \caption{Carbamate formation.}
        \label{fig:carbamates}
    \end{figure}
    \begin{itemize}
        \item Use an alcohol and catalytic base.
    \end{itemize}
    \item This is the reaction behind guys on YouTube spraying insulation/fire retardant foam and it expanding on the wall behind them.
    \begin{itemize}
        \item You have one diisocyanate and add ethylene glycol at the last second; the foaming up is the polymerization resulting in polyurethane.
        \item We will not be asked about the foam thing specifically, but we may be asked to draw the product of a compound with two isocyanates at each end.
        \item Levin disses Snyder lol -- "not gonna ask you what color tie I'm wearing either."
    \end{itemize}
    \item Converting from a carboxylic acid to an isocyanate without going through an acid chloride intermediate.
    \begin{center}
        \footnotesize
        \setchemfig{atom sep=1.4em}
        \schemestart
            \chemfig{R-[:30](=[2]O)-[:-30]OH}
            \arrow{->[DPPA]}
            \chemfig{RN=C=O}
        \schemestop
    \end{center}
    \item \textbf{DPPA}: Diphenylphosphoryl azide. \emph{Structure}
    \begin{figure}[h!]
        \centering
        \footnotesize
        \chemfig{P(=[2]O)(-[:-30]N_3)(-[:-110]PhO)(-[:-150]PhO)}
        \caption{Diphenylphosphoryl azide (DPPA).}
        \label{fig:DPPA}
    \end{figure}
    \begin{itemize}
        \item Just like \ce{SOCl2} and \ce{POCl3} work as dehydrating agents (with chloride), DPPA works as a dehydrating agent (with azide).
    \end{itemize}
    \item Beckmann rearrangement.
    \item General form.
    \begin{center}
        \footnotesize
        \setchemfig{atom sep=1.4em}
        \schemestart
            \chemfig{*6(----(=O)--)}
            \arrow{->[\ce{H2NOH}][\ce{H3O+}]}[,1.2]
            \chemfig{[:-12.86]*7(----[,,,1]NH-(=O)--)}
        \schemestop
    \end{center}
    \begin{itemize}
        \item You can do this all in one go, or you can isolate the oximes from the first reagent and removing water, and then add in acid to finish it off.
        \item Quite similar to the Schmidt, but hydroxyl amine is not as toxic, volatile, or explosive, so this is the preferred one.
    \end{itemize}
    \item Mechanism.
    \begin{figure}[h!]
        \centering
        \vspace{2em}
        \footnotesize
        \schemestart
            % \chemfig{*6(----(=O)--)}
            % \arrow{->[\ce{H2NOH}][-\ce{H2O}]}[,1.2]
            \chemfig{*6(----(=N-[:30]@{O1}\charge{90=\:}{O}H)--)}
            \arrow{->[\chemfig[atom sep=1.4em]{@{H2}H-[@{sb2}]@{O2}\charge{90:3pt=$\oplus$}{O}H_2}][-\ce{H2O}]}[,1.3]
            \chemfig{*6(----(=@{N3}N-[@{sb3a}:30]@{O3}\charge{90:3pt=$\oplus$}{O}H_2)-[@{sb3b}]-)}
            \arrow{->[][-\ce{H2O}]}
            \chemleft{[}
                \subscheme{
                    \chemfig{[:{-360/28}]*7(----@{C4}\charge{45:1pt=$\oplus$}{}=[@{db4}]@{N4}\charge{90=\:}{N}--)}
                    \arrow{<->}
                    \chemfig{[:{-360/28}]*7(----~\charge{90:3pt=$\oplus$}{N}--)}
                }
            \chemright{]}
            \arrow{->[\chemfig{H_2@{O6}\charge{90=\:}{O}}]}
            \chemfig{[:{-360/28}]*7(----(-[@{sb7a}]\charge{45:1pt=$\oplus$}{O}(-[2]H)-[@{sb7b}0]@{H7}H)=[@{db7}]@{N7}N--)}
            \arrow{->[*{0}\chemfig{H_2@{O8}\charge{90=\:}{O}}][*{0}-\ce{H3O+}]}[-90]
            \chemfig{[:{-360/28}]*7(----(=\charge{90:3pt=$\oplus$}{O}-[::-60]H)-@{N9}\charge{90:3pt=$\ominus$}{N}--)}
            \arrow{->[*{0.-90} {\chemfig[atom sep=1.4em]{@{H10}H-[@{sb10}]@{O10}\charge{90:3pt=$\oplus$}{O}H_2}}][*{0.90}-\ce{H2O}]}[180,1.3]
            \chemfig{[:{-360/28}]*7(----(=@{O11}\charge{90:3pt=$\oplus$}{O}-[@{sb11}::-60]@{H11}H)-\chemabove{N}{H}--)}
            \arrow{->[*{0.-90} \chemfig{H_2@{O12}\charge{90=\:}{O}}][*{0.90}-\ce{H3O+}]}[180]
            \chemfig{[:{-360/28}]*7(----(=O)-\chemabove{N}{H}--)}
        \schemestop
        \chemmove{
            \draw [curved arrow={6pt}{2pt}] (O1) to[out=90,in=90,looseness=2] (H2);
            \draw [curved arrow={2pt}{2pt}] (sb2) to[out=110,in=130,looseness=3] (O2);
            \draw [curved arrow={2pt}{2pt}] (sb3b) to[out=120,in=-150,looseness=1.2] (N3);
            \draw [curved arrow={2pt}{2pt}] (sb3a) to[bend right=90,looseness=3] (O3);
            \draw [curved arrow={6pt}{2pt},blx] (N4) to[out=90,in=60,looseness=4] (db4);
            \draw [curved arrow={6pt}{11pt}] (O6) to[out=90,in=45] (C4);
            \draw [curved arrow={6pt}{2pt}] (O8) to[out=90,in=-90] (H7);
            \draw [curved arrow={2pt}{2pt}] (sb7b) to[bend left=60,looseness=1.5] (sb7a);
            \draw [curved arrow={4pt}{2pt}] (db7) to[bend left=90,looseness=3] (N7);
            \draw [curved arrow={11pt}{2pt}] (N9) to[out=90,in=90,looseness=1.2] (H10);
            \draw [curved arrow={2pt}{2pt}] (sb10) to[out=110,in=130,looseness=3] (O10);
            \draw [curved arrow={6pt}{2pt}] (O12) to[out=90,in=90,looseness=1.2] (H11);
            \draw [curved arrow={2pt}{2pt}] (sb11) to[bend left=90,looseness=3] (O11);
        }
        \caption{Beckmann rearrangement mechanism.}
        \label{fig:mechanismBeckmann}
    \end{figure}
    \begin{itemize}
        \item The first part of the mechanism proceeds just like oxime formation (see Aldehydes and Ketones 1). This is why we show the mechanism beginning from an oxime.
        \item There is debate over the mechanism. We are only responsible for the one above, though.
        \item The triple-bonded nitrogen resonance form is quite strained, and thus the carbocation species is the major contributor.
    \end{itemize}
    \item Caperlactam (the end product in Figure \ref{fig:mechanismBeckmann}) is made from cyclohexanone in quantites of millions of tons per year because it is a precursor to nylon, which is just caperlactam following a ring opening.
    \item Migratory aptitude (same as for Baeyer-Villiger and Schmidt).
    \item An orbital explanation of the migratory aptitude in this case.
    \begin{itemize}
        \item The step 2 migration is an S\textsubscript{N}2 process.
        \item As such, we want to see donation into the antibonding $\sigma$ orbital of the \ce{N-O} bond to make this proceed. This is why the carbon "behind" the oxime selectively migrates.
        \item However, in acidic solution, oximes exist in equilibrium with their \emph{cis}/\emph{trans} counterpart.
        \item As such, since sterics disfavor the \ce{OH} being on the same side as a bulky group, we will more commonly observe the oxime in solution where the \ce{OH} points away from the bulky group, thus forming more of this product.
    \end{itemize}
    \item The Beckmann rearrangement also helps create azithromycin, the active ingredient in the common Z-pak antibiotics.
    \begin{itemize}
        \item Erythromycin is produced by some bacteria to defend against other bacteria.
        \item You need a big dose of it because it's half-life in your body is 1.5 hours. It also is really tough on your body because it kills all your gut bacteria.
        \item A couple of chemical steps including the Beckmann rearrangement takes it to azithromycin, which has a half-life of 68 hours.
    \end{itemize}
\end{itemize}




\end{document}