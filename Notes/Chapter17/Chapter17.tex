\documentclass[../notes.tex]{subfiles}

\pagestyle{main}
\renewcommand{\chaptermark}[1]{\markboth{\chaptername\ \thechapter\ (#1)}{}}
\setcounter{chapter}{16}

\begin{document}




\chapter{Carboxylic Acids and Derivatives}
\section{Carboxylic Acids and Derivatives 1}
\begin{itemize}
    \item \marginnote{4/7:}We now consider compounds that have heteroatoms where the $\alpha$ carbon of the carbonyl used to be.
    \begin{itemize}
        \item The heteroatoms can be oxygen (esters), nitrogen, etc.
    \end{itemize}
    \item Today, we will do oxygen and nitrogen nucleophiles but in this context.
    \begin{itemize}
        \item Next Tuesday, we will do carbon and hydrogen nucleophiles in this context.
    \end{itemize}
    \item Carboxylic acid derivatives.
    \begin{figure}[h!]
        \centering
        \footnotesize
        \begin{subfigure}[b]{0.19\linewidth}
            \centering
            \chemfig{R-[:30](=[2]O)-[:-30]OH}
            \caption{Carboxylic acid.}
            \label{fig:carboxylicAcidDerivativesa}
        \end{subfigure}
        \begin{subfigure}[b]{0.19\linewidth}
            \centering
            \chemfig{R-[:30](=[2]O)-[:-30]O-[:30]R'}
            \caption{Ester.}
            \label{fig:carboxylicAcidDerivativesb}
        \end{subfigure}
        \begin{subfigure}[b]{0.19\linewidth}
            \centering
            \chemfig{R-[:30](=[2]O)-[:-30]X}
            \caption{Acid halide.}
            \label{fig:carboxylicAcidDerivativesc}
        \end{subfigure}
        \begin{subfigure}[b]{0.19\linewidth}
            \centering
            \chemfig{R-[:30](=[2]O)-[:-30]O-[:30](=[2]O)-[:-30]R'}
            \caption{Acid anhydride.}
            \label{fig:carboxylicAcidDerivativesd}
        \end{subfigure}
        \begin{subfigure}[b]{0.19\linewidth}
            \centering
            \chemfig{R-[:30](=[2]O)-[:-30]N(-[6]R'')-[:30]R'}
            \caption{Amide.}
            \label{fig:carboxylicAcidDerivativese}
        \end{subfigure}\\[2em]
        \begin{subfigure}[b]{0.22\linewidth}
            \centering
            \chemfig{R-C~N}
            \caption{Nitrile.}
            \label{fig:carboxylicAcidDerivativesf}
        \end{subfigure}
        \begin{subfigure}[b]{0.22\linewidth}
            \centering
            \chemfig{R-[:-30]O-[:30](=[2]O)-[:-30]O-[:30]R'}
            \caption{Carbonate.}
            \label{fig:carboxylicAcidDerivativesg}
        \end{subfigure}
        \begin{subfigure}[b]{0.22\linewidth}
            \centering
            \chemfig{R-[:-30]O-[:30](=[2]O)-[:-30]N(-[6]R'')-[:30]R'}
            \caption{Carbamate.}
            \label{fig:carboxylicAcidDerivativesh}
        \end{subfigure}
        \begin{subfigure}[b]{0.22\linewidth}
            \centering
            \chemfig{R-[:-30]N(-[6]R')-[:30](=[2]O)-[:-30]N(-[6]R''')-[:30]R''}
            \caption{Urea.}
            \label{fig:carboxylicAcidDerivativesi}
        \end{subfigure}
        \caption{Carboxylic acid derivatives.}
        \label{fig:carboxylicAcidDerivatives}
    \end{figure}
    \begin{itemize}
        \item Once again, we will not be tested on nomenclature, but it's good to know.
        \item Acid anhydrides are so named because it is two carboxylic acids, minus a water molecule.
        \item Nitriles are still a carbon bonded to three heteroatoms; it's just the same heteroatom.
    \end{itemize}
    \item A key property of carboxylic acids is that they're\dots acidic.
    \item Acidity.
    \begin{itemize}
        \item Gives the $\pKa$'s of benzoic acid, benzyl alcohol, and phenol to demonstrate that resonance is king in determining acidity.
        \begin{itemize}
            \item Benzoic acid is more acidic than phenol, which is more acidic than benzyl alcohol.
        \end{itemize}
        \item Inductive effects (changes to the $\alpha$ carbon) play a smaller role.
        \item EWGs on arene rings when present play an even smaller role.
        \item These latter two effects allow us to fine-tune acidity.
    \end{itemize}
    \item Methods of carboxylic acid synthesis.
    \begin{enumerate}
        \item Overoxidation.
        \item Carboxylation of Grignards or lithiates.
        \item Nitrile hydrolysis.
    \end{enumerate}
    \item Overoxidation.
    \item General form.
    \begin{equation*}
        \ce{CRH(OH) ->[CrO3, H2SO4][H2O] RCOOH}
    \end{equation*}
    \begin{itemize}
        \item Note that the reagents constitute Jones reagent.
    \end{itemize}
    \item Mechanism.
    \begin{itemize}
        \item Virtually identical to that from \textcite{bib:CHEM22100Notes}.
    \end{itemize}
    \item Carboxyliation of Grignards and lithiates.
    \item General form.
    \begin{equation*}
        \ce{RLi ->[1. CO2][2. H3O+] RCOOH}
    \end{equation*}
    \begin{itemize}
        \item Note that we may use either lithiates (\ce{RLi}) or Grignards (\ce{RMgBr}), even though only an organolithium compound is shown above.
    \end{itemize}
    \item Mechanism.
    \begin{figure}[h!]
        \centering
        \footnotesize
        \schemestart
            \chemfig{R-[@{sb1}]Li}
            \arrow{->[\chemfig[atom sep=1.4em]{O=@{C2}C=[@{db2}]@{O2}O}]}[,1.5]
            \chemname{\chemfig{R-[:30](=[2]O)-[:-30]\charge{45:1pt=$\ominus$}{O}-[,0.6,,,white]\charge{45:1pt=$\oplus$}{Li}}}{Carboxylate salt}
            \arrow{->[\ce{H3O+}]}
            \chemfig{R-[:30](=[2]O)-[:-30]OH}
        \schemestop
        \chemmove{
            \draw [curved arrow={2pt}{2pt}] (sb1) to[bend left=90,looseness=1.5] (C2);
            \draw [curved arrow={3pt}{2pt}] (db2) to[bend left=90,looseness=3] (O2);
        }
        \caption{Carboxylation of lithiates mechanism.}
        \label{fig:mechanismLithiateCarboxylation}
    \end{figure}
    \item Mechanistic interlude: Nucleophilic acyl substitution.
    \begin{figure}[H]
        \centering
        \footnotesize
        \begin{subfigure}[b]{\linewidth}
            \centering
            \schemestart
                \chemfig{R-[:30](=[2]O)-[:-30]LG}
                \arrow{0}[,0.1]\+
                \chemfig{Nu-H}
                \arrow{->[cat. \ce{HX}]}[,1.2]
                \chemfig{R-[:30](-[:110]HO)(-[:70]Nu)-[:-30]LG}
                \arrow{->[cat. \ce{HX}]}[,1.2]
                \chemfig{R-[:30](=[2]O)-[:-30]Nu}
                \arrow{0}[,0.1]\+
                \chemfig{LG-H}
            \schemestop
            \caption{Acid-catalyzed reactivity.}
            \label{fig:acidCarboxylica}
        \end{subfigure}\\[2em]
        \begin{subfigure}[b]{\linewidth}
            \centering
            \schemestart
                \chemfig{@{Nu1}Nu-[@{sb1}]@{H1}H}
                \arrow{->[\chemfig{@{B2}\charge{90=\:}{B}}]}
                \chemfig{@{Nu3}\charge{90=\:,45:1pt=$\ominus$}{Nu}}
                \+
                \chemfig{H\charge{90:3pt=$\oplus$}{B}}
                \arrow{-U>[\chemfig[atom sep=1.4em]{R-[:30]@{C5}(=[@{db5}2]@{O5}O)-[:-30]LG}][][][][80]}[,1.5]
                \chemfig{R-[:30](-[@{sb6a}:110]@{O6}\charge{180=\:,90:3pt=$\ominus$}{O})(-[:70]Nu)-[@{sb6b}:-30]@{LG6}LG}
                \arrow{0}[,0.1]\+
                \chemfig{H\charge{90:3pt=$\oplus$}{B}}
                \arrow{-U>[][\chemfig[atom sep=1.4em]{R-[:30](=[2]O)-[:-30]Nu}][][][80]}[,1.3]
                \chemfig{@{LG9}\charge{45:1pt=$\ominus$}{LG}}
                \arrow{0}[,0.1]\+
                \chemfig{@{H10}H-[@{sb10}]@{B10}\charge{90:3pt=$\oplus$}{B}}
                \arrow{->[][-\ce{B}]}
                \chemfig{LG-H}
            \schemestop
            \chemmove{
                \draw [curved arrow={6pt}{2pt}] (B2) to[out=90,in=90,looseness=2] (H1);
                \draw [curved arrow={2pt}{2pt}] (sb1) to[bend right=90,looseness=3] (Nu1);
                \draw [curved arrow={6pt}{3pt}] (Nu3) to[out=90,in=150] (C5);
                \draw [curved arrow={3pt}{2pt}] (db5) to[bend right=90,looseness=3] (O5);
                \draw [curved arrow={6pt}{2pt}] (O6) to[out=180,in=-150,in looseness=4,out looseness=3] (sb6a);
                \draw [curved arrow={2pt}{2pt}] (sb6b) to[bend left=90,looseness=3] (LG6);
                \draw [curved arrow={10pt}{2pt}] (LG9) to[bend left=50,looseness=1.5] (H10);
                \draw [curved arrow={2pt}{2pt}] (sb10) to[bend right=90,looseness=3] (B10);
            }
            \caption{Base-catalyzed reactivity.}
            \label{fig:acidCarboxylicb}
        \end{subfigure}
        \caption{The typical reactivity of carboxylic acid derivatives.}
        \label{fig:acidCarboxylic}
    \end{figure}
    \begin{itemize}
        \item This mode of reactivity is the one that is most typical of carboxylic acid derivatives.
        \begin{itemize}
            \item It is so-named because the portion of a carboxylic acid derivative that is not the leaving group is called an acyl group, and we are substituting one group on the acyl for another.
        \end{itemize}
        \item Think of all of the carboxylic acid derivatives (see Figure \ref{fig:carboxylicAcidDerivatives}) as containing a leaving group on one of their sides.
        \begin{itemize}
            \item When these compounds react nucleophiles, the nucleophile replaces the leaving group.
        \end{itemize}
        \item These reactions are either acid- or base-catalyzed.
        \begin{itemize}
            \item In the acid-catalyzed version (Figure \ref{fig:acidCarboxylica}), the first step proceeds exactly as in Figure \ref{fig:acidPromotedNua}, except that $\ce{R$'$}=\ce{LG}$. The second step proceeds exactly as in Figure \ref{fig:acidPromotedNub}, except that it is the leaving group that is protonated and kicked out instead of the nucleophile we just added in.
            \item The basic mechanism is related to Figure \ref{fig:basePromotedNu}, but rather than being a straight replication, the alkoxide species produced in Figure \ref{fig:basePromotedNua} proceeds straight to the reactivity of the alkoxide in Figure \ref{fig:basePromotedNub} (see Figure \ref{fig:acidCarboxylicb}).
        \end{itemize}
    \end{itemize}
    \item \textbf{Tetrahedral intermediates}: The nucleophilic acyl substitution intermediates (of both the acidic and basic pathways) that have four groups attached to the central carbon.
    \begin{figure}[h!]
        \centering
        \footnotesize
        \begin{subfigure}[b]{0.3\linewidth}
            \centering
            \chemfig{R-[:30](-[:110]HO)(-[:70]Nu)-[:-30]LG}
            \caption{Acidic intermediate.}
            \label{fig:tetrahedralIntermediatesa}
        \end{subfigure}
        \begin{subfigure}[b]{0.3\linewidth}
            \centering
            \chemfig{R-[:30](-[:110]\charge{135:1pt=$\ominus$}{O})(-[:70]Nu)-[:-30]LG}
            \caption{Basic intermediate.}
            \label{fig:tetrahedralIntermediatesb}
        \end{subfigure}
        \caption{The tetrahedral intermediates.}
        \label{fig:tetrahedralIntermediates}
    \end{figure}
    \begin{itemize}
        \item Historically, the name arose when scientists were arguing about whether or not an $sp^3$ carbon could be in this reaction. Some scientists supported the theory that these tetrahedral intermediates existed, while others disagreed.
    \end{itemize}
    \item Nitrile hydrolysis.
    \item General form.
    \begin{equation*}
        \ce{RCN + H3O+ -> RCOOH + NH4+}
    \end{equation*}
    \begin{itemize}
        \item Note that here we're using a stoichiometric full equivalent of acid, not just catalytic acid, because we are liberating ammonia which mops up our acid, forming \ce{NH4+} as a byproduct.
        \item The existence of this reaction is the reason we consider nitriles to be carboxylic acid derivatives (i.e., because we can interconvert them with carboxylic acids). 
    \end{itemize}
    \item Mechanism.
    \begin{figure}[h!]
        \centering
        \footnotesize
        \schemestart
            \chemfig{R-C~@{N1}\charge{90=\:}{N}}
            \arrow{->[\chemfig[atom sep=1.4em]{@{H2}H-[@{sb2}]@{O2}\charge{90:3pt=$\oplus$}{O}H_2}][-\ce{H2O}]}[,1.3]
            \chemfig{R-@{C3}C~[@{tb3}]@{N3}\charge{90:3pt=$\oplus$}{N}-H}
            \arrow{->[\chemfig{H_2@{O4}\charge{90=\:}{O}}]}
            \chemfig{R-[:30](-[2]@{O5}\charge{90:3pt=$\oplus$}{O}(-[@{sb5}:30]@{H5}H)(-[:150]H))=[:-30]N-[:30]H}
            \arrow{->[\chemfig{H_2@{O6}\charge{90=\:}{O}}][-\ce{H3O+}]}
            \chemleft{[}
                \subscheme{
                    \chemfig{R-[:30](-[@{sb7}2]@{O7}\charge{0=\:}{O}-[:150]H)=[@{db7}:-30]@{N7}N-[:30]H}
                    \arrow{<->}[-90]
                    \chemfig{R-[:30](=[2]\charge{45:1pt=$\oplus$}{O}-[:150]H)-[:-30]@{N8}\charge{-90:3pt=$\ominus$}{N}-[:30]H}
                }
            \chemright{]}
            \arrow{->[*{0}\setchemfig{atom sep=1.4em}\chemfig{@{H9}H-[@{sb9}]@{O9}\charge{90:3pt=$\oplus$}{O}H_2}]}[-90]
            \chemfig{R-[:30]@{C10}(=[@{db10}2]@{O10}\charge{45:1pt=$\oplus$}{O}-[:150]H)-[:-30]N(-[6]H)-[:30]H}
            \arrow{->[*{0.-90}\chemfig{H_2@{O11}\charge{90=\:}{O}}]}[180]
            \chemfig{R-[:30](-[:110]HO)(-[:70]@{O12}\charge{-70:2pt=$\oplus$}{O}H-[@{sb12}2]@{H12}H)-[:-30]NH_2}
            \arrow{->[*{0.-90}\chemfig{H_2@{O13}\charge{90=\:}{O}}][-\ce{H3O+}]}[180]
            \chemfig{R-[:30](-[:110]HO)(-[:70]OH)-[:-30]@{N14}\charge{90=\:}{N}H_2}
            \arrow{->[\chemfig[atom sep=1.4em]{@{H15}H-[@{sb15}]@{O15}\charge{90:3pt=$\oplus$}{O}H_2}][-\ce{H2O}]}[180,1.3]
            \chemfig{R-[:30](-[@{sb16a}:110]H@{O16}\charge{90=\:}{O})(-[:70]OH)-[@{sb16b}:-30]@{N16}\charge{-90:3pt=$\oplus$}{N}H_3}
            \arrow[-90]
            \subscheme{
                \chemfig{R-[:30](=[2]@{O17}\charge{90:3pt=$\oplus$}{O}-[@{sb17}:30]@{H17}H)-[:-30]OH}
                \arrow{0}[,0.1]\+
                \chemfig{@{N18}\charge{90=\:}{N}H_3}
            }
            \arrow{->[][-\ce{NH4+}]}
            \chemfig{R-[:30](=[2]O)-[:-30]OH}
        \schemestop
        \chemmove{
            \draw [curved arrow={6pt}{2pt}] (N1) to[out=90,in=90,looseness=3] (H2);
            \draw [curved arrow={2pt}{2pt}] (sb2) to[out=110,in=130,looseness=3] (O2);
            \draw [curved arrow={6pt}{2pt}] (O4) to[out=90,in=90,looseness=1.2] (C3);
            \draw [curved arrow={4pt}{2pt}] (tb3) to[bend right=90,looseness=3] (N3);
            \draw [curved arrow={6pt}{2pt}] (O6) to[out=90,in=0] (H5);
            \draw [curved arrow={2pt}{2pt}] (sb5) to[bend left=90,looseness=3] (O5);
            \draw [curved arrow={6pt}{2pt},blx] (O7) to[bend left=90,looseness=3] (sb7);
            \draw [curved arrow={3pt}{2pt},blx] (db7) to[bend right=90,looseness=3] (N7);
            \draw [curved arrow={0pt}{2pt}] ([yshift=-10pt]N8.south) to[out=-90,in=75] (H9);
            \draw [curved arrow={2pt}{2pt}] (sb9) to[bend right=90,looseness=4] (O9);
            \draw [curved arrow={6pt}{3pt}] (O11) to[out=90,in=150,looseness=1.5] (C10);
            \draw [curved arrow={3pt}{2pt}] (db10) to[bend right=90,looseness=3] (O10);
            \draw [curved arrow={6pt}{2pt}] (O13) to[out=90,in=180,looseness=1.1] (H12);
            \draw [curved arrow={2pt}{2pt}] (sb12) to[bend right=70,looseness=2.5] (O12);
            \draw [curved arrow={6pt}{3pt}] (N14) to[out=75,in=90,out looseness=2] (H15);
            \draw [curved arrow={2pt}{2pt}] (sb15) to[out=110,in=130,looseness=3] (O15);
            \draw [curved arrow={6pt}{2pt}] (O16) to[out=90,in=-150,looseness=7.5] (sb16a);
            \draw [curved arrow={2pt}{2pt}] (sb16b) to[bend left=90,looseness=3] (N16);
            \draw [curved arrow={6pt}{2pt}] (N18) to[out=90,in=0,looseness=1.1] (H17);
            \draw [curved arrow={2pt}{2pt}] (sb17) to[bend left=90,looseness=3] (O17);
        }
        \caption{Nitrile hydrolysis mechanism.}
        \label{fig:mechanismNitrileHydrolysis}
    \end{figure}
    \begin{itemize}
        \item Note that the fourth intermediate is one deprotonation away from being an amide.
        \begin{itemize}
            \item However, the reaction conditions do not produce an amide but continue as drawn to a carboxylic acid.
            \item This is because in general, the amide oxygen is more basic than the nitrile nitrogen, so if the conditions are such that the nitrile will begin the reaction, the amide will certainly finish it.
        \end{itemize}
        \item Note that there are some enzymes that can stop at the amide through various mechanisms that recognize one species as substrate but not another.
        \item Every once in a while, people will claim that they've isolated the amide in this mechanism, but these results are hard to reproduce because of the above facts.
        \item If we do add up all of the equivalents of water and acid added, we can see that only one equivalent of acid is added, overall (and two equivalents of water).
    \end{itemize}
    \item Dehydration of amides.
    \item General form.
    \begin{equation*}
        \ce{RCONH2 ->[reagents][\Delta] RCN}
    \end{equation*}
    \begin{itemize}
        \item This is the reverse reaction to nitrile hydrolysis.
        \item Reagents is either \ce{SOCl2} or \ce{POCl3}.
        \item \ce{SOCl2} and \ce{POCl3} are \textbf{dehydrating agents}.
    \end{itemize}
    \item \textbf{Dehydrating agent}: A chemical that drives conversions in which water is lost from a molecule.
    \begin{itemize}
        \item Notice how the amide overall loses two hydrogens and an oxygen (i.e., a water molecule overall) in Figure \ref{fig:mechanismAmideDehydration}.
    \end{itemize}
    \item Mechanism.
    \begin{figure}[h!]
        \centering
        \footnotesize
        \schemestart
            \chemfig{R-[:30](=[@{db1}2]O)-[@{sb1}:-30]@{N1}\charge{90=\:}{N}H_2}
            \arrow{->[\chemfig[atom sep=1.4em]{@{S2}S(=[2]O)(-[@{sb2}:-30]@{Cl2}Cl)(-[:-150]Cl)}]}[,1.5]
            \chemfig{R-[:30](-[2]O-[:30]S(=[2]O)-[:-30]Cl)=[:-30]@{N3}\charge{90:3pt=$\oplus$}{N}(-[@{sb3}6]@{H3}H)(-[:30]H)}
            \arrow{0}[,0.1]\+
            \chemfig{@{Cl4}\charge{-90=\:,45:1pt=$\ominus$}{Cl}}
            \arrow{->[][-\ce{HCl}]}
            \chemfig{R-[:30](-[@{sb5a}2]O-[@{sb5b}:30]S(=[2]O)-[@{sb5c}:-30]@{Cl5}Cl)=[@{db5}:-30]@{N5}\charge{-90=\:}{N}H}
            \arrow{->[][*{0}-\ce{SO2}]}[-90,1.5,shorten <=5mm,shorten >=3mm]
            \subscheme{
                \chemfig{R-C~@{N6}\charge{90:3pt=$\oplus$}{N}-[@{sb6}]@{H6}H}
                \arrow{0}[,0.1]\+
                \chemfig{@{Cl7}\charge{90=\:,45:1pt=$\ominus$}{Cl}}
            }
            \arrow{->[][*{0.90}-\ce{HCl}]}[180]
            \chemfig{R-C~N}
        \schemestop
        \chemmove{
            \draw [curved arrow={6pt}{2pt}] (N1) to[bend right=70,looseness=2.5] (sb1);
            \draw [curved arrow={3pt}{2pt}] (db1) to[out=10,in=150] (S2);
            \draw [curved arrow={2pt}{2pt}] (sb2) to[bend left=90,looseness=3] (Cl2);
            \draw [curved arrow={6pt}{2pt}] (Cl4) to[out=-90,in=0,looseness=1.1] (H3);
            \draw [curved arrow={2pt}{2pt}] (sb3) to[bend left=90,looseness=3] (N3);
            \draw [curved arrow={6pt}{3pt}] (N5) to[out=-90,in=-120,looseness=4] (db5);
            \draw [curved arrow={2pt}{2pt}] (sb5a) to[bend right=60,looseness=1.5] (sb5b);
            \draw [curved arrow={2pt}{2pt}] (sb5c) to[bend left=90,looseness=3] (Cl5);
            \draw [curved arrow={6pt}{2pt}] (Cl7) to[out=90,in=90,looseness=2] (H6);
            \draw [curved arrow={2pt}{2pt}] (sb6) to[bend left=90,looseness=3] (N6);
        }
        \caption{Dehydration of amides mechanism.}
        \label{fig:mechanismAmideDehydration}
    \end{figure}
    \begin{itemize}
        \item Part of the reason the amide oxygen is such a good nucleophile is because the nitrogen can participate, as in step 1 above.
        \item Driving force: Kicking out a gas (\ce{SO2}) and chloride.
        \item Note that the mechanism implies that we must have an amide with two \ce{H}'s (esp., we cannot have one or two \ce{R} groups in their place).
        \item Although only the mechanism for \ce{SOCl2} is illustrated, the mechanism is virtually identical for \ce{POCl3}.
    \end{itemize}
    \item Comparing methods 2 and 3 of synthesizing carboxylic acids.
    \begin{figure}[H]
        \centering
        \footnotesize
        \begin{tikzpicture}
            \node{\chemfig{*6(---(-Br)---)}};
            \draw (1.5,0) -- ++(1,0);
            \draw [CF-CF] (3.5,1.5) -- node[above]{\ce{KCN}} ++(-1,0) -- ++(0,-3) -- node[above]{\ce{Mg${}^\circ$}} ++(1,0);
            
            \node at (5,1.5)  {\chemfig{*6(---(-[,,,,white]\phantom{MgBr})(-CN)---)}};
            \node at (5,-1.5) {\chemfig{*6(---(-MgBr)---)}};
            \draw (6.5,1.5) -- node[above]{\ce{H3O+}} ++(1.5,0) -- ++(0,-3) -- node[above]{1. \ce{CO2}\rule{2mm}{0pt}} node[below]{2. \ce{H3O+}} ++(-1.5,0);
            \draw [-CF] (8,0) -- ++(1,0);
    
            \node at (11,0) {\chemfig{*6(---(-(=[2]O)-[:-30]OH)---)}};
        \end{tikzpicture}
        \caption{Two ways to synthesize a carboxylic acid from an alkyl halide.}
        \label{fig:2and3}
    \end{figure}
    \begin{itemize}
        \item Both carboxylation and nitrile hydrolysis achieve the same end result from the same starting material, begging the question of why both are necessary.
        \item The answer lies in the fact that both suit different types of reaction conditions.
        \item Carboxylation is strongly basic, so we can't use molecules with free \ce{H}'s.
        \item Nitrile hydrolysis proceeds through S\textsubscript{N}2 to start, so we can't use tertiary bromides.
        \begin{itemize}
            \item This is important on part of PSet 1!
        \end{itemize}
    \end{itemize}
    \item Methods of ester synthesis.
    \begin{enumerate}
        \item Nucleophilic.
        \item Fischer esterification.
    \end{enumerate}
    \item Nucleophilic.
    \item General form.
    \begin{center}
        \footnotesize
        \setchemfig{atom sep=1.4em}
        \schemestart
            \chemfig{R-[:30](=[2]O)-[:-30]OH}
            \arrow{->[\ce{K2CO3}][-\ce{KHCO3}]}[,1.3]
            \chemfig{R-[:30](=[2]\textcolor{grx}{O})-[:-30]\charge{45:1pt=$\ominus$}{\textcolor{grx}{O}}-[,0.6,,,white]\charge{45:1pt=$\oplus$}{K}}
            \arrow{->[\ce{R$'$I}]}
            \chemfig{R-[:30](=[2]\textcolor{grx}{O})-[:-30]\textcolor{grx}{O}-[:30]R'}
        \schemestop
    \end{center}
    \begin{itemize}
        \item We deprotonate the carboxylic acid using a relatively weak base.
        \begin{itemize}
            \item \ce{K2CO3} is often the weak base of choice because it's insoluble in most solvents but will react in a biphasic mixture.
            \item Additionally, since \ce{KHCO3} is usually insoluble and the carboxylate is typically soluble in the organic solvent in which the reaction is being carried out, it's really easy to separate the two.
        \end{itemize}
        \item The second step proceeds via an S\textsubscript{N}2 mechanism, so methyl or primary alkyl halides are best.
        \item Note that the two initial oxygens (green) proceed through the whole of the process and end up in the product.
    \end{itemize}
    \item Fischer esterification.
    \item General form.
    \begin{center}
        \footnotesize
        \setchemfig{atom sep=1.4em}
        \schemestart
            \chemfig{R-[:30](=[2]\textcolor{grx}{O})-[:-30]\textcolor{grx}{O}H}
            \arrow{->[\ce{H+}][\ce{R$'${\color{blx}O}H}]}
            \chemfig{R-[:30](=[2]\textcolor{grx}{O})-[:-30]\textcolor{blx}{O}-[:30]R'}
        \schemestop
    \end{center}
    \begin{itemize}
        \item The acid is a catalyst, and we need an excess of the alcohol, which we typically just use as our solvent.
        \item Reasons we need an excess of the alcohol.
        \begin{itemize}
            \item This is essentially a thermoneutral reaction; there's not a great thermodynamic driving force between the carboxylic acid and ester.
            \item Thus, the only way to get the reaction to go forward is to overwhelm it with an excess of the alcohol so that Le Ch\^{a}telier's principle comes into play.
        \end{itemize}
        \item Removing water can also help drive the reaction.
        \item \ce{H3O+} (i.e., excess water) reverses the reaction.
        \item Note that the mechanism here is a nucleophilic attack, and it is the \emph{methanol} oxygen (blue) that gets incorporated into the final ester (whose initial oxygens are colored green).
    \end{itemize}
    \item \textbf{Saponification}: Subjecting an ester to a single equivalent of \ce{KOH} (or any other hydroxide base) to form the carboxylate and the alcohol.
    \begin{itemize}
        \item This is very old chemistry.
        \item Sapon- is the Latin prefix for soap.
        \item Ancient peoples discovered that combining and heating animal fat, wood ash, and a bit of water creates soap.
        \item Combining triglycerides with pot ash yields glycerol soap and long-chain fatty acid carboxylates.
        \begin{itemize}
            \item Pot ash is where we get the name for potassium, because the ashes from a wood stove are rich in potassium hydroxide.
            \item Fatty acid carboxylates serve to solublize grease in water because the lipid end interacts with the grease and the carboxylate end interacts with the water. This is how all soaps work!
        \end{itemize}
    \end{itemize}
    \item General form.
    \begin{equation*}
        \ce{RCOOR$'$ ->[KOH] RCOOK + R$'$OH}
    \end{equation*}
    \begin{itemize}
        \item The carboxylate is an end-stage product. Resonance delocalizes the negative charge over the carbon atom, significantly decreasing its electrophilicity and hence its capacity to participate in future reactions.
        \item The presence of basic conditions make it so that this reaction is not reversible. Indeed, if we mix a base with \ce{RCOOH}, we will just deprotonate the acid and return to the carboxylate form.
    \end{itemize}
    \item Mechanism.
    \begin{itemize}
        \item Hydroxide attacks the ester as a nucleophile, and \ce{OR-} leaves to form a carboxylic acid. But \ce{OR-} (a strong base) will then deprotonate \ce{RCOOH} (a strong acid) to form the carboxylate and alcohol.
    \end{itemize}
    \item Acid chloride synthesis.
    \item General form.
    \begin{equation*}
        \ce{RCOOH ->[SOCl2][Py] RCOCl + [PyH]Cl + SO2}
    \end{equation*}
    \begin{itemize}
        \item Pyridine is not strictly necessary, but it greatly increases the reaction rate.
        \item Driven in a similar way to the dehydration of amides; we release \ce{SO2} gas, expel a water molecule, and mop up the extra \ce{Cl-} with pyridine.
    \end{itemize}
    \item Mechanism.
    \begin{figure}[h!]
        \centering
        \footnotesize
        \schemestart
            \chemfig{R-[:30](=[2]O)-[:-30]@{O1}O-[@{sb1}:30]@{H1}H}
            \arrow{->[*{0}\chemfig{@{Py2}\charge{0=\:}{Py}}][*{0}-\ce{PyH+}]}[-90]
            \chemfig{R-[:30](=[2]O)-[:-30]@{O3}\charge{90=\:,45:1pt=$\ominus$}{O}}
            \arrow{->[\chemfig[atom sep=1.4em]{@{S4}S(=[2]O)(-[@{sb4}:-30]@{Cl4}Cl)(-[:-150]Cl)}]}[,1.5]
            \chemfig{R-[:30]@{C5}(=[@{db5}2]@{O5}O)-[:-30]O-[:30]S(=[2]O)-[:-30]Cl}
            \arrow{0}[,0.1]\+
            \chemfig{@{Cl6}\charge{90=\:,45:1pt=$\ominus$}{Cl}}
            \arrow
            \chemfig{R-[:30](-[@{sb7a}:110]@{O7}\charge{180=\:,135:1pt=$\ominus$}{O})(-[:70]Cl)-[@{sb7b}:-30]O-[@{sb7c}:30]S(=[2]O)-[@{sb7d}:-30]@{Cl7}Cl}
            \arrow{->[][-\ce{SO2, Cl-}]}[,1.3]
            \chemfig{R-[:30](=[2]O)-[:-30]Cl}
        \schemestop
        \chemmove{
            \draw [curved arrow={6pt}{2pt}] (Py2) to[out=0,in=-90] (H1);
            \draw [curved arrow={2pt}{2pt}] (sb1) to[bend right=90,looseness=3] (O1);
            \draw [curved arrow={6pt}{2pt}] (O3) to[out=90,in=150,looseness=1.5] (S4);
            \draw [curved arrow={2pt}{2pt}] (sb4) to[bend left=90,looseness=3] (Cl4);
            \draw [curved arrow={6pt}{3pt}] (Cl6) to[out=100,in=50,out looseness=2] (C5);
            \draw [curved arrow={3pt}{2pt}] (db5) to[bend left=90,looseness=3] (O5);
            \draw [curved arrow={6pt}{3pt}] (O7) to[out=180,in=-150,looseness=4] (sb7a);
            \draw [curved arrow={2pt}{2pt}] (sb7b) to[bend left=60,looseness=1.5] (sb7c);
            \draw [curved arrow={2pt}{2pt}] (sb7d) to[bend left=90,looseness=3] (Cl7);
        }
        \caption{Acid chloride synthesis mechanism.}
        \label{fig:mechanismAcidChloride}
    \end{figure}
    \begin{itemize}
        \item Since chloride is a fairly week nucleophile, it's addition in step 3 takes a while and is reversible.
        \begin{itemize}
            \item However, this step is driven in the forward direction by releasing \ce{SO2} gas from the resulting tetrahedral intermediate (Le Ch\^{a}telier's principle).
        \end{itemize}
    \end{itemize}
    \item Anhydride synthesis.
    \item General form (standard).
    \begin{equation*}
        \ce{RCOOH ->[\Delta][{[-H2O]}] RCOOCOR}
    \end{equation*}
    \begin{itemize}
        \item High heat is required.
        \item If you use two different carboxylic acids, you will get a statistical mixture (no real selectivity).
    \end{itemize}
    \item You can selectively create 5-6 membered rings containing anhydrides because this reaction proceeds intramolecularly as well as intramolecularly.
    \item General form (intramolecular).
    \begin{center}
        \footnotesize
        \setchemfig{atom sep=1.4em}
        \schemestart
            \chemfig{[4]*6(OH-(=O)--(-[:-170])(-[:-130])-(=O)-OH)}
            \arrow{->[$\Delta$][$[-\ce{H2O}]$]}[,1.2]
            \chemfig{*5((-[:-164])(-[:-124])-(=O)-O-(=O)--)}
        \schemestop
    \end{center}
    \begin{itemize}
        \item In particular, if you have a single molecule with two different carboxylic acid groups 2-3 carbons apart, then heating a sample of said molecule while removing water will result in a ring-closing anhydridization.
        \item If we want to make a ring with another number of carbons, we should go through acid chlorides (see below).
    \end{itemize}
    \item A way to selectively create anhydrides is via acid chlorides and sodium carboxylates.
    \item Mixed anhydride synthesis.
    \item General form.
    \begin{center}
        \footnotesize
        \setchemfig{atom sep=1.4em}
        \schemestart
            \chemfig{R-[:30](=[2]O)-[:-30]Cl}
            \arrow{0}[,0.1]\+{,,1.5em}
            \chemfig{R'-[:30](=[2]O)-[:-30]\charge{45:1pt=$\ominus$}{O}-[,0.6,,,white]\charge{45:1pt=$\oplus$}{Na}}
            \arrow
            \chemfig{R-[:30](=[2]O)-[:-30]O-[:30](=[2]O)-[:-30]R'}
            \arrow{0}[,0.1]\+
            \chemfig{NaCl}
        \schemestop
    \end{center}
    \begin{itemize}
        \item This reaction proceeds via nucleophilic substitution.
    \end{itemize}
    \item Amide synthesis.
    \item General form.
    \begin{equation*}
        \ce{RCOOH + NHR$'$R$''$ ->[DCC][Py] RCONR$'$R$''$}
    \end{equation*}
    \item Mechanism.
    \begin{figure}[h!]
        \centering
        \vspace{1em}
        \footnotesize
        \schemestart
            \chemfig{R-[:30](=[2]O)-[:-30]@{O1}O-[@{sb1}:30]@{H1}H}
            \arrow{->[\chemfig{@{Py2}\charge{90=\:}{Py}}]}
            \chemfig{R-[:30](=[2]O)-[:-30]@{O3}\charge{90=\:,45:1pt=$\ominus$}{O}}
            \arrow{0}[,0.1]\+
            \chemfig{\charge{90:3pt=$\oplus$}{Py}H}
            \arrow{->[\chemfig[atom sep=1.4em]{CyN=@{C5}C=[@{db5}]@{N5}NCy}]}[,2]
            \subscheme{
                \chemfig{R-[:30](=[2]O)-[:-30]O-[:30](-[2]@{N6}\charge{90=\:,135:3pt=$\ominus$}{N}Cy)(=[:-30]NCy)}
                \arrow{0}[,0.1]\+
                \chemfig{@{H7}H-[@{sb7}]@{Py7}\charge{90:3pt=$\oplus$}{Py}}
            }
            \arrow{->[][*{0}-\ce{Py}]}[-90]
            \chemfig{R-[:30]@{C8}(=[@{db8}2]@{O8}O)-[:-30]O-[:30](-[2]NHCy)(=[:-30]NCy)}
            \arrow{->[*{0.-90}\setchemfig{atom sep=1.4em}\chemfig{@{N9}\charge{90=\:}{N}HR'R''}]}[180,1.3]
            \chemfig[atom sep=2.5em]{[:120]*6(@{N10a}\charge{[extra sep=1.5pt]-135=\:}{N}Cy=(-[,0.8]NHCy)-O-(-[:130,0.8]\charge{135:1pt=$\ominus$}{O})(-[:170,0.8]R)-@{N10b}\charge{45:1pt=$\oplus$}{N}(-[:-170,0.8]R')(-[:-130,0.8]R'')-[@{sb10}]@{H10}H)}
            \arrow[180]
            \chemfig[atom sep=2.5em]{[:120]*6(@{N11}\charge{-90:3pt=$\oplus$}{N}HCy=[@{db11}](-[,0.8]NHCy)-[@{sb11a}]O-[@{sb11b}](-[@{sb11c}:130,0.8]@{O11}\charge{90=\:,135:1pt=$\ominus$}{O})(-[:170,0.8]R)-N(-[,0.8]R'')-R')}
            \arrow{->[][*{0}-\ce{DCU}]}[-90]
            \chemfig{R-[:30](=[2]O)-[:-30]N(-[6]R'')-[:30]R'}
        \schemestop
        \chemmove{
            \draw [curved arrow={6pt}{2pt}] (Py2) to[out=90,in=90,looseness=2] (H1);
            \draw [curved arrow={2pt}{2pt}] (sb1) to[bend right=90,looseness=3] (O1);
            \draw [curved arrow={6pt}{2pt}] (O3) to[out=90,in=90,looseness=1.3] (C5);
            \draw [curved arrow={3pt}{2pt}] (db5) to[bend left=90,looseness=4] (N5);
            \draw [curved arrow={6pt}{2pt}] (N6) to[out=90,in=90,looseness=1.3] (H7);
            \draw [curved arrow={2pt}{2pt}] (sb7) to[bend right=90,looseness=3] (Py7);
            \draw [curved arrow={6pt}{3pt}] (N9) to[out=90,in=150] (C8);
            \draw [curved arrow={3pt}{2pt}] (db8) to[bend right=90,looseness=3] (O8);
            \draw [curved arrow={6pt}{2pt}] (N10a) to[bend left=20,looseness=1] (H10);
            \draw [curved arrow={2pt}{2pt}] (sb10) to[bend left=70,looseness=2.5] (N10b);
            \draw [curved arrow={6pt}{2pt}] (O11) to[out=90,in=40,looseness=4] (sb11c);
            \draw [curved arrow={2pt}{2pt}] (sb11b) to[bend right=60,looseness=1.3] (sb11a);
            \draw [curved arrow={3pt}{2pt}] (db11) to[bend right=90,looseness=3] (N11);
        }
        \caption{Amide synthesis mechanism.}
        \label{fig:mechanismAmide}
    \end{figure}
    \begin{itemize}
        \item Note that as in other mechanisms, DCC eventually transforms into a type of leaving group.
        \item Normally, we use external reagents for proton transfers because doing an internal one would in most cases involve a transition state with a 4-membered ring, which is highly strained.
        \begin{itemize}
            \item However, in step 5 here, we can do an internal proton transfer because the transition state's conformation is that of a 6-membered ring.
        \end{itemize}
    \end{itemize}
    \item \textbf{DCC}: Dicyclohexylcarbodiimide, a dehydrating reagent key to amide synthesis. \emph{Structure}
    \begin{figure}[H]
        \centering
        \footnotesize
        \chemfig{C(=[:30]N-[::-60]*6(------))(=[:-150]N-[::-60]*6(------))}
        \caption{Dicyclohexylcarbodiimide (DCC).}
        \label{fig:DCC}
    \end{figure}
    \item DCC reacts with water as follows.
    \begin{figure}[H]
        \centering
        \footnotesize
        \schemestart
            \chemfig{C(=[:30]N-[::-60]*6(------))(=[:-150]N-[::-60]*6(------))}
            \arrow{->[\ce{H2O}]}
            \chemname{\chemfig{(-[:-30]N(-[6]H)-[:30]*6(------))(-[:-150]N(-[6]H)-[:150]*6(------))(=[2]O)}}{Dicyclohexylurea}
        \schemestop
        \caption{DCC and water.}
        \label{fig:DCCH2O}
    \end{figure}
    \item \textbf{DCU}: Dicyclohexylurea, the product of the reaction of DCC and water.
    \item Reactivity scale.
    \begin{equation*}
        \text{acid chloride} > \text{anhydride}
        > \text{ester}
        > \text{amide}
        > \text{carboxylate}
    \end{equation*}
    \begin{itemize}
        \item It should make intuitive sense that acid chlorides are the most reactive carboxylic acid derivatives and carboxylates are the least.
        \begin{itemize}
            \item Acid chlorides have an electronegative group on the already electrophilic carbon, exacerbating the molecular dipole.
            \item Carboxylates delocalize their negative charge over the carbon (as discussed earlier), greatly reducing or eliminating the molecular dipole.
            \item A good rule of thumb is that the compound with the best leaving group and worst nucleophile (an acid chloride) is the most reactive, and vice versa in that the compound with the worst leaving group and the best nucleophile (a carboxylate) is the most reactive.
        \end{itemize}
        \item What we mean by "reactivity" is that compounds higher on the reactive scale can react with an appropriate nucleophile to become compounds lower on the scale.
        \begin{itemize}
            \item For instance, we can take an acid chloride to an anhydride, ester, amide, or carboxylate (and we have reactions to do that), but we cannot take all (or any) of these molecules back to an acid chloride without forcing conditions.
            \item Some things that qualify as forcing conditions are the use of acidic conditions and dehydrating reagents.
            \item In other words, this reactivity scale is for the compounds in basic media with no dehydrating reagents present.
        \end{itemize}
    \end{itemize}
    \item MCAT comments.
    \item Trialkyl amines and pyridines.
    \begin{itemize}
        \item According to our reactivity scale, we should be able to react \ce{NEt3} with \ce{RCOCl} to yield an amine, for example.
        \begin{itemize}
            \item However, this leads to a positively charged nitrogen in the amine that cannot be quenched (e.g., by deprotonation). Thus, this is a highly reversible reaction that favors the reactants.
        \end{itemize}
        \item Similarly, we should be able to react an anhydride with pyridine.
        \begin{itemize}
            \item But since pyridine cannot be deprotonated either, the reactants are favored in this reversible reaction once again.
        \end{itemize}
    \end{itemize}
    \item However, this implies that pyridines can be used to catalyze nucleophilic acyl substitutions.
    \item \textbf{DMAP}: Dimethylaminopyridine, which is one of the best catalysts for nucleophilic acyl substitutions. \emph{Structure}
    \begin{figure}[h!]
        \centering
        \footnotesize
        \chemfig{[:30]**6(--(-N(-[:60])(-[:-60]))---N-)}
        \caption{Dimethylaminopyridine (DMAP).}
        \label{fig:DMAP}
    \end{figure}
    \begin{itemize}
        \item Levin gives an example synthesis using DMAP, namely nucleophilic addition to an anhydride.
        \begin{itemize}
            \item In essence, DMAP adds to the carbonyl, kicks out the leaving group, and then the nucleophile adds to the carbonyl and kicks out DMAP.
        \end{itemize}
        \item Adding DMAP can accelerate a reaction that would take overnight to taking only a few minutes.
    \end{itemize}
    \item Acid chlorides, anhydrides, and esters all create the same product (an amide) when reacting with an amine.
    \begin{itemize}
        \item But, you need only one equivalent of the amine for esters while you need two equivalents for the first two.
        \item This is because of the $\pKa$'s. The first two byproducts (\ce{HCl} and \ce{RCOOH}) protonate amines in solution, whereas \ce{ROH} does not (as much).
    \end{itemize}
\end{itemize}



\section{Discussion Section}
\begin{itemize}
    \item \marginnote{4/8:}We will be working with hot sand baths in the next lab, so just leave them to cool and do not dispose of the contents unless you're sure they're cool.
    \item Practice problems.
    \begin{enumerate}
        \item ${\color{white}hi}$
        \begin{center}
            \footnotesize
            \setchemfig{atom sep=1.4em}
            \schemestart
                \chemfig{*4(--(-(=[::60]O)-[::-60]OEt)--)}
                \arrow{->[\ce{NaOH}]}[,1.1]
                \color{rex}
                \chemfig{*4(--(-(=[::60]O)-[::-60]\charge{45:1pt=$\ominus$}{O})--)}
            \schemestop
        \end{center}
        \begin{itemize}
            \item We form a \ce{COO-} ion instead of the carboxylic acid because we are in basic solution.
            \item The mechanism is a nucleophilic attack on the carbonyl, the oxygen electrons swinging back down and kicking out \ce{EtO-}, and then deprotonation of the acid.
        \end{itemize}
        \item ${\color{white}hi}$
        \begin{center}
            \footnotesize
            \setchemfig{atom sep=1.4em}
            \schemestart
                \chemfig{-[:-30]-[:30](=[2]O)-[:-30]H}
                \arrow{->[\color{rex}1. \ce{CrO3, H2SO4, H2O}][\color{rex}2. \ce{NH3, DCC}\rule{1.1cm}{0pt}]}[,2.5]
                \chemfig{-[:-30]-[:30](=[2]O)-[:-30]NH_2}
            \schemestop
        \end{center}
        \begin{itemize}
            \item The intermediate after step 1 is the carboxylic acid, as we have used aqueous Jones reagent.
        \end{itemize}
        \item ${\color{white}hi}$
        \begin{center}
            \footnotesize
            \setchemfig{atom sep=1.4em}
            \schemestart
                \chemfig{EtO-[:30](=[2]O)-[:-30]-[:30](=[2]O)-[:-30]OH}
                \arrow{->[\ce{MeOH}][\ce{H2SO4}]}[,1.2]
                \color{rex}
                \chemfig{MeO-[:30](-[2]OH)-[:-30]-[:30](=[2]O)-[:-30]OMe}
            \schemestop
        \end{center}
        \begin{itemize}
            \item The reaction of the ester (left) is called \textbf{transesterification}; the reaction of the carboxylic acid (right) is called ether formation.
            \item It's important to know that you can get ester formation in both of these cases.
            \item This is a common problematic side reaction in synthetic chemistry.
            \item Mechanism: Methanol attacks each carbonyl, the other group leaves, and then deprotonation.
        \end{itemize}
        \item ${\color{white}hi}$
        \begin{center}
            \footnotesize
            \setchemfig{atom sep=1.4em}
            \schemestart
                \chemfig{-[:-30](-[6])-[:30](=[2]O)-[:-30]}
                \arrow{0}[,0.1]\+{,,1em}
                \chemfig{[:18]*5(---\chemabove{N}{H}--)}
                \arrow{->[\ce{H3O+}]}
                \color{rex}
                \chemfig{-[:-30](-[6])=_[:30](-[2]N*5(-----))-[:-30]}
            \schemestop
        \end{center}
        \begin{itemize}
            \item We choose this enamine as the major product by Zaitsev's rule.
        \end{itemize}
        \item ${\color{white}hi}$
        \begin{center}
            \footnotesize
            \setchemfig{atom sep=1.4em}
            \schemestart
                \chemfig{H-[:30](=[2]O)-[:-30]-[:30]-[:-30](=[6]O)-[:30]}
                \arrow{->[\chemfig[atom sep=1.4em]{HO-[:30]-[:-30]-[:30]OH}][\ce{H3O+}]}[,1.8]
                \color{rex}
                \chemfig{H-[:30](-[2,0.01]*5(-O---O-))-[:-30]-[:30]-[:-30](=[6]O)-[:30]}
                \color{black}
                \arrow{->[1. \ce{MeLi}\rule{1.5mm}{0pt}][2. \ce{H3O+}]}[,1.3]
                \color{rex}
                \chemfig{H-[:30](=[2]O)-[:-30]-[:30]-[:-30](-[:-70]OH)(-[:-110])-[:30]}
            \schemestop
        \end{center}
        \item ${\color{white}hi}$
        \begin{center}
            \footnotesize
            \setchemfig{atom sep=1.4em}
            \schemestart
                \chemfig{-[:30]*6(----(-OH)-(-)-)}
                \arrow(R--P1){->[1. PCC\rule{8.5mm}{0pt}][2. \chemfig[atom sep=1.4em]{-[:30]=_[:-30]PPh_3}]}[,1.7]
                \color{rex}
                \chemfig{-[:30]*6(----(=_-[:30])-(-)-)}
                \color{black}
                \arrow(@R--P2){->[*{0.-90}\ce{H2SO4}]}[180,1.2]
                \color{rex}
                \chemfig{-[:30]*6(-----(-)=)}
            \schemestop
        \end{center}
        \item ${\color{white}hi}$
        \begin{center}
            \footnotesize
            \setchemfig{atom sep=1.4em}
            \schemestart
                \chemfig{-[:-30](=[6])-[:30](=[2]O)-[:-30]}
                \arrow{->[\ce{Me2CuLi}]}[,1.3]
                \color{rex}
                \chemfig{-[:-30](-[6]-[:-150])-[:30](=[2]O)-[:-30]}
                \color{black}
                \arrow{->[1. \ce{MeLi}\rule{1.5mm}{0pt}][2. \ce{H3O+}]}[,1.3]
                \color{rex}
                \chemfig{-[:-30](-[6]-[:-150])-[:30](-[:110])(-[:70]OH)-[:-30]}
            \schemestop
        \end{center}
    \end{enumerate}
\end{itemize}



\section{Office Hours (Levin)}
\begin{itemize}
    \item \marginnote{4/11:}$\alpha,\beta$-unsaturated carbonyls?
    \begin{figure}[h!]
        \centering
        \footnotesize
        \begin{subfigure}[b]{\linewidth}
            \centering
            \schemestart
                \chemfig{-[:30](=[@{db1a}2]O)-[@{sb1}:-30]=_[@{db1b}6]@{C1}}
                \arrow{->[\chemfig[atom sep=1.4em]{@{H2}H-[@{sb2}]@{O2}OMe}][\chemfig[atom sep=1.4em]{\charge{45:1pt=$\oplus$}{Na}-[,0.6,,,white]H-[@{sb3}]\charge{90:3pt=$\ominus$}{B}H_3-[2,0.6,,,opacity=0]}]}[,1.7]
                \chemfig{-[:30](-[2]OH)=_[:-30]-[6]}
                \arrow
                \chemfig{-[:30](=[2]O)-[:-30]-[6]}
                \arrow{->[\ce{NaBH4}][\ce{MeOH}]}[,1.2]
                \chemname{\chemfig{-[:30](-[2]OH)-[:-30]=_[6]}}{50\%}
                \arrow{0}[,0.1]\+
                \chemname{\chemfig{-[:30](-[2]OH)-[:-30]-[6]}}{50\%}
            \schemestop
            \chemmove{
                \draw [curved arrow={2pt}{2pt}] (sb3) to[out=-90,in=-30] (C1);
                \draw [curved arrow={4pt}{2pt}] (db1b) to[bend left=60,looseness=2] (sb1);
                \draw [curved arrow={3pt}{2pt}] (db1a) to[bend left=30] (H2);
                \draw [curved arrow={2pt}{2pt}] (sb2) to[bend left=90,looseness=4] (O2);
            }
            \caption{\ce{NaBH4}.}
            \label{fig:alphaBetaReductiona}
        \end{subfigure}\\[2em]
        \begin{subfigure}[b]{\linewidth}
            \centering
            \schemestart
                \chemfig{-[:30](=[@{db1a}2]@{O1}O)-[@{sb1}:-30]=_[@{db1b}6]@{C1}}
                \arrow{->[][\chemfig[atom sep=1.4em]{\charge{45:1pt=$\oplus$}{Li}-[,0.6,,,white]H-[@{sb2}]\charge{90:3pt=$\ominus$}{Al}H_3-[2,0.6,,,opacity=0]}]}[,1.7]
                \chemfig{-[:30](-[2]O-[:30]\charge{90:3pt=$\ominus$}{Al}H_3)=_[:-30]-[6]}
                \arrow{->[\ce{H3O+}]}
                \chemname{\chemfig{-[:30](-[2]OH)-[:-30]=_[6]}}{Major}
                \arrow{0}[,0.1]\+
                \chemname{\chemfig{-[:30](=[2]O)-[:-30]-[6]}}{Minor}
            \schemestop
            \chemmove{
                \draw [curved arrow={2pt}{2pt}] (sb2) to[out=-90,in=-30] (C1);
                \draw [curved arrow={4pt}{2pt}] (db1b) to[bend left=60,looseness=2] (sb1);
                \draw [curved arrow={3pt}{2pt}] (db1a) to[bend right=90,looseness=3] (O1);
            }
            \caption{\ce{LiAlH4}.}
            \label{fig:alphaBetaReductionb}
        \end{subfigure}
        \caption{Reduction of $\alpha,\beta$ unsaturated compounds.}
        \label{fig:alphaBetaReduction}
    \end{figure}
    \begin{itemize}
        \item Levin's predictions basically line up with those from \textcite{bib:CHEM22100Notes}, although he has a different way of deriving them.
        \item The 1,2-reduction product is the same in both. But for the \ce{NaBH4}, you get full reduction as the other major byproduct.
        \item We will never be asked to use this reaction synthetically because it is not selective.
        \item We're most likely to encounter alkyllithiums or cuprates. The thing to keep in mind with the messy ones is that they're messy. We're more just interested in introducing enolate chemistry with these.
    \end{itemize}
    \item Problem Set 1, Question 3a: We form one bond with the best stereochemistry and then do an S\textsubscript{N}2 to simultaneously form the epoxide and kick out \ce{SPh2}.
    \item Problem Set 1, Question 2f: Cyclic systems are one of the only places you see hemi-acetals.
    \item Problem Set 1, Question 3b: The transition state has too much ring strain, so show proton transfers as being mediated by solvent molecules.
    \item n-butyl lithium stands for "normal"-butyl lithium; s-butyl lithium is sec-butyl lithium.
\end{itemize}




\end{document}