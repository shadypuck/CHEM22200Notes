\documentclass[../notes.tex]{subfiles}

\pagestyle{main}
\renewcommand{\chaptermark}[1]{\markboth{\chaptername\ \thechapter\ (#1)}{}}
\setcounter{chapter}{3}

\begin{document}




\chapter{Exam and Enol(ate) Reactivity}
\section{Problem Session}
\begin{itemize}
    \item \marginnote{4/19:}Practice problems.
    \begin{enumerate}
        \item ${\color{white}hi}$
        \begin{center}
            \footnotesize
            \setchemfig{atom sep=1.4em}
            \schemestart
                \chemfig{H_2N-[:30](=[2]O)-[:-30]-[:30](=[2]O)-[:-30]OH}
                \arrow{->[\begin{tabular}{l}
                    1. \ce{SOCl2}\\
                    2. \ce{NaN3}\\
                    3. \ce{NaOH}\\
                \end{tabular}]}[,1.4]
                \color{orx}
                \chemfig{NC-[:-30]-[:30](=[2]O)-[:-30]Cl}
                \arrow
                \color{orx}
                \chemfig{NC-[:-30]-[:30]N=C=O}
                \arrow
                \color{rex}
                \chemfig{H_2N-[:30](=[2]O)-[:-30]-[:30]NH_2}
            \schemestop
        \end{center}
        \item ${\color{white}hi}$
        \begin{center}
            \footnotesize
            \setchemfig{atom sep=1.4em}
            \schemestart
                \chemfig{NC-[:-30]-[:30]\charge{90:3pt=$\oplus$}{P}Ph_3}
                \arrow{->[\begin{tabular}{l}
                    1. \ce{KO^{$t$}Bu}\\
                    2. \ce{EtCOH}\\
                \end{tabular}]}[,1.4]
                \color{rex}
                \chemfig{-[:-30]-[:30](=[2](-[:30]CN)(-[:150]H))-[:-30]H}
            \schemestop
        \end{center}
        \begin{itemize}
            \item This reaction involves a stabilized ylide, hence the formation of the \emph{trans} product.
        \end{itemize}
        \item ${\color{white}hi}$
        \begin{center}
            \footnotesize
            \setchemfig{atom sep=1.4em}
            \schemestart
                \chemfig{**6(-N--(-(=[2]O)-[:-30]OH)---)}
                \arrow{->[\color{rex}DCC / \ce{HNEt2}]}[,1.8]
                \chemfig{**6(-N--(-(=[2]O)-[:-30]NEt_2)---)}
            \schemestop
        \end{center}
        \begin{itemize}
            \item You could add catalytic amounts of pyridine, DMAP, or any other nonnucleophilic source of nitrogen to speed up this reaction.
        \end{itemize}
        \item ${\color{white}hi}$
        \begin{center}
            \footnotesize
            \setchemfig{atom sep=1.4em}
            \schemestart
                \chemfig{*6(--=---)}
                \arrow{->[\color{rex}\begin{tabular}{l}
                    1. \ce{O3}\\
                    2. \ce{Me2S}\\
                    3. \ce{H2NNH2} / \ce{KOH} / $\Delta$\\
                \end{tabular}]}[,2.7]
                \chemfig{-[:30]-[:-30]-[:30]-[:-30]-[:30]}
            \schemestop
        \end{center}
        \begin{itemize}
            \item In an exam setting, we won't be charged with knowing that we need heat.
        \end{itemize}
        \item ${\color{white}hi}$
        \begin{center}
            \footnotesize
            \setchemfig{atom sep=1.4em}
            \schemestart
                \chemfig{*6(----(=O)-(-)-)}
                \arrow{->[\begin{tabular}{l}
                    1. mCPBA\\
                    2. \ce{H3O+}\\
                \end{tabular}]}[,1.5]
                \color{orx}
                \chemfig{[:-12.86]*7((-)-----(=O)-O-)}
                \arrow
                \color{rex}
                \chemfig{-[:30](-[2]OH)-[:-30]-[:30]-[:-30]-[:30]-[:-30](=[6]O)-[:30]OH}
            \schemestop
        \end{center}
        \begin{itemize}
            \item The second step proceeds as a consequence of the acid \ce{H-OH2} to a carboxylic acid derivative, as per Figure \ref{fig:acidCarboxylica}.
        \end{itemize}
        \item ${\color{white}hi}$
        \begin{center}
            \footnotesize
            \setchemfig{atom sep=1.4em}
            \schemestart
                \chemfig{H-[:30](=[2]O)-[:-30]-[:30]-[:-30]-[:30]=_[:-30]}
                \arrow{->[\begin{tabular}{l}
                    1. \ce{OsO4}\\
                    2. \ce{H+} $[-\ce{H2O}]$\\
                \end{tabular}]}[,1.8]
                \color{rex}
                \chemfig{[:-12.86]*7(--(-[:120,0.9]O?)--O-?--)}
            \schemestop
        \end{center}
        \item Mechanism: Goes over the Curtius rearrangement.
        \item Retrosynthesis.
        \begin{center}
            \footnotesize
            \setchemfig{atom sep=1.4em}
            \schemestart
                \chemfig{-[:30]-[:-30]-[:30]-[:-30]Br}
                \arrow{->[*{0} {\color{rex}\chemfig[atom sep=1.2em]{MgBr-~-H}}]}[-90]
                \color{rex}
                \chemfig{-[:-30]-[:30]-[:-30]-[:30]~[:30]}
                \arrow{->[\color{rex}\ce{Ph3PAu+}][\color{rex}\ce{H3O+}]}[,1.3]
                \color{rex}
                \chemfig{-[:-30]-[:30]-[:-30]-[:30](=[2]O)-[:-30]}
                \arrow{->[\color{rex}\chemfig[atom sep=1.2em]{Ph3P=_[2]-[:30]CO2Et}]}[,1.5]
                \color{rex}
                \chemfig{-[:-30]-[:30]-[:-30]-[:30](=_[2]-[:30](=[2]O)-[:-30]OEt)-[:-30]}
                \arrow{->[\color{rex}DIBAL-D][\color{rex}\ce{H3O+}]}[,1.3]
                \chemfig{-[:-30]-[:30]-[:-30]-[:30](=_[2]-[:30](=[2]O)-[:-30]D)-[:-30]}
            \schemestop
        \end{center}
        \begin{itemize}
            \item That deuterated aldehyde should indicate DIBAL-D.
            \item \ce{COOEt} is an EWG, and we will get the desired trans product in a Wittig with it.
        \end{itemize}
        \item Retrosynthesis.
        \begin{center}
            \footnotesize
            \setchemfig{atom sep=1.4em}
            \schemestart
                \chemfig{[:18]*5(---(<Br)--)}
                \arrow{->[\color{rex}\ce{NaCN}]}
                \color{rex}\chemfig{[:18]*5(---(<:CN)--)}
                \arrow{->[\color{rex}\ce{MeLi}][\color{rex}\ce{H3O+}]}
                \color{rex}\chemfig{[:18]*5(---(<:(-[:30])(=_[:150]O))--)}
                \arrow{->[\color{rex}\ce{HN3}]}
                \chemfig{[:18]*5(---(<:N(-[:150]H)(-[:30](=[2]O)-[:-30]))--)}
            \schemestop
        \end{center}
        \item Retrosynthesis.
        \begin{center}
            \footnotesize
            \setchemfig{atom sep=1.4em}
            \schemestart
                \chemfig{**6(---(-Br)---)}
                \arrow{->[\color{rex}\ce{Mg${}^\circ$}]}
                \color{rex}\chemfig{**6(---(-MgBr)---)}
                \arrow{->[\color{rex}\begin{tabular}{l}
                    1. \ce{CO2}\\
                    2. \ce{H3O+}\\
                \end{tabular}]}[,1.3]
                \color{rex}\chemfig{**6(---(-(=[2]O)-[:-30]OH)---)}
                \arrow{->[\color{rex}\ce{DCC, NH3}][\color{rex}cat. \ce{Py}]}[,1.4]
                \color{rex}\chemfig{**6(---(-(=[2]O)-[:-30]NH_2)---)}
                \arrow{->[\color{rex}\begin{tabular}{l}
                    1. \ce{LiAlH4}\\
                    2. \ce{H3O+}\\
                \end{tabular}]}[,1.4]
                \chemfig{**6(---(--[:-30]NH_2)---)}
            \schemestop
        \end{center}
        \begin{itemize}
            \item We will get credit if our synthesis is right even if it is not the most efficient.
        \end{itemize}
    \end{enumerate}
    \item For mechanism questions, if we're struggling, think back to the sentence trick from the very beginning of the course.
    \item For synthesis questions, just throw as many reactions out there as we can think of.
\end{itemize}
\newpage



\section{Midterm 1 Review Sheet}
\subsection*{Reactions}
\subsubsection*{Carbonyl Synthesis}
\begin{enumerate}
    \footnotesize
    \setchemfig{atom sep=1.4em}
    \item \marginnote{4/20:}
        \schemestart
            \chemfig{R-[:30]-[2]OH}
            \arrow{->[PCC]}
            \chemfig{R-[:30](=[2]O)-[:-30]H}
        \schemestop
    \item 
        \schemestart
            \chemfig{MeO-[:30]*6(=-=-=-)}
            \arrow{0}[,0.1]\+{,,1.6em}
            \chemfig{Cl-[:30](=[2]O)-[:-30]}
            \arrow{->[\ce{AlCl3}]}[,1.1]
            \chemfig{MeO-[:30]*6(=-(-[,,,,white]-[6,,,,white]\phantom{O})=(-(=[2]O)-[:-30])-=-)}
        \schemestop
    \begin{itemize}[label={--}]
        \item Recall ortho/para selectivity.
    \end{itemize}
    \item 
        \schemestart
            \chemfig{-[:30]=_[:-30]-[:30]}
            \arrow{->[1. \ce{O3}\rule{3.4mm}{0pt}][2. \ce{Me2S}]}[,1.3]
            \chemfig{O=[4](-[::60]H)(-[::-60])}
            \+
            \chemfig{O=(-[::60]H)(-[::-60])}
            \arrow{0}[,0.1]\+{,,0.8em}
            \chemfig{S(=[2]O)(-[:-30])(-[:-150])}
        \schemestop
    \begin{itemize}[label={--}]
        \item Not adding \ce{Me2S} traps the ozonide intermediate.
    \end{itemize}
    \item 
        \schemestart
            \chemfig{[:30]*6(---(<OH)-(<HO)--)}
            \arrow{->[\ce{HIO4}]}
            \chemfig{H-[:30](=[2]O)-[:-30]-[:30]-[:-30]-[:30]-[:-30](=[6]O)-[:30]H}
        \schemestop
    \begin{itemize}[label={--}]
        \item \emph{cis}-diols react faster, but are not required.
    \end{itemize}
    \item 
        \schemestart
            \chemfig{R-~-H}
            \arrow{->[\ce{Ph3PAu+}][\ce{H2O}]}[,1.3]
            \chemfig{R-[:30](=[2]O)-[:-30](-[:-70]H)(-[:-110]H)-[:30]H}
        \schemestop
    \item 
        \schemestart
            \chemfig{R-~-H}
            \arrow{->[1. 9-BBN-H\rule{3.3mm}{0pt}][2. \ce{H2O2, HO-}]}[,1.8]
            \chemfig{R-[:30](-[:70]H)(-[:110]H)-[:-30](=[6]O)-[:30]H}
        \schemestop
\end{enumerate}

\subsubsection*{Nucleophilic Addition to Carbonyls}
\textbf{General}
\begin{enumerate}
    \footnotesize
    \setchemfig{atom sep=1.4em}
    \item 
        \schemestart
            \chemfig{R-[:30](=[2]O)-[:-30]R'}
            \arrow{0}[,0.1]\+
            \chemfig{NuH}
            \arrow{<=>[acid or][base]}[,1.1]
            \chemfig{R-[:30](-[:70]Nu)(-[:110]HO)-[:-30]R'}
        \schemestop
\end{enumerate}
\textbf{Oxygen Nucleophiles}
\begin{enumerate}
    \footnotesize
    \setchemfig{atom sep=1.4em}
    \item 
        \schemestart
            \chemfig{R-[:30](=[2]O)-[:-30]R'}
            \arrow{0}[,0.1]\+
            \chemfig{H_2O}
            \arrow{<=>[reagents]}[,1.3]
            \chemfig{R-[:30](-[:70]OH)(-[:110]HO)-[:-30]R'}
        \schemestop
    \begin{itemize}[label={--}]
        \item Reagents is either \ce{H3O+} or \ce{OH-}.
        \item Proceeds faster for carbonyls with less bulky, more electron withdrawing substituents.
    \end{itemize}
    \item 
        \schemestart
            \chemfig{-[:30](=[2]O)-[:-30]}
            \arrow{0}[,0.1]\+
            2 \chemfig{MeOH}
            \arrow{->[\ce{H+}][$[-\ce{H2O}]$]}[,1.2]
            \chemfig{-[:30](-[:70]OMe)(-[:110]MeO)-[:-30]}
        \schemestop
    \begin{itemize}[label={--}]
        \item Remove water with a Dean-Stark apparatus (and toluene) or \SI{3}{\angstrom} aluminosilicates.
        \item Will not work in basic conditions (will get stuck at the hemiketal). \ce{OH-} is not a good enough leaving group and needs an acid to protonate it.
    \end{itemize}
    \item 
        \schemestart
            \chemfig{-[:30](=[2]O)-[:-30]}
            \arrow{->[\chemfig[atom sep=1.4em]{HO-[:60]--[:-60]OH}][\ce{H+} $[-\ce{H2O}]$]}[,1.6]
            \chemfig{-[:30](-[:60,1.2]O-[:104,0.75]-[4]?)(-[:120,1.2]O?)-[:-30]}
            \arrow{->[\ce{H3O+}]}
            \chemfig{-[:30](=[2]O)-[:-30]}
        \schemestop
    \begin{itemize}[label={--}]
        \item We use ketals as protecting groups when we want our ketone to be able to withstand basic conditions (remember, ketals need acid to form and be unformed).
        \item We can also protect 1,2- and 1,3-diols from basic conditions with acetone.
    \end{itemize}
\end{enumerate}
\textbf{Nitrogen Nucleophiles}
\begin{enumerate}
    \footnotesize
    \setchemfig{atom sep=1.4em}
    \item 
        \schemestart
            \chemfig{-[:30](=[2]O)-[:-30]}
            \arrow{0}[,0.1]\+
            \chemfig{MeNH_2}
            \arrow
            \chemfig{-[:30](=[2]N-[:30])-[:-30]}
        \schemestop
    \begin{itemize}[label={--}]
        \item Acidic, basic, or neutral conditions.
        \item We have only been taught the acidic mechanism, though, which is analogous to the other nucleophilic addition to carbonyl mechanisms we've been working with.
    \end{itemize}
    \item 
        \schemestart
            \chemfig{-[:30](=[2]O)-[:-30]}
            \arrow{0}[,0.1]\+
            \chemfig{H_2N-OH}
            \arrow{->[cat. \ce{H+}]}[,1.2]
            \chemfig{-[:30](=[2]N-[:30]OH)-[:-30]}
        \schemestop
    \item 
        \schemestart
            \chemfig{-[:30](=[2]O)-[:-30]}
            \arrow{0}[,0.1]\+
            \chemfig{H_2N-NH_2}
            \arrow{->[cat. \ce{H+}]}[,1.2]
            \chemfig{-[:30](=[2]N-[:30]NH_2)-[:-30]}
        \schemestop
    \item 
        \schemestart
            \chemfig{-[:30](=[2]O)-[:-30]}
            \arrow{0}[,0.1]\+{,,-1em}
            \chemfig{-[:-30]\chembelow{N}{H}-[:30]}
            \arrow{->[cat. \ce{H+}]}[,1.2]
            \chemfig{-[:30](-[2]N(-[:150])(-[:30]))=[:-30]}
        \schemestop
    \item 
        \schemestart
            \chemfig{-[:30](=[2]N-[:30]NH_2)-[:-30]}
            \arrow{->[\ce{NaOH, H2O}][$\Delta$ ($\SI{200}{\celsius}\,+$)]}[,1.6]
            \chemfig{-[:30](-[:110]H)(-[:70]H)-[:-30]}
            \arrow{0}[,0.1]\+
            \chemfig{N~N}
        \schemestop
\end{enumerate}
\textbf{Hydride Nucleophiles}
\begin{enumerate}
    \footnotesize
    \setchemfig{atom sep=1.4em}
    \item \ce{NaBH4 + MeOH} reduces aldehydes and ketones to alcohols.
    \item \ce{LiAlH4} followed by \ce{H3O+} reduces an ester to its two component alcohols.
\end{enumerate}
\textbf{Carbide Nucleophiles}
\begin{enumerate}
    \footnotesize
    \setchemfig{atom sep=1.4em}
    \item \ce{RBr ->[Mg${}^\circ$][Et2O] RMgBr}
    \item \ce{RBr ->[2Li${}^\circ$][Et2O] RLi + LiBr}
    \item 
        \schemestart
            \chemfig{R-[:30](=[2]O)-[:-30]H}
            \arrow{->[1. \ce{R$'$MgBr}][2. \ce{H3O+}\rule{3mm}{0pt}]}[,1.4]
            \chemfig{R-[:30](-[2]OH)-[:-30]R'}
        \schemestop
    \item 
        \schemestart
            \chemfig{R-[:30](=[2]O)-[:-30]R'}
            \arrow{->[1. \ce{R$''$MgBr}][2. \ce{H3O+}\rule{3mm}{0pt}]}[,1.4]
            \chemfig{R-[:30](-[2]OH)(-[:-70]R'')-[:-30]R'}
        \schemestop
    \item 
        \schemestart
            \chemfig{R-[:30](=[2]O)-[:-30]OR'}
            \arrow{->[1. \ce{R$''$MgBr}][2. \ce{H3O+}\rule{3mm}{0pt}]}[,1.4]
            \chemfig{R-[:30](-[2]OH)(-[:-70]R'')-[:-30]R''}
            \arrow{0}[,0.1]\+
            \chemfig{R'OH}
        \schemestop
    \item 
        \schemestart
            \chemfig{-[:30](=[2]O)-[:-30]}
            \arrow{->[\ce{HCN}]}
            \chemfig{-[:30](-[:110]HO)(-[:70]CN)-[:-30]}
        \schemestop
    \begin{itemize}[label={--}]
        \item Can be accelerated by an acid/base catalyst, but no catalyst is necessary. Acid catalysts are more common.
    \end{itemize}
\end{enumerate}
\textbf{Ylide Nucleophiles}
\begin{enumerate}
    \footnotesize
    \setchemfig{atom sep=1.4em}
    \item \ce{Ph3P ->[MeBr] Ph3BrP-CH3 ->[KO^{$t$}Bu] Ph3P=CH2}
    \item 
        \schemestart
            \chemfig{R-[:30](=[2]O)-[:-30]H}
            \arrow{0}[,0.1]\+
            \chemfig{Ph_3\charge{90:3pt=$\oplus$}{P}-\charge{90:3pt=$\ominus$}{C}H_2}
            \arrow
            \chemfig{R-[:30](=[2]CH_2)-[:-30]H}
            \arrow{0}[,0.1]\+
            \chemfig{Ph_3P=O}
        \schemestop
    \begin{itemize}[label={--}]
        \item Presence or lack thereof of a betaine in the mechanism.
        \item We need at least one hydrogen on the carbon portion of the ylide.
        \item Stereoselective for the \emph{cis} product.
        \begin{itemize}
            \item Except if there's an EWG on the ylide; then we form the \emph{trans} product.
        \end{itemize}
        \item Ketone Wittigs still proceed, but slower. Biggest groups end up \emph{cis} (except, once again, in the case of ylide EWGs).
    \end{itemize}
\end{enumerate}
\textbf{$\bm{\alpha}$,$\bm{\beta}$-Unsaturated Carbonyls}
\begin{figure}[h!]
    \centering
    \footnotesize
    \begin{subfigure}[b]{\linewidth}
        \centering
        \schemestart
            \chemfig{-[:30](=[@{db1a}2]O)-[@{sb1}:-30]=_[@{db1b}6]@{C1}}
            \arrow{->[\chemfig[atom sep=1.4em]{@{H2}H-[@{sb2}]@{O2}OMe}][\chemfig[atom sep=1.4em]{\charge{45:1pt=$\oplus$}{Na}-[,0.6,,,white]H-[@{sb3}]\charge{90:3pt=$\ominus$}{B}H_3-[2,0.6,,,opacity=0]}]}[,1.7]
            \chemfig{-[:30](-[2]OH)=_[:-30]-[6]}
            \arrow
            \chemfig{-[:30](=[2]O)-[:-30]-[6]}
            \arrow{->[\ce{NaBH4}][\ce{MeOH}]}[,1.2]
            \chemname{\chemfig{-[:30](-[2]OH)-[:-30]=_[6]}}{50\%}
            \arrow{0}[,0.1]\+
            \chemname{\chemfig{-[:30](-[2]OH)-[:-30]-[6]}}{50\%}
        \schemestop
        \chemmove{
            \draw [curved arrow={2pt}{2pt}] (sb3) to[out=-90,in=-30] (C1);
            \draw [curved arrow={4pt}{2pt}] (db1b) to[bend left=60,looseness=2] (sb1);
            \draw [curved arrow={3pt}{2pt}] (db1a) to[bend left=30] (H2);
            \draw [curved arrow={2pt}{2pt}] (sb2) to[bend left=90,looseness=4] (O2);
        }
        \caption*{\ce{NaBH4}.}
    \end{subfigure}\\[2em]
    \begin{subfigure}[b]{\linewidth}
        \centering
        \schemestart
            \chemfig{-[:30](=[@{db1a}2]@{O1}O)-[@{sb1}:-30]=_[@{db1b}6]@{C1}}
            \arrow{->[][\chemfig[atom sep=1.4em]{\charge{45:1pt=$\oplus$}{Li}-[,0.6,,,white]H-[@{sb2}]\charge{90:3pt=$\ominus$}{Al}H_3-[2,0.6,,,opacity=0]}]}[,1.7]
            \chemfig{-[:30](-[2]O-[:30]\charge{90:3pt=$\ominus$}{Al}H_3)=_[:-30]-[6]}
            \arrow{->[\ce{H3O+}]}
            \chemname{\chemfig{-[:30](-[2]OH)-[:-30]=_[6]}}{Major}
            \arrow{0}[,0.1]\+
            \chemname{\chemfig{-[:30](=[2]O)-[:-30]-[6]}}{Minor}
        \schemestop
        \chemmove{
            \draw [curved arrow={2pt}{2pt}] (sb2) to[out=-90,in=-30] (C1);
            \draw [curved arrow={4pt}{2pt}] (db1b) to[bend left=60,looseness=2] (sb1);
            \draw [curved arrow={3pt}{2pt}] (db1a) to[bend right=90,looseness=3] (O1);
        }
        \caption*{\ce{LiAlH4}.}
    \end{subfigure}
    \caption*{Reduction of $\alpha,\beta$ unsaturated compounds.}
\end{figure}
\setcounter{figure}{0}
\begin{enumerate}
    \footnotesize
    \setchemfig{atom sep=1.4em}
    \item Organolithiums select for 1,2-addition.
    \item Grignards are intermediate.
    \item Cuprates select for 1,4-addition.
    \begin{itemize}[label={--}]
        \item \ce{2MeLi ->[CuI] LiCuMe2}
    \end{itemize}
\end{enumerate}

\subsubsection*{Carboxylic Acid Synthesis}
\begin{enumerate}
    \footnotesize
    \setchemfig{atom sep=1.4em}
    \item 
        \schemestart
            \chemfig{R-[:30]-[2]OH}
            \arrow{->[\ce{CrO3, H2SO4}][\ce{H2O}]}[,1.7]
            \chemfig{R-[:30](=[2]O)-[:-30]OH}
        \schemestop
    \item 
        \schemestart
            \chemfig{RMgBr}
            \arrow{->[1. \ce{CO2}\rule{2.5mm}{0pt}][2. \ce{H3O+}]}[,1.2]
            \chemfig{R-[:30](=[2]O)-[:-30]OH}
        \schemestop
    \begin{itemize}[label={--}]
        \item Either lithiates or Grignards will suffice.
    \end{itemize}
    \item 
        \schemestart
            \chemfig{RCN}
            \arrow{->[\ce{H3O+}]}
            \chemfig{R-[:30](=[2]O)-[:-30]OH}
        \schemestop
\end{enumerate}

\subsubsection*{Nucleophilic Acyl Substitution}
\textbf{General}
\begin{enumerate}
    \footnotesize
    \setchemfig{atom sep=1.4em}
    \item 
        \schemestart
            \chemfig{R-[:30](=[2]O)-[:-30]LG}
            \arrow{0}[,0.1]\+
            \chemfig{Nu-H}
            \arrow{->[acid or][base]}[,1.1]
            \chemfig{R-[:30](=[2]O)-[:-30]Nu}
            \arrow{0}[,0.1]\+
            \chemfig{LG-H}
        \schemestop
\end{enumerate}
\textbf{Dehydration of Amides}
\begin{enumerate}
    \footnotesize
    \setchemfig{atom sep=1.4em}
    \item 
        \schemestart
            \chemfig{R-[:30](=[2]O)-[:-30]NH_2}
            \arrow{->[reagents][$\Delta$]}[,1.2]
            \chemfig{RCN}
        \schemestop
    \begin{itemize}[label={--}]
        \item Reagents are either \ce{SOCl2} or \ce{POCl3}.
        \item We need an amide with two hydrogens to run this.
    \end{itemize}
\end{enumerate}
\textbf{S\textsubscript{N}2 Nitrile Formation}
\begin{enumerate}
    \footnotesize
    \setchemfig{atom sep=1.4em}
    \item 
        \schemestart
            \chemfig{-[:-30]-[:30](<[2]Br)-[:-30]}
            \arrow{->[\ce{KCN}]}
            \chemfig{-[:-30]-[:30](<:[2]CN)-[:-30]}
        \schemestop
\end{enumerate}
\textbf{Ester Synthesis}
\begin{enumerate}
    \footnotesize
    \setchemfig{atom sep=1.4em}
    \item 
        \schemestart
            \chemfig{R-[:30](=[2]O)-[:-30]OH}
            \arrow{->[\ce{K2CO3}]}[,1.2]
            \chemfig{R-[:30](=[2]\textcolor{grx}{O})-[:-30]\charge{45:1pt=$\ominus$}{\textcolor{grx}{O}}-[,0.6,,,white]\charge{45:1pt=$\oplus$}{K}}
            \arrow{->[\ce{R$'$I}]}
            \chemfig{R-[:30](=[2]\textcolor{grx}{O})-[:-30]\textcolor{grx}{O}-[:30]R'}
        \schemestop
    \item 
        \schemestart
            \chemfig{R-[:30](=[2]\textcolor{grx}{O})-[:-30]\textcolor{grx}{O}H}
            \arrow{->[\ce{H+}][\ce{R$'${\color{blx}O}H}]}
            \chemfig{R-[:30](=[2]\textcolor{grx}{O})-[:-30]\textcolor{blx}{O}-[:30]R'}
        \schemestop
    \begin{itemize}[label={--}]
        \item The acid is a catalyst.
        \item The alcohol usually doubles as the solvent, esp. since we need it in excess.
        \item Removing water can drive the forward reaction; excess \ce{H3O+} reverses it.
    \end{itemize}
\end{enumerate}
\textbf{Ester Reactions}
\begin{enumerate}
    \footnotesize
    \setchemfig{atom sep=1.4em}
    \item 
        \schemestart
            \chemfig{R-[:30](=[2]O)-[:-30]OR'}
            \arrow{->[\ce{KOH}]}
            \chemfig{R-[:30](=[2]O)-[:-30]OK}
            \arrow{0}[,0.1]\+
            \chemfig{R'OH}
        \schemestop
\end{enumerate}
\textbf{Acid Chloride Synthesis}
\begin{enumerate}
    \footnotesize
    \setchemfig{atom sep=1.4em}
    \item 
        \schemestart
            \chemfig{R-[:30](=[2]O)-[:-30]OH}
            \arrow{->[\ce{SOCl2}][\ce{Py}]}
            \chemfig{R-[:30](=[2]O)-[:-30]Cl}
            \arrow{0}[,0.1]\+
            \chemfig{\ce{[PyH]Cl}}
            \arrow{0}[,0.1]\+
            \chemfig{SO_2}
        \schemestop
\end{enumerate}
\textbf{Anhydride Synthesis}
\begin{enumerate}
    \footnotesize
    \setchemfig{atom sep=1.4em}
    \item 
        \schemestart
            \chemfig{2}\arrow{0}[,0.1]\chemfig{R-[:30](=[2]O)-[:-30]OH}
            \arrow{->[$\Delta$][$[-\ce{H2O}]$]}[,1.2]
            \chemfig{R-[:30](=[2]O)-[:-30]O-[:30](=[2]O)-[:-30]R}
        \schemestop
    \begin{itemize}[label={--}]
        \item Also works intramolecularly.
        \item Combining different carboxylic acids leads to a statistical mixture of products.
    \end{itemize}
    \item 
        \schemestart
            \chemfig{R-[:30](=[2]O)-[:-30]Cl}
            \arrow{0}[,0.1]\+{,,1.5em}
            \chemfig{R'-[:30](=[2]O)-[:-30]\charge{45:1pt=$\ominus$}{O}-[,0.6,,,white]\charge{45:1pt=$\oplus$}{Na}}
            \arrow
            \chemfig{R-[:30](=[2]O)-[:-30]O-[:30](=[2]O)-[:-30]R'}
            \arrow{0}[,0.1]\+
            \chemfig{NaCl}
        \schemestop
\end{enumerate}
\textbf{Amide Synthesis}
\begin{enumerate}
    \footnotesize
    \setchemfig{atom sep=1.4em}
    \item 
        \schemestart
            \chemfig{R-[:30](=[2]O)-[:-30]OH}
            \arrow{0}[,0.1]\+
            \chemfig{HN(-[:30]R')(-[6,,2]R'')}
            \arrow{->[DCC][Py]}
            \chemfig{R-[:30](=[2]O)-[:-30]N(-[:30]R')(-[6]R'')}
        \schemestop
\end{enumerate}
\textbf{Carbide Nucleophiles}
\begin{enumerate}
    \footnotesize
    \setchemfig{atom sep=1.4em}
    \item 
        \schemestart
            \chemfig{R-[:30](=[2]O)-[:-30]R'}
            \arrow{->[1. \ce{R$''$Li}\rule{1.5mm}{0pt}][2. \ce{H3O+}]}[,1.3]
            \chemfig{R-[:30](-[2]OH)(-[:-70]R'')-[:-30]R'}
        \schemestop
    \item 
        \schemestart
            \chemfig{R-[:30](=[2]O)-[:-30]OH}
            \arrow{->[\ce{R$''$Li}]}
            \chemfig{R-[:30](=[2]O)-[:-30]OLi}
            \arrow{0}[,0.1]\+
            \chemfig{R'H}
        \schemestop
    \item 
        \schemestart
            \chemfig{R-[:30](=[2]O)-[:-30]OR'}
            \arrow{->[1. \ce{R$''$Li}\rule{1.5mm}{0pt}][2. \ce{H3O+}]}[,1.3]
            \chemfig{R-[:30](-[2]OH)(-[:-70]R'')-[:-30]R''}
            \arrow{0}[,0.1]\+
            \chemfig{R'OH}
        \schemestop
\end{enumerate}
\textbf{Hydride Nucleophiles}
\begin{enumerate}
    \footnotesize
    \setchemfig{atom sep=1.4em}
    \item Esters and \ce{NaBH4} do not react.
    \item 
        \schemestart
            \chemfig{R-[:30](=[2]O)-[:-30]OR'}
            \arrow{->[1. \ce{LiAlH4}][2. \ce{H3O+}\rule{1mm}{0pt}]}[,1.3]
            \chemfig{R-[:30]-[2]OH}
            \arrow{0}[,0.1]\+
            \chemfig{R'OH}
        \schemestop
    \item 
        \schemestart
            \chemfig{R-[:30](=[2]O)-[:-30]OR'}
            \arrow{->[1. DIBAL-H][2. \ce{H3O+}\rule{4.5mm}{0pt}]}[,1.6]
            \chemfig{R-[:30](=[2]O)-[:-30]H}
            \arrow{0}[,0.1]\+
            \chemfig{R'OH}
        \schemestop
    \item 
        \schemestart
            \chemfig{R-[:30](=[2]O)-[:-30]NR'R''}
            \arrow{->[\ce{LiAlH4}]}[,1.2]
            \chemfig{R-[:30]-[:-30]NR'R''}
        \schemestop
    \item 
        \schemestart
            \chemfig{R-[:30](=[2]O)-[:-30]NR'R''}
            \arrow{->[1. DIBAL-H][2. \ce{H3O+}\rule{4.5mm}{0pt}]}[,1.6]
            \chemfig{R-[:30](=[2]O)-[:-30]H}
            \arrow{0}[,0.1]\+
            \chemfig{HNR'R''}
        \schemestop
\end{enumerate}
\textbf{Nitrile Reactions}
\begin{enumerate}
    \footnotesize
    \setchemfig{atom sep=1.4em}
    \item 
        \schemestart
            \chemfig{RCN}
            \arrow{->[1. \ce{R$'$Li}\rule{2.5mm}{0pt}][2. \ce{H3O+}]}[,1.3]
            \chemfig{R-[:30](=[2]O)-[:-30]R'}
        \schemestop
    \item 
        \schemestart
            \chemfig{RCN}
            \arrow{->[1. DIBAL-H][2. \ce{H3O+}\rule{4.5mm}{0pt}]}[,1.6]
            \chemfig{R-[:30](=[2]O)-[:-30]H}
        \schemestop
    \item 
        \schemestart
            \chemfig{RCN}
            \arrow{->[1. \ce{LiAlH4}][2. \ce{H3O+}\rule{1mm}{0pt}]}[,1.3]
            \chemfig{R-[:30]-[:-30]NH_2}
        \schemestop
\end{enumerate}
\textbf{Acid to Ketone}
\begin{enumerate}
    \footnotesize
    \setchemfig{atom sep=1.4em}
    \item 
        \schemestart
            \chemfig{R-[:30](=[2]O)-[:-30]OH}
            \arrow{->[1. \ce{R$'$Li}, $\Delta$\rule{6mm}{0pt}][2. \ce{H3O+}, time]}[,1.8]
            \chemfig{R-[:30](=[2]O)-[:-30]R'}
        \schemestop
    \begin{itemize}[label={--}]
        \item We need excess lithiate here.
        \item Grignards won't work.
    \end{itemize}
\end{enumerate}
\textbf{Insertion Reactions}
\begin{enumerate}
    \footnotesize
    \setchemfig{atom sep=1.4em}
    \item 
        \schemestart
            \chemfig{*6(----(=O)--)}
            \arrow{->[mCPBA]}[,1.2]
            \chemfig{[:-12.86]*7(----O-(=O)--)}
        \schemestop
    \begin{itemize}[label={--}]
        \item Migratory aptitude favors bulkier groups.
    \end{itemize}
    \item 
        \schemestart
            \chemfig{[:18]*5(---(=O)--)}
            \arrow{->[\ce{HN3}]}
            \chemfig{*6(---NH-(=O)--)}
        \schemestop
    \begin{itemize}[label={--}]
        \item Can be supplemented by external acid catalyst (some acid stronger than hydrazoic acid).
    \end{itemize}
    \item 
        \schemestart
            \chemfig{R-[:30](=[2]O)-[:-30]Cl}
            \arrow{->[\ce{NaN3}][$\Delta$]}
            \chemfig{RN=C=O}
            \arrow{->[\ce{NaOH}][\ce{H2O}]}
            \chemfig{RNH_2}
        \schemestop
    \item 
        \schemestart
            \chemfig{RN=C=O}
            \arrow{->[\ce{R$'$OH}][cat. base]}[,1.3]
            \chemfig{HRN-[:30](=[2]O)-[:-30]OR'}
        \schemestop
    \item 
        \schemestart
            \chemfig{R-[:30](=[2]O)-[:-30]OH}
            \arrow{->[DPPA]}
            \chemfig{RN=C=O}
        \schemestop
    \item 
        \schemestart
            \chemfig{*6(----(=O)--)}
            \arrow{->[\ce{H2NOH}][\ce{H3O+}]}[,1.2]
            \chemfig{[:-12.86]*7(----[,,,1]NH-(=O)--)}
        \schemestop
\end{enumerate}


\subsection*{Reminders}
\begin{itemize}
    \item Carbonyl electrophilicity has to do with sterics (the primary factor) and electronics.
    \item Reactivity scale.
    \begin{equation*}
        \text{acid chloride} > \text{anhydride}
        > \text{ester}
        > \text{amide}
        > \text{carboxylate}
    \end{equation*}
    \item DMAP is one of the best catalysts for nucleophilic acyl substitutions.
\end{itemize}
\newpage



\section[Reactions at the \texorpdfstring{$\alpha$}{TEXT}-Carbon of Carbonyl Compounds 1]{Reactions at the \texorpdfstring{$\bm{\alpha}$}{TEXT}-Carbon of Carbonyl Compounds 1}
\begin{itemize}
    \item \marginnote{4/19:}Comparing Units 1-3.
    \begin{itemize}
        \item Units 1 and 2 were about nucleophiles adding to electrophilic carbonyls.
        \item Unit 3 talks about carbonyls as nucleophiles (i.e., when they've been deprotonated at the $\alpha$-position).
    \end{itemize}
    \item \textbf{Enolate}: The class of molecules that resonate between a carbonyl with a carbanion at the $\alpha$-position and a deprotonated, negatively charged enol. \emph{Structure}
    \begin{figure}[h!]
        \centering
        \footnotesize
        \schemestart
            \chemfig{R-[:-30]@{C1}\charge{-90:3pt=$\ominus$}{}-[@{sb1}:30](=[@{db1}2]@{O1}O)-[:-30]R'}
            \arrow{<->}
            \chemfig{R-[:-30]=^[:30](-[2]\charge{45:1pt=$\ominus$}{O})-[:-30]R'}
        \schemestop
        \chemmove{
            \draw [curved arrow={10pt}{2pt},blx] (C1) to[bend right=100,looseness=6] (sb1);
            \draw [curved arrow={3pt}{2pt},blx] (db1) to[bend right=90,looseness=3] (O1);
        }
        \caption{Enolate.}
        \label{fig:enolate}
    \end{figure}
    \begin{itemize}
        \item We care about enolates as a way to form \ce{C-C} bonds.
    \end{itemize}
    \item Our current list of \ce{C-C} bond forming reactions includes\dots
    \begin{enumerate}
        \item Wittig.
        \begin{itemize}
            \item Combines an ylide and a carbonyl electrophile.
        \end{itemize}
        \item Friedel-Crafts.
        \begin{itemize}
            \item Combines an arene and a carbonyl electrophile.
        \end{itemize}
        \item Cyanide nucleophile.
        \begin{itemize}
            \item Combines \ce{HCN} or a \ce{CN-} source and a carbonyl electrophile.
        \end{itemize}
        \item Organometallics: Grignards, lithiates, and alkylyl anions.
        \begin{itemize}
            \item Combine carbanions and a carbonyl electrophile.
        \end{itemize}
        \item Diels-Alder.
        \item \textbf{Simmons Smith cyclopropanation}.
    \end{enumerate}
    \item Simmons Smith cyclopropanation.
    \item General form.
    \begin{equation*}
        \ce{C2H4 ->[Zn/Cu][CH2I2] C3H6}
    \end{equation*}
    \begin{itemize}
        \item This reaction is commonly taught in CHEM 22000 or CHEM 22100; the fact that it was not our year does not now make it our responsibility on tests.
    \end{itemize}
    \item The takeaway from this refresher of \ce{C-C} bond forming reactions is that of the six ways we know to make \ce{C-C} bonds, four involve carbonyls (and in all of these, the carbonyl role plays as an electrophile).
    \begin{itemize}
        \item As mentioned above, Unit 3 is about flipping this paradigm, i.e., making carbonyls into nucleophiles.
    \end{itemize}
    \item $\pKa$'s.
    \begin{itemize}
        \item Deprotonating an \ce{O-H} bond: Recall that acetic acid ($\pKa\approx 5$) is $10^{10}$ times more acidic than ethanol ($\pKa\approx 15$) due to resonance stabilization of the conjugate base in the former.
        \item Deprotonating a \ce{C-H} bond: A hydrogen on the 1-carbon of propane ($\pKa\approx 50$) is \numrange{e25}{e30} times less acidic than a hydrogen on acetone ($\pKa\approx\text{\numrange{20}{25}}$) once again due to resonance stabilization of the latter (note that deprotonated acetone constitutes an enolate).
    \end{itemize}
    \item Enolates have two main modes of reactivity.
    \begin{figure}[H]
        \centering
        \footnotesize
        \begin{subfigure}[b]{\linewidth}
            \centering
            \schemestart
                \chemfig{-[:30](-[@{sb1}2]@{O1}\charge{180=\:,45:1pt=$\ominus$}{O})=^[@{db1}:-30]}
                \arrow{0}[,0.6]
                \chemfig{@{E2}\charge{45:1pt=$\oplus$}{E}}
                \arrow
                \chemfig{-[:30](=[2]O)-[:-30]-[:30]E}
            \schemestop
            \chemmove{
                \draw [curved arrow={6pt}{2pt}] (O1) to[bend right=90,looseness=3] (sb1);
                \draw [curved arrow={4pt}{2pt}] (db1) to[out=60,in=180] (E2);
            }
            \caption{Adding an electrophile at the carbon.}
            \label{fig:enolateReactivitya}
        \end{subfigure}\\[3em]
        \begin{subfigure}[b]{\linewidth}
            \centering
            \schemestart
                \chemfig{-[:30](-[2]@{O1}\charge{90=\:,45:1pt=$\ominus$}{O})=^[:-30]}
                \arrow{0}[,0.6]
                \chemfig{@{E2}\charge{45:1pt=$\oplus$}{E}}
                \arrow
                \chemfig{-[:30](-[2]OE)=^[:-30]}
            \schemestop
            \chemmove{
                \draw [curved arrow={6pt}{2pt}] (O1) to[out=90,in=90,looseness=1.5] (E2);
            }
            \caption{Adding an electrophile at the oxygen.}
            \label{fig:enolateReactivityb}
        \end{subfigure}
        \caption{Reactions of enolates and electrophiles.}
        \label{fig:enolateReactivity}
    \end{figure}
    \begin{itemize}
        \item We will focus on the mode in Figure \ref{fig:enolateReactivitya} because we're most interested in making new bonds to carbon.
    \end{itemize}
    \item If $\ce{E+}=\ce{H+}$, then we can either generate a ketone (via Figure \ref{fig:enolateReactivitya}) or an \textbf{enol} (via Figure \ref{fig:enolateReactivityb}).
    \item \textbf{Enol}: The class of molecules containing adjacent alk\underline{en}e and alcoh\underline{ol} functional groups. \emph{Structure}
    \begin{figure}[h!]
        \centering
        \footnotesize
        \chemfig{R-[:-30]-[:30](-[2]OH)=^[:-30]R'}
        \caption{Enol.}
        \label{fig:enol}
    \end{figure}
    \item \textbf{Tautomers}: Two constitutional isomers that rapidly interconvert. \emph{Etymology} from Greek \textbf{taut} "same" and \textbf{mer} "part."
    \begin{itemize}
        \item Example: Enols and ketones are tautomers.
    \end{itemize}
    \item Enol formation (acid-catalyzed).
    \item General form.
    \begin{center}
        \footnotesize
        \setchemfig{atom sep=1.4em}
        \schemestart
            \chemfig{-[:30](=[2]O)-[:-30]-[:30]H}
            \arrow{->[cat. \ce{H+}]}[,1.2]
            \chemfig{-[:30](-[2]OH)=^[:-30]}
        \schemestop
    \end{center}
    \item Mechanism.
    \begin{figure}[h!]
        \centering
        \footnotesize
        \schemestart
            \chemfig{-[:30](=[2]@{O1}\charge{90=\:}{O})-[:-30]}
            \arrow{->[\chemfig[atom sep=1.4em]{@{H2}H-[@{sb2}]@{X2}X}]}[,1.1]
            \chemfig{-[:30](=[@{db3}2]@{O3}\charge{90:3pt=$\oplus$}{O}-[:30]H)-[@{sb3a}:-30]-[@{sb3b}:30]@{H3}H}
            \arrow{0}[,0.6]
            \chemfig{@{X4}\charge{90=\:,45:1pt=$\ominus$}{X}}
            \arrow{->[][-\ce{HX}]}
            \chemfig{-[:30](-[2]OH)=^[:-30]}
        \schemestop
        \chemmove{
            \draw [curved arrow={6pt}{2pt}] (O1) to[out=90,in=90,looseness=2] (H2);
            \draw [curved arrow={2pt}{2pt}] (sb2) to[bend left=90,looseness=3.5] (X2);
            \draw [curved arrow={6pt}{2pt}] (X4) to[out=90,in=90,looseness=2] (H3);
            \draw [curved arrow={2pt}{2pt}] (sb3b) to[bend right=60,looseness=2] (sb3a);
            \draw [curved arrow={3pt}{2pt}] (db3) to[bend left=90,looseness=3] (O3);
        }
        \caption{Acid-catalyzed enol formation mechanism.}
        \label{fig:enolFormationAcid}
    \end{figure}
    \item Enol formation (base-catalyzed).
    \item General form.
    \begin{center}
        \footnotesize
        \setchemfig{atom sep=1.4em}
        \schemestart
            \chemfig{-[:30](=[2]O)-[:-30]-[:30]H}
            \arrow{->[cat. \ce{B}:]}[,1.1]
            \chemfig{-[:30](-[2]OH)=^[:-30]}
        \schemestop
    \end{center}
    \item Mechanism.
    \begin{figure}[H]
        \centering
        \vspace{1em}
        \footnotesize
        \schemestart
            \chemfig{-[:30](=[@{db1}2]@{O1}O)-[@{sb1a}:-30]-[@{sb1b}:30]@{H1}H}
            \arrow{->[\chemfig{@{B2}\charge{90=\:}{B}}]}
            \chemfig{-[:30](-[2]@{O3}\charge{90=\:,45:1pt=$\ominus$}{O})=^[:-30]}
            \arrow{0}[,0.6]
            \chemfig{@{H4}H-[@{sb4}]@{B4}\charge{90:3pt=$\oplus$}{B}}
            \arrow{->[][-\ce{B}]}
            \chemfig{-[:30](-[2]OH)=^[:-30]}
        \schemestop
        \chemmove{
            \draw [curved arrow={6pt}{2pt}] (B2) to[out=90,in=90,looseness=2] (H1);
            \draw [curved arrow={2pt}{2pt}] (sb1b) to[bend right=60,looseness=2] (sb1a);
            \draw [curved arrow={3pt}{2pt}] (db1) to[bend left=90,looseness=3] (O1);
            \draw [curved arrow={6pt}{2pt}] (O3) to[out=90,in=90,looseness=1.5] (H4);
            \draw [curved arrow={2pt}{2pt}] (sb4) to[bend right=90,looseness=3] (B4);
        }
        \caption{Base-catalyzed enol formation mechanism.}
        \label{fig:enolFormationBase}
    \end{figure}
    \begin{itemize}
        \item If the base has a $\pKa$ greater than that of the carbonyl, then the compound gets stuck at the enolate.
        \begin{itemize}
            \item In other words, the enol will only form when the base is weak enough to do the initial deprotonation but not the reverse deprotonation, i.e., it can set up a keto-enol equilibrium but not stoichiometrically deprotonate the ketone.
        \end{itemize}
        \item All of next lecture is on really strong bases and enolates.
    \end{itemize}
    \item Levin also draws the reverse mechanism for both of these reactions as per the principle of microscopic reversibility.
    \begin{itemize}
        \item It follows that there is an equilibrium between a ketone and its enol.
    \end{itemize}
    \item The position of the equilibrium depends largely on the resonance stability of both tautomers, although ketones are favored in general.
    \begin{itemize}
        \item The equilibrium between 1-phenylpropan-1-one and (Z)-1-phenylprop-1-en-1-ol lies heavily on the side of the ketone.
        \begin{itemize}
            \item Resonance between the carbonyl and the benzene ring favors the ketone.
        \end{itemize}
        \item The equilibrium between pentane-2,4-dione and (Z)-4-hydroxypent-3-en-2-one lies mostly on the side of the ketone.
        \begin{itemize}
            \item An extra resonance form stabilizes the enol.
        \end{itemize}
        \item The equilibrium between cyclohexa-2,4-dien-1-one and phenol lies heavily on the side of the enol.
        \begin{itemize}
            \item Aromaticity stabilizes the enol.
        \end{itemize}
    \end{itemize}
    \item Evidence for the existence of enols (which are usually present in such a small portion as to not be isolable).
    \begin{figure}[H]
        \centering
        \footnotesize
        \begin{subfigure}[b]{0.49\linewidth}
            \centering
            \schemestart
                \chemfig{-[:30](=[2]O)-[:-30]-[:30]-[:-30]}
                \arrow{->[\ce{D3O+}][\ce{D2O / DO-}]}[,1.5]
                \chemfig{D-[:-30](-[:-110]D)(-[:-70]D)-[:30](=[2]O)-[:-30](-[:-110]D)(-[:-70]D)-[:30]-[:-30]}
            \schemestop
            \caption{Deuteration.}
            \label{fig:enolEvidencea}
        \end{subfigure}
        \begin{subfigure}[b]{0.49\linewidth}
            \centering
            \schemestart
                \chemfig{Ph-[:30](=[2]O)-[:-30](<[:-110]H_3C)(<:[:-70]H)-[:30]-[:-30]}
                \arrow{->[\ce{H3O+}][\ce{H2O / HO-}]}[,1.5]
                \chemfig{Ph-[:30](=[2]O)-[:-30](-[6,,,,decorate,decoration={snake,amplitude=1.5pt,segment length=5.3pt}]CH_3)-[:30]-[:-30]}
            \schemestop
            \caption{Racemization.}
            \label{fig:enolEvidenceb}
        \end{subfigure}
        \caption{Evidence for the existence of enols.}
        \label{fig:enolEvidence}
    \end{figure}
    \begin{enumerate}
        \item Deuteration of carbonyl compounds (Figure \ref{fig:enolEvidencea}).
        \begin{itemize}
            \item Proves the existence of a process that is "washing in" the deuterium, but only at the $\alpha$-positions.
            \item Note that \ce{D2O / DO-} denotes basic deuterated water, and that only one of acidic \emph{or} basic deuterated water is used at one time.
        \end{itemize}
        \item Racemization of compounds that are enantiopure at the $\alpha$-position (Figure \ref{fig:enolEvidenceb}).
        \begin{itemize}
            \item Thus, we're removing the hydrogen, forming an achiral intermediate, and then putting that hydrogen back but randomly this time.
        \end{itemize}
    \end{enumerate}
    \item Halogenation of enols (acid-catalyzed).
    \item General form.
    \begin{center}
        \footnotesize
        \setchemfig{atom sep=1.4em}
        \schemestart
            \chemfig{*6(=-=(-(=[2]O)-[:-30])-=-)}
            \arrow{->[\ce{AcOH}][\ce{Br2}]}
            \chemfig{*6(=-=(-(=[2]O)-[:-30]-Br)-=-)}
        \schemestop
    \end{center}
    \item Mechanism.
    \begin{figure}[h!]
        \centering
        \vspace{2em}
        \footnotesize
        \schemestart
            \chemfig{*6(=-=(-(=[2]@{O1}\charge{90=\:}{O})-[:-30])-=-)}
            \arrow{<=>[\chemfig[atom sep=1.4em]{-[:30](=[@{db2}2]@{O2}O)-[@{sb2a}:-30]O-[@{sb2b}:30]@{H2}H}]}[,1.6]
            \chemfig{*6(=-=(-(=[@{db3}2]@{O3}\charge{90:3pt=$\oplus$}{O}H)-[@{sb3a}:-30]-[@{sb3b}]@{H3}H)-=-)}
            \arrow{0}[,0.6]
            \chemfig{-[:30](-[2]@{O4}\charge{90=\:,45:1pt=$\ominus$}{O})=[:-30]O}
            \arrow{<=>[][-\ce{AcOH}]}[,1.2]
            \chemfig{*6(=-=(-(-[@{sb5}2]@{O5}\charge{180=\:}{O}H)=[@{db5}:-30])-=-)}
            \arrow{->[*{0} {\chemfig[atom sep=1.4em]{@{Br6a}Br-[@{sb6}]@{Br6b}Br}}]}[-90]
            \subscheme{
                \chemfig{*6(=-=(-(=[2]@{O7}\charge{90:3pt=$\oplus$}{O}-[@{sb7}]@{H7}H)-[:-30]-Br)-=-)}
                \arrow{0}[,0.6]
                \chemfig{@{Br8}\charge{90=\:,45:1pt=$\ominus$}{Br}}
            }
            \arrow{->[][*{0.90}-\ce{HBr}]}[180]
            \chemfig{*6(=-=(-(=[2]O)-[:-30]-Br)-=-)}
        \schemestop
        \chemmove{
            \draw [curved arrow={6pt}{2pt}] (O1) to[out=90,in=90,looseness=1.2] (H2);
            \draw [curved arrow={2pt}{2pt}] (sb2b) to[bend right=60,looseness=1.8] (sb2a);
            \draw [curved arrow={3pt}{2pt}] (db2) to[bend left=90,looseness=3] (O2);
            \draw [curved arrow={6pt}{2pt}] (O4) to[out=90,in=90,looseness=1.5] (H3);
            \draw [curved arrow={2pt}{2pt}] (sb3b) to[bend right=60,looseness=1.8] (sb3a);
            \draw [curved arrow={3pt}{2pt}] (db3) to[bend left=90,looseness=3] (O3);
            \draw [curved arrow={6pt}{2pt}] (O5) to[bend right=90,looseness=3] (sb5);
            \draw [curved arrow={3pt}{2pt}] (db5) to[out=-120,in=90] (Br6a);
            \draw [curved arrow={2pt}{2pt}] (sb6) to[bend left=90,looseness=3] (Br6b);
            \draw [curved arrow={6pt}{2pt}] (Br8) to[out=90,in=0,looseness=1.2] (H7);
            \draw [curved arrow={2pt}{2pt}] (sb7) to[bend left=60,looseness=2] (O7);
        }
        \caption{Acid-catalyzed halogenation of enols mechanism.}
        \label{fig:mechanismEnolHalogenationAcid}
    \end{figure}
    \begin{itemize}
        \item Remember that only a tiny percentage of the enol will be formed in the equilibrium constituting the first two steps, but these few molecules formed will be piped through the rest of the reaction over time and will pull more through via Le Ch\^{a}telier's principle.
    \end{itemize}
    \item Haloform reaction.
    \item General form.
    \begin{center}
        \footnotesize
        \setchemfig{atom sep=1.4em}
        \schemestart
            \chemfig{*6(=-=(-(=[2]O)-[:-30])-=-)}
            \arrow{->[\ce{NaOH / H2O}][\ce{Br2}]}[,1.6]
            \chemfig{*6(=-=(-(=[2]O)-[:-30]\charge{45:1pt=$\ominus$}{O})-=-)}
            \arrow{0}[,0.1]\+{1em}
            \chemfig{(-[2]H)(-[:-30]Br)(<[:-110]Br)(<:[:-150]Br)}
        \schemestop
    \end{center}
    \begin{itemize}
        \item This reaction essentially constitutes the base-catalyzed halogentation of enols.
        \item We can run this reaction with any halogen (not just bromine), hence the name "\emph{halo}form reaction."
        \item This is how chloroform is made!
    \end{itemize}
    \item \textbf{Bromoform}: The right product of the haloform general reaction above. \emph{Also known as} \textbf{tribromomethane}.
    \begin{itemize}
        \item More generally, any trihalomethane has an old-school, common -form name. For example, we also have \textbf{chloroform} (trichloromethane) and \textbf{iodoform} (triiodomethane).
    \end{itemize}
    \item Mechanism.
    \begin{figure}[h!]
        \centering
        \footnotesize
        \schemestart
            \chemfig{*6(=-=(-(=[@{db1}2]@{O1}O)-[@{sb1a}:-30]-[@{sb1b}]@{H1}H)-=-)}
            \arrow{<=>[\chemfig{@{O2}\charge{90=\:,135:1pt=$\ominus$}{O}H}][\ce{H2O}]}
            \chemfig{*6(=-=(-(-[@{sb3}2]@{O3}\charge{180=\:,90:3pt=$\ominus$}{O})=^[@{db3}:-30])-=-)}
            \arrow{->[\chemfig[atom sep=1.4em]{@{Br4a}Br-[@{sb4}]@{Br4b}Br}][-\ce{Br-}]}[,1.2]
            \chemfig{*6(=-=(-(=[@{db5}2]@{O5}O)-[@{sb5a}:-30](-[:-110]H)(-[:-70]Br)-[@{sb5b}]@{H5}H)-=-)}
            \arrow{<=>[\chemfig{@{O6}\charge{90=\:,135:1pt=$\ominus$}{O}H}][\ce{H2O}]}
            \chemfig{*6(=-=(-(-[@{sb7}2]@{O7}\charge{180=\:,90:3pt=$\ominus$}{O})=_[@{db7}:-30]-Br)-=-)}
            \arrow{->[*{0} {\chemfig[atom sep=1.4em]{@{Br8a}Br-[@{sb8}]@{Br8b}Br}}][*{0}-\ce{Br-}]}[-90]
            \chemfig{*6(=-=(-(=[@{db9}2]@{O9}O)-[@{sb9a}:-30](-[@{sb9b}:-110]@{H9}H)(-[:-70]Br)-Br)-=-)}
            \arrow{<=>[*{0.-90}\ce{H2O}][*{0.90}\chemfig{@{O10}\charge{-90=\:,135:1pt=$\ominus$}{O}H-[2,0.5,,,white]}]}[180]
            \chemfig{*6(=-=(-(-[@{sb11}2]@{O11}\charge{180=\:,90:3pt=$\ominus$}{O})=^[@{db11}:-30](-[6]Br)-Br)-=-)}
            \arrow{->[*{0.-90} {\chemfig[atom sep=1.4em]{@{Br12a}Br-[@{sb12}]@{Br12b}Br}}][*{0.90}-\ce{Br-}]}[180,1.2]
            \chemfig{*6(=-=(-@{C13}(=[@{db13}2]@{O13}O)-[:-30](-[:-110]Br)(-[:-70]Br)-Br)-=-)}
            \arrow{->[*{0.-90}\chemfig{@{O14}\charge{90=\:,135:1pt=$\ominus$}{O}H}]}[180]
            \chemfig{*6(=-=(-(-[@{sb15a}:110]@{O15}\charge{180=\:,90:3pt=$\ominus$}{O})(-[:70]OH)-[@{sb15b}:-30]@{C15}(-[:-110]Br)(-[:-70]Br)-Br)-=-)}
            \arrow[-90]
            \subscheme{
                \chemfig{*6(=-=(-(=[2]O)-[:-30]@{O16}O-[@{sb16}]@{H16}H)-=-)}
                \arrow{0}[,0.6]
                \chemfig{@{C17}\charge{180=\:,90:3pt=$\ominus$}{C}Br_3}
            }
            \arrow
            \chemfig{*6(=-=(-(=[2]O)-[:-30]\charge{45:1pt=$\ominus$}{O})-=-)}
            \arrow{0}[,0.1]\+{1em}
            \chemfig{(-[2]H)(-[:-30]Br)(<[:-110]Br)(<:[:-150]Br)}
        \schemestop
        \chemmove{
            \draw [curved arrow={6pt}{2pt}] (O2) to[out=90,in=90,looseness=2.5] (H1);
            \draw [curved arrow={2pt}{2pt}] (sb1b) to[bend right=60,looseness=1.8] (sb1a);
            \draw [curved arrow={3pt}{2pt}] (db1) to[bend left=90,looseness=3] (O1);
            \draw [curved arrow={6pt}{2pt}] (O3) to[bend right=90,looseness=3] (sb3);
            \draw [curved arrow={4pt}{2pt}] (db3) to[out=60,in=150,looseness=1.2] (Br4a);
            \draw [curved arrow={2pt}{2pt}] (sb4) to[bend left=90,looseness=3] (Br4b);
            \draw [curved arrow={6pt}{2pt}] (O6) to[out=90,in=90,looseness=2.5] (H5);
            \draw [curved arrow={2pt}{2pt}] (sb5b) to[bend right=60,looseness=1.8] (sb5a);
            \draw [curved arrow={3pt}{2pt}] (db5) to[bend left=90,looseness=3] (O5);
            \draw [curved arrow={6pt}{2pt}] (O7) to[bend right=90,looseness=3] (sb7);
            \draw [curved arrow={4pt}{2pt}] (db7) to[out=-120,in=90] (Br8a);
            \draw [curved arrow={2pt}{2pt}] (sb8) to[bend left=90,looseness=3] (Br8b);
            \draw [curved arrow={6pt}{2pt}] (O10) to[out=-90,in=-90,looseness=1.1] (H9);
            \draw [curved arrow={2pt}{2pt}] (sb9b) to[bend left=60,looseness=1.8] (sb9a);
            \draw [curved arrow={3pt}{2pt}] (db9) to[bend left=90,looseness=3] (O9);
            \draw [curved arrow={6pt}{2pt}] (O11) to[bend right=90,looseness=3] (sb11);
            \draw [curved arrow={4pt}{2pt}] (db11) to[out=60,in=90,out looseness=3,in looseness=1.5] (Br12b);
            \draw [curved arrow={2pt}{2pt}] (sb12) to[bend right=90,looseness=3] (Br12a);
            \draw [curved arrow={6pt}{3pt}] (O14) to[out=90,in=150] (C13);
            \draw [curved arrow={3pt}{2pt}] (db13) to[bend right=90,looseness=3] (O13);
            \draw [curved arrow={6pt}{2pt}] (O15) to[out=180,in=-150,looseness=4] (sb15a);
            \draw [curved arrow={2pt}{2pt}] (sb15b) to[out=70,in=90,looseness=3] (C15);
            \draw [curved arrow={6pt}{2pt}] (C17) to[out=180,in=0] (H16);
            \draw [curved arrow={2pt}{2pt}] (sb16) to[bend left=90,looseness=3] (O16);
        }
        \caption{Haloform reaction mechanism.}
        \label{fig:mechanismHaloformRxn}
    \end{figure}
    \begin{itemize}
        \item As with the acid-catalyzed version, only a little bit of the enol will be present at each stage, but Le Ch\^{a}telier's principle is our friend here.
        \item Carbons are not usually good leaving groups, but with three strongly electron-withdrawing halogens, it will leave when the hydroxide is out of options in a last-ditch nucleophilic acyl substitution.
    \end{itemize}
    \item Explaining the difference in the acid- vs. base-catalyzed halogenation of enols.
    \begin{itemize}
        \item Consider the molecule which doubles as the product in the acidic mechanism and the second intermediate in the basic mechanism.
        \item If we are to react this molecule further in the acidic mechanism\dots
        \begin{itemize}
            \item The first step is protonation of the carbonyl.
            \item The bromine (an EWG) \emph{destabilizes} the positive oxygen.
            \item Thus, the SM (which lacks the EWG bromine) reacts faster under acidic conditions. Therefore, all of it will react before any of the product reacts.
        \end{itemize}
        \item If we are to react this molecule further in the basic mechanism\dots
        \begin{itemize}
            \item The first step is deprotonation at the $\alpha$-carbon, resulting in an alkoxide anion.
            \item The bromine (an EWG) \emph{stabilizes} the negative oxygen.
            \item Thus, the monobrominated species reacts faster under basic conditions. This favoritism is exacerbated by the addition of further bromines. Therefore, one molecule of the monobrominated species will react to completion before any more of the SM reacts.
        \end{itemize}
        \item As further evidence, if we do the basic version with only 1 equivalent of bromine, we observe 1/3 carboxylate, a corresponding amount of bromoform, and 2/3 SM in the products.
    \end{itemize}
    \item The haloform reaction doesn't always work (see Figure \ref{fig:haloformBeta}).
    \begin{figure}[h!]
        \centering
        \footnotesize
        \schemestart
            \chemfig{*6(=-=(-(=[2]O)-[:-30]-)-=-)}
            \arrow{->[\ce{NaOH / H2O}][\ce{Br2}]}[,1.6]
            \chemfig{*6(=-=(-(=[2]O)-[:-30]=_[6])-=-)}
        \schemestop
        \caption{$\beta$-hydrogens in the haloform reaction.}
        \label{fig:haloformBeta}
    \end{figure}
    \begin{itemize}
        \item When there are $\beta$-hydrogens, we generate an $\alpha,\beta$-unsaturated ketone.
        \item This is because we'll brominate once (the $\alpha$-hydrogens still have a far lower $\pKa$ than the $\beta$-hydrogens, so they attract the base) and then do an E2.
    \end{itemize}
    \item Synthetically, the haloform reaction has uses most similar to the Baeyer-Villiger.
    \begin{figure}[h!]
        \centering
        \footnotesize
        \schemestart
            \chemfig{*6(=-=(-(=[2]O)-[:-30])-=-)}
            \arrow(a--b){->[\ce{NaOH / H2O}][\ce{Br2}]}[,1.6]
            \chemfig{*6(=-=(-(=[2]O)-[:-30]\charge{45:1pt=$\ominus$}{O})-=-)}
            \arrow(@a--c){->[*{0.-90}1. mCPBA][*{0.90}2. \ce{KOH}\rule{1.4em}{0pt}]}[180,1.5]
            \chemfig{\charge{135:1pt=$\ominus$}{O}-[:30](=[2]O)-[:-30]}
        \schemestop
        \caption{Synthetic uses of the haloform reaction.}
        \label{fig:haloformSynthetic}
    \end{figure}
    \begin{itemize}
        \item Suppose we have a ketone and want to create a carboxylate.
        \item The haloform reaction selectively cleaves methyl groups, installing an oxygen anion.
        \item The Baeyer-Villiger selectively inserts an ether into the bond to larger groups.
        \begin{itemize}
            \item We can then cleave the larger group via the saponification mechanism.
        \end{itemize}
        \item Note also that this reaction is useful as a \ce{C-C} bond \emph{cleaving} reaction.
        \begin{itemize}
            \item We have even less of these than we do \ce{C-C} bond forming reactions.
            \item The only ones we have are periodate cleavage, ozonolysis, and the two techniques just described here.
        \end{itemize}
    \end{itemize}
    \item Midterm questions and review.
    \item Origin of selectivity for the Beckmann?
    \begin{itemize}
        \item Discusses the transition state.
        \item Goes into the $\sigma^*$ orbital explanation.
        \item Since $\sigma^*$ is higher in energy than $\sigma$ is low, filling $\sigma^*$ breaks the bond.
        \item The external lobe is significantly bigger than the internal (along the bond) lobe.
    \end{itemize}
    \item The more sterically hindered the ketone, the harder it will be to do stuff to it.
    \item \ce{SOCl2} releases \ce{HCl} when there's no pyridine around.
    \begin{itemize}
        \item We're only being graded on the presence of the organic products, though.
    \end{itemize}
    \item In the Wolff-Kirshner, we do need both hydrogens in the hydrazone.
    \begin{itemize}
        \item Modify notes!
    \end{itemize}
    \item If we have some steps in the beginning of a mechanism and some steps in the end with a gap in between, we will get credit for what's on both sides.
    \item DPPA is paired with \ce{NEt3}.
    \begin{itemize}
        \item Modify notes!
    \end{itemize}
    \item Ketal formation happens on ketones and aldehydes only (not carboxylic acid derivatives).
    \begin{itemize}
        \item Modify notes!
    \end{itemize}
\end{itemize}




\end{document}